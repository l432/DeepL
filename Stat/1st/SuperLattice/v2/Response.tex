

%\documentclass[aip,graphicx]{revtex4-1}
\documentclass [sort&compress] {elsarticle}
\usepackage{graphicx}% Include figure files
\usepackage{dcolumn}
\usepackage{color}
%\draft % marks overfull lines with a black rule on the right
\bibliographystyle{elsarticle-num}
\begin{document}
Dear Editor,

We like to express our appreciation to the reviewers for their comments.
We are resubmitting the revised version of the paper number SM--2018--103.
We have studied the comments of the reviewer carefully, and have changed the text according to the comments they
have listed.
Below we refer to each of the reviewer’s comments.




\subsection*{Response to Reviewer \#1 }

\noindent
\textcolor[rgb]{0.00,0.50,1.00}{\textbf{Comment~1.}}
\emph{Concentration of recombination centers such as B-O, Fe-B is more than $10^6$ times smaller than that of matrix atoms. That might reduce the probability of ultrasound-induced tuning of recombination properties of such defects}

\noindent
\textcolor[rgb]{0.51,0.00,0.00}{\textbf{Reply:}}

The point defects can effectively interact with ultrasound waves regardless of their low concentration.
In our opinion, the striking examples are the experimental methods, which are based on analyse of 
 anomalies in speed and attenuation  of ultrasound wave
 and allow to investigate the native and impurity defects \cite{USM:SEYIDOV2016,USM:Zhevstovskikh,USM:Averkin2014,USM:Akatsu2009,USM:Okabe2013}.
For instance, the concentration of vacations in the silicon investigated by ultrasonic experiments has been down to $10^{13}-10^{15}$cm$^{−3}$~\cite{USM:Averkin2014,USM:Okabe2013}.
Besides, it was reported about acoustically induced 
transformation of the antisite defect in GaAs \cite{Ostapenko1994},
dissociation of the Fe--B pairs in silicon \cite{Ostapenko1995},
modification of the emission rate from the donors in Si \cite{Korotchenkov1995},
variation of the Na density in Si--SiO$_2$ system \cite{UST:Medvid},
annealing of the radiative defects \cite{Gorb2010,Podolian2012}
\emph{etc}.


\vspace{1cm}
\noindent
\textcolor[rgb]{0.00,0.50,1.00}{\textbf{Comment~2.}}
\emph{The reason of the degradation is explained by the acoustically induced increase
in the carrier capture coefficient of point or extended defects. More detailed discussion of this point would be interesting as to whether it means ultrasound--induced transformation of the defects or why the change happened.}

\noindent
\textcolor[rgb]{0.51,0.00,0.00}{\textbf{Reply:}}
As was mentioned above the ultrasound can be a tool of the point defect transformation and study.
The acousto--defect interaction depends on a strain which is deals with the disturbance of lattice periodicity. 
In fact, the displacement (with respect to surrounding) of impurity atoms, which produce tensile stress, 
is opposite to the displacement of defect with compressive stress \cite{MirzadeJAP2011,PeleshchakUJF2016}.
In our assumption, the ultrasound loading leads to the increase of the average distance $r$ between the non--equivalent
component of a defect complex.
According to \cite{CDLR:JAP,CDLR:R2}, the carrier capture cross section is proportional to $r^2$.
Therefore the acoustically induced increase in the carrier capture coefficient of point defects is observed.

Dislocations are generally held responsible as a possible source of shunt resistance \cite{Rsh:Breitenstein,TAT:Gopal,Rsh:Baker}.
The impedance value is inversely proportional to the surface area of the dislocation \cite{Rsh:Gopal2003,Rsh:Gopal2004}.
In our opinion, the extended defect oscillation under the influence of an applied ultrasound stress leads to enlarge of its effective surface.
As a result, the dislocation contribution to the recombination current increases and shunt resistance decreases.
The acoustically induced $R_{sh}$ decrease was observed experimentally.


\vspace{1cm}
\noindent
\textcolor[rgb]{0.00,0.50,1.00}{\textbf{Comment~3.}}
\emph{How the surface of Si has been treated before the study of the influence of ultrasound}

\noindent
\textcolor[rgb]{0.51,0.00,0.00}{\textbf{Reply:}}
The special surface treatment after the standard solar cell manufacturing was not carried out.
The piezoelectric transducer was attached to a full--area Al back contact.
The acoustic contact was made by using the gluten ``BF--6''.
The gluten can be dissolved by an ethanol and 
the metal contact surface have been remaining undamaged after transducer detachment.

More detailed information about ultrasound loading was added to the 2.2 item.






\bibliography{olikh}

\end{document}

