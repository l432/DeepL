

%\documentclass[aip,graphicx]{revtex4-1}
\documentclass[aip,jap,preprint]{revtex4-1}
\usepackage{graphicx}% Include figure files
\usepackage{dcolumn}
\usepackage{color}
%\draft % marks overfull lines with a black rule on the right

\begin{document}
Dear editor,

We like to express our appreciation to the reviewers for their comments.
We are resubmitting the revised version of the paper number J--DIS17--17--5066.
We have studied the comments of the reviewer carefully, and have changed the text according to the comments they
have listed.
Below we refer to each of the reviewer’s comments.



\subsection*{Response to Reviewer \#1 }

\noindent
\textcolor[rgb]{0.00,0.50,1.00}{\textbf{Comment~1.}}
\emph{The authors point out multiple times that the ultrasonic loading discussed in the manuscript does not induce permanent annealing effects on the samples and thus the results are fully reversible.
However, there is not data supporting this argument.
Also how long does it take for the system to relax back to its original state and how well?
A cyclic loading-unloading graph may be helpful to clarify this point.}

\noindent
\textcolor[rgb]{0.51,0.00,0.00}{\textbf{Reply:}}
The related graph was added (Fig.2 in the revised version).
It was reported previously\cite{Ostapenko1995,YOlikhTPL2011,Olikh:Ultras,Olikh2011Sem,Ostrovskii2001,OlikhPSS} that time constant of change in the silicon structure parameters under the ultrasound action  did not exceed $2\cdot10^3$~s.
The time constant of change in the silicon parameters after the ultrasound stopping is bigger but does not exceed $1\cdot10^4$~s\cite{OlikhPSS}.
We examined samples state in $\sim24$~hours after ultrasound stopping and found out a full (in experimental error limits) parameter recovery.



\vspace{1cm}
\noindent
\textcolor[rgb]{0.00,0.50,1.00}{\textbf{Comment~2.}}
\emph{It is understandable that the authors would like to "keep the volume of this paper at an rather reasonable size" in the defects analysis section, but I wonder if those important data can be appendiced as supplementary information?}

\noindent
\textcolor[rgb]{0.51,0.00,0.00}{\textbf{Reply:}}
More detailed information about illumination influence was added (Fig.10 and 11 in the revised version, the last paragraph in page 9, the first three paragraphs in page 10).



\vspace{1cm}
\noindent
\textcolor[rgb]{0.00,0.50,1.00}{\textbf{Comment~3.}}
\emph{ Why is g6SC sample missing the 0.2 Wus (W/cm2) experimental data?}

\noindent
\textcolor[rgb]{0.51,0.00,0.00}{\textbf{Reply:}}
Honestly, corresponding experiment was not carried out accidentally.
On the other hand, in our opinion, these experimental data are not a matter of principle.
Furthermore, the radiation damage effect in g6SC is comparatively small (see Table~I) and
all acoustically induced effects (e.g. ideality factor variation) is not expected at low ultrasound intensity.

\vspace{1cm}
\noindent
\textcolor[rgb]{0.00,0.50,1.00}{\textbf{Comment~4.}}
\emph{Can the authors comment on the effect of ultrasound frequency on the sample?}

\noindent
\textcolor[rgb]{0.51,0.00,0.00}{\textbf{Reply:}}
Usually ultrasound influence on the silicon structures had arose with wave frequency\cite{Olikh2011Sem,Olikh:Ultras2016}.
But recent investigation \cite{Olikh:inpress} shown that ultrasound influence efficiency depended on wave type too.
So, transverse waves are more effective instrument then longitudinal waves.
And just transverse waves were using in presented paper.


\vspace{1cm}
\noindent
\textcolor[rgb]{0.00,0.50,1.00}{\textbf{Comment~5.}}
\emph{In figure (a),"The dashed, dot--dashed and dotted lines in (a) are the base, SCR and shunt components of iSC current, respectively." are those calculated curves or experimental? Under what conditions?}

\noindent
\textcolor[rgb]{0.51,0.00,0.00}{\textbf{Reply:}}
All lines in Fig.~1(a) are calculated curves.
Broken lines represent contributions of different component to the shown fitted curve for iSC sample.
The necessary information was added in Fig.~1 caption and in page 2 (right column, 2-nd paragraph from the bottom).



\vspace{1cm}
\noindent
\textcolor[rgb]{0.00,0.50,1.00}{\textbf{Comment~6.}}
\emph{In figure 5 "Axis $|u_\mathtt{D}+u_\mathtt{A}|$ corresponds to $\delta=0^\circ$ case, whereas axis $|u_\mathtt{D}+u_\mathtt{A}|$ corresponds to $\delta=180^\circ$ case.", The first "$|u_\mathtt{D}+u_\mathtt{A}|$" should probably be "$|u_\mathtt{D}-u_\mathtt{A}|$".}

\noindent
\textcolor[rgb]{0.51,0.00,0.00}{\textbf{Reply:}}
Reviewer is quite right.
The corrections were done.

\vspace{1cm}
\noindent
\textcolor[rgb]{0.00,0.50,1.00}{\textbf{Comment~7.}}
\emph{ The language could be polished more. There are some sentences that may cause confusions, for example in the abstract:
"The effects of reactor neutrons and 60Co gamma radiation on ultrasound influence were studied." makes one wonder if the radiation is acting on the Si crystal or the ultrasound;
Some grammar mistakes, such as
". One can recognize, that τn effected mainly by CiOi and vacancy clusters in γ- and neutron-irradiated samples,"}

\noindent
\textcolor[rgb]{0.51,0.00,0.00}{\textbf{Reply:}}
The text was revised.






\bibliography{olikh}

\end{document}

