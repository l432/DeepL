

%\documentclass[aip,graphicx]{revtex4-1}
\documentclass[aip,jap,preprint]{revtex4-1}
\usepackage{graphicx}% Include figure files
\usepackage{dcolumn}
\usepackage{color}
%\draft % marks overfull lines with a black rule on the right

\begin{document}
Dear editor, 

We like to express our appreciation to the reviewers for their comments. 
We are resubmitting the revised version of the paper number J--DIS17--17--5066. 
We have studied the comments of the reviewer carefully, and have changed the text according to the comments they 
have listed. 
Below we refer to each of the reviewer’s comments / suggestions.



\subsection*{Response to Reviewer \#1 }

\textcolor[rgb]{0.00,0.50,1.00}{\textbf{Comment~1.}}
\emph{The authors point out multiple times that the ultrasonic loading discussed in the manuscript does not induce permanent annealing effects on the samples and thus the results are fully reversible.
However, there is not data supporting this argument. 
Also how long does it take for the system to relax back to its original state and how well? 
A cyclic loading-unloading graph may be helpful to clarify this point.}

\textcolor[rgb]{0.51,0.00,0.00}{\textbf{Reply:}}


\emph{1. The reason for selection of power, frequency of ultrasound and duration of processing by ultrasound should be discussed. }

It was reported previously that
(i)~the ultrasound with a frequency in the range of 1--30~MHz was able to affect defects in silicon; \cite{Ostapenko1995,Davletova2008,Davletova2009,Gorb2010,Ostapenko1999,Podolian2012,Parchinskii2006,Pashaev2014,Mirsagatov}
(ii)~the acoustic vibration, which had the intensity $\geq0.5$~W/cm$^2$, could led to an unreversible  modification of properties of silicon layered structures without radiative defects;\cite{Davletova2008,Davletova2009,Pashaev2014,Mirsagatov,Zaveryukhin2002,Vlasov2009}
(iii)~the time constant of change in the silicon structure parameters under the ultrasound action  did not exceed $2\cdot10^3$~s. \cite{Ostapenko1995,YOlikhTPL2011,Olikh:Ultras,Olikh2011Sem,Ostrovskii2001,OlikhPSS}
We used ultrasound at a frequency of $8.4$~MHz and an intensity of $< 0.3$~ W/cm$^2$ to investigate the reversible variation of the silicon Schottky barrier structure characteristic.
Besides the $I$--$V$ characteristics were measured in an hour after the US loading start.
The time interval between consecutive US loading was larger than one day.

The related information has been added (second paragraph in "Experimental and calculation details").

\vspace{1cm}
\emph{2. Frequency of the ultrasound is 8.4 MHz, which is equivalent to wavelength of about 36 m, which exceeds thickness of 250 micrometer of the silicon processed by the ultrasound. Some discussion about absorbency of the ultrasound by the Si would be interesting.}

The velocity of longitudinal acoustic wave in silicon is $8430$~m$/$s.\cite{Cheeke}
Therefore the wavelength of the 8.4~MHz ultrasound is about $1000~\mu$m.
The ultrasound intensity was estimated by using second piezoelectric transducer, which placed on opposite end of sample (not showed on Fig.~1).
The measurements, which have been carried out both with and without the semiconductor sample, showed the occurrence of sample ultrasound attenuation.
However the exact determination of the ultrasound attenuation coefficient was outside of work objective.
The additional evidence of absorbency of the ultrasound by the Si is the structure parameters modification.


\subsection*{Response to Reviewer \#2}

\emph{1. By means of the equation (7) the flat band barrier height is determined. However, the comparison with theoretical value of barrier height is not reported. It is necessary to give some detail on this topic.}

According to Rhoderick and Williams \cite{Rhoderick1988}, the flat band barrier height can be expressed as
\begin{equation}
\label{Fbf}
    q\Phi_{b,F}=\theta(\phi_m-\chi_s)+(1-\theta)(E_g-\phi_0)
\end{equation}
where $\theta=[1+(qD_s\delta)/(\varepsilon_0\varepsilon_i)]^{-1}$, $\phi_m$ is the metal  work  function, $\chi_s$ is  the  electron  affinity of  the  semiconductor (4.05~eV for Si), $\phi_0$ is the neutral  level  of  the interface  states, $D_s$  is  the  interface  state  density, $\delta$ and $\varepsilon_i$ are the thickness and permittivity of the oxide interface  layer.
Eq.~(\ref{Fbf}) does not take into account the barrier image-force lowering.
Mo work function depends on a crystallographic plane as well as on a manufacturing condition  and ranging from 4.53 to 4.95~eV for Mo/Si contacts.\cite{MoWF2001,MoWF2002}
Therefore $\Phi_{b,F}$ tends to the Schottky-Mott limit $q\Phi_{b,F}^{\text{SM}}=\phi_m-\chi_s=(0.48\div0.90)$~eV as $D_s\rightarrow0$ and to the Badeen limit $q\Phi_{b,F}^{\text{B}}=E_g-\phi_0$ as $D_s\rightarrow\infty$.
The determinated values $(0.82\div0.85)$~V are within theoretical range.

The related information has been added in page 4 (left column, 2-nd paragraph from the top).


\vspace{1cm}
\emph{3. The double Gaussian distribution show the presence of two different barrier heights: $\Phi_{b,1}$ and $\Phi_{b,2}$ . Which is the physics explanation of this behavior? Can be attributed at particular defects? It is important to report some details in the text.}

According to Jiang \emph{et al}.\cite{Jiang:DGJap}, the Gaussian distribution with a little contribute (the $\rho$ value) is caused by patches that deal with the incomplete and inhomogeneous diffusion process of metal atoms.
Therefore $\overline{\Phi}_{b,2}$ and $\sigma_{\Phi,2}$ can be attributed at these defects.
The $\overline{\Phi}_{b,2}$ value indicates that the large interface state density and the low neutral  level are inherent in that sort of patches --- see Eq.~(\ref{Fbf}).
At the same time, $\overline{\Phi}_{b,1}$ deals with uniform region and $\sigma_{\Phi,1}$ can be attributed at other nature patches (processing
remnants, surface roughness, an uneven doping profile, crystal defects, and grain boundaries \emph{etc}.\cite{Gammon2013}).

The related information has been added in page 5 (left column, 3-rd paragraph from the top).

\vspace{1cm}
\emph{2. It is demonstrated that there is a distribution of different patches with varies barrier height, but it is not mentioned the theoretical value of barrier height and its position in the Gaussian distribution. The values obtained of double Gaussian distribution include the ideal value?}

The theoretical value of barrier height is determined for the uniform contact.
Therefore the mean value $\overline{\Phi}_{b,1}$ can be compared with the ideal value.
On the other hand, Rhoderick and Williams \cite{Rhoderick1988} showed that the barrier height was reduced from the flat band value by an amount which was proportional to the maximum electric field in the semiconductor.
The extracted relation $\Phi_{b,F}>\overline{\Phi}_{b,1}$ is relevant to the predicted tendency.

The related information has been added in page 5 (left column, 3-rd paragraph from the top).


\vspace{1cm}
\emph{4. In the page 4 is reported an explanation of the reversible Schottky barrier height variation due to ultrasound action. The paper is based on the effect of ultrasound on the interface characteristics; it can be useful to insert a schematic (such as band diagram or something else) which tries to better explain the hypothesis.}

The Fig.~9 has been added in page 6.
The last three paragraphs in "Result and discussion" (page 6) have been rebuilt.

\vspace{1cm}
\emph{5. The equation (10) is related to the reference number 38, but in this paper is not reported this equation. It is necessary to change the reference or to integrate it with appropriate one.}

The reference has been integrated.
The equation (10) is previously used \emph{e.g.} by Tung\cite{Tung:MSE} (Eq.~(5.3.4)) or by Iucoolano \emph{et al}.\cite{Iucolano2007JAP} (Eq.~(6)).



\bibliography{olikh}

\end{document}

