\chapter{\MakeUppercase{Передумови та особливості використання активного ультразвука}}



\section{Ефекти впливу ультразвука на мікроелектронні структури та \\ матеріали \label{Oglyad}}
\cite{Si_Auger}

In the literature, there are several models that describe the current--voltage ($I-V$) characteristics of the solar cells (SCs).
These models contain some parameters, which reflect the processes within the structures and are related to the main characteristics of the photovoltaic conversion.
So single diode model with three parameters has been used to represent the SC static characteristic because of simplicity:
\begin{equation}
\label{eqIVs}
    I=I_{0}\left[\exp\left(-\frac{qV}{nkT}\right)-1\right]-I_{ph}\,,
\end{equation}
were
$I_0$ is the saturation current,
$n$ is the diode ideality factor,
$I_{ph}$ is the total current generated by solar cell.
The ideality factor value indicates about a defect related recombination and directly determines open--circuit magnitude:
\begin{equation}
\label{eqVoc}
    V_{oc}=\frac{nkT}{q}\ln\left(\frac{I_{ph}}{I_0}+1\right)\,.
\end{equation}
Eq.~(\ref{eqIVs}) does not take into account a leakage current, a series losses of load current.
Besides, the widely used double diode model is developed by considering the effect of recombination current loss
in the depletion region \cite{2Diod:Ishaque,2Diod:Buhler,Breitenstein2013}:
\begin{equation}
\label{eqIVd}
    I=I_{01}\left[\exp\left(-\frac{q(V-R_sI)}{kT}\right)-1\right]
      + I_{02}\left[\exp\left(-\frac{q(V-R_sI)}{nkT}\right)-1\right]
      +\frac{V-R_sI}{R_{sh}}
      -I_{ph}\,,
\end{equation}
where
first term closely related to the recombination in the quasi--neutral region,
second term describes the overall space charge region (SCR) recombination,
$R_s$ and $R_sh$ are the series and shunt resistance, respectively.
In this case the relationship between the ideality factor and SC characteristics is more complicated.
Певні приклади щодо взаємозв'язку n з напругою холостого ходу та fill factor  рамках дво-діодної моделі можна знайти в \cite{Olikh2018SM}.
Typically, the value of the ideality factor ranges from 1 to 2 for real devices and depends on ambient conditions and recombination center parameters,
including the concentration of traps \cite{n2_Beier,n2McIntosh,n2Kaminski,HAMEIRI2013251,Heide}.
This makes the ideality factor an important parameter that can be used to describe the electrical behavior of photovoltaic devices and characterize the recombination in SCs \cite{Duan}.

One of the main obstacles of such a convenient and express method developing is the multiparameter relationship between
the $n$ value and the concentration of recombination centers.
This paper attempts to get over these difficulties by the simulation of $I-V$ characteristic of silicon solar cells,
the determination of ideality factor, and the study of  $n$ value depending on simulation parameters.
In contrast to the previous paper \cite{Olikh2019SM} in this case the $n^+$--$p$--$p^+$-structure,
which is closer to the real SC, is under consideration.
Additionally the base thickness is known \cite{Sach_d,FeB:Schmidt} to affect SC efficiency;
therefore the paper considers the influence of this parameter on the ideality factor value.

The paper focuses on the case when the main recombination centers are the iron related defects.
On the one hand, the iron atoms  are one of the most common as well as the most harmful impurities in silicon solar cell.
On the other hand, the $\mathrm{Fe}_i\mathrm{B}_s$ pairs can be readily dissociated by illumination \cite{FeB:Schmidt};
the association reaction can take place when exposed in darkness for ten minutes \cite{FeB:kinetic}.
Such a change in recombination center state should lead to a change in a ideality factor value,
that is easy to obtain experimentally and to use to SC characterization.
Therefore, the paper also pays attention to dependencies of $n$ value change.

\section{Simulation details}


Its main parts are the emitter layer with thickness $d_n$, the base with hole conductivity and thickness $d_p$
and the $p^+$ layer with thickness $d_{BSF}$ intended to back surface field (BSF) creation.
BSF-layer is designed to increase the photovoltaic converter efficiency by reducing the losses concerned with surface recombination
and such structure is widely used for both manufacturing of real solar cells and modeling \cite{SCAPSuseSi4,SCAPSuseSi1,SCAPSuseSi5}.

The material of all layers was assumed to be monocrystalline silicon.
The temperature dependencies of bandgap was calculated according to P\"assler equations \cite{Pasler}.
The bandgap narrowing, thermal carrier velocities,
and  free carrier effective mass
were taken from Yan and Cuevas \cite{EgNarrow}, Green \cite{Nc:Green}, and O'Mara \emph{et al.} \cite{OMara}, respectively.
Data from Couderc~\emph{et al.} \cite{Si_ni_Couderc} were used to evaluate intrinsic carrier density and density of states effective masses.
The temperature dependencies carrier mobilities were described by Klaassen's theory \cite{KLAASSEN953,Hull}.

It was assumed uniform doping with phosphorus (the emitter layer, concentration $N_\mathrm{D}$)
and boron (base and BSF--layer, concentrations $N_\mathrm{A}$ and $N_\mathrm{BSF}$, respectively).

The following recombination processes were taken into account:
i)~the outside surface recombination with electron and hole velocities $10^3$~cm/s;
ii)~the intrinsic recombination (radiative band--to--band and Auger with coefficients,
which depend on temperature and doping level according
to Nguyen~\emph{et al.} \cite{Si_BtB} and Altermatt~\emph{et al.} \cite{Si_Auger});
iii)~the Shockley–-Read-–Hall (SRH) recombination.

In the last case, as the base and BSF--layer uniform contaminant, iron is assumed to be in concentration $N_\mathrm{Fe}$.
It is well known that the iron atom locates in lone interstitial lattice position in silicon ($\mathrm{Fe}_i$) or
interacts with ionized acceptors and combines into $\mathrm{Fe}_i\mathrm{B}_s$ pair.
The two cases were under consideration.
In the first one, uniformly distributed $\mathrm{Fe}_i$ with concentration $N_\mathrm{Fe}$ was assumed.
Such case is realizing under constant illumination or immediately after its termination.
The temperature independent donor level $E_{\mathrm{Fe}_i} = E_V+0.394$~eV \cite{Rein2,MurphyJAP2011,Kohno}
and electron $\sigma_{n,{\mathrm{Fe}}}=3.47\times10^{-15}T^{-1.48}$~m$^2$ and
hole $\sigma_{p,{\mathrm{Fe}}}=4.54\times10^{-20}\exp\left(-\frac{0.05}{kT}\right)$~m$^2$ capture cross-sections \cite{ROUGIEUX2018,Paudyal}
are associated with $\mathrm{Fe}_i$.
In the second one, $\mathrm{Fe}_i$ and $\mathrm{Fe}_i\mathrm{B}_s$ were coexistence.
Their concentration were non--uniformly distributed through base and BSF--layer.
The more details are presented elsewhere \cite{Olikh2019SM} and the representative

Such case is realizing under dark equilibrium condition.
The $\mathrm{Fe}_i\mathrm{B}_s$ is amphoteric defect and donor level $E_{\mathrm{FeB}}^\mathrm{D}= E_V+0.10$~eV,
$\sigma_{n,{\mathrm{FeB}}}^\mathrm{D}=4\times10^{-17}$~m$^2$,
$\sigma_{p,{\mathrm{FeB}}}^\mathrm{D}=2\times10^{-18}$~m$^2$
and acceptor level $E_{\mathrm{FeB}}^\mathrm{A}= E_C-0.26$~eV,
$\sigma_{n,{\mathrm{FeB}}}^\mathrm{A}=5.1\times10^{-13}T^{-2.5}$~m$^2$,
$\sigma_{p,{\mathrm{FeB}}}^\mathrm{A}=3.32\times10^{-14}\exp\left(-\frac{0.262}{kT}\right)$~m$^2$
\cite{Istratov1999,Rein2,MurphyJAP2011,ROUGIEUX2018,Paudyal,FeB:kinetic}
are used in simulation.

The dark forward dark $I-V$ characteristic were generated by
one-dimensional code SCAPS~3.3.08 \cite{SCAPS1,SCAPS2}
over a voltage range up to $0.45$~V with step 0.01~V.
This software is widely applied in modeling various solar cells \cite{SCAPSuse1,SCAPSuse2,SCAPSuse3,SCAPSuse5,SCAPSuseSi1,SCAPSuseSi4,SCAPSuseSi3},
silicon based devices including \cite{SCAPSuseSi1,SCAPSuseSi3,SCAPSuseSi4}.
Thus, the varied parameters were the boron concentrations in the base, iron concentration, base thickness and temperature.
Taking into account two defect configuration, 15048 structures were simulated.
The simulated $I-V$ characteristic were fitted by following equation:
\begin{equation}
\label{eqIV}
    I=I_{01}\left[\exp\left(-\frac{qV}{kT}\right)-1\right]+ I_{02}\left[\exp\left(-\frac{qV}{nkT}\right)-1\right]\,.
\end{equation}
Eq.~(\ref{eqIV}) corresponds to the dark double diode model with neglected both series and shunt resistances.
The first diode represents the ``ideal'' diode, describing the so--called diffusion current characterized by a saturation current $I_{01}$,
and the second diode is the so--called recombination current, characterized by the saturation current $I_{02}$ and ideality factor $n$ \cite{Breitenstein2013}.
$n$, $I_{01}$, and $I_{02}$ were taken as fitting parameters and the meta--heuristic method IJAVA \cite{IJAVA} was used.



In the case of lone unpaired $\mathrm{Fe}_i$ the following value were calculated:
$n_\mathrm{Fe}^\mathrm{srh}$ is the ideality factor if the SRH recombination was taken into account only;
$n_\mathrm{Fe}$ is the ideality factor if the both SRH recombination and intrinsic recombination were allowed;
$\delta n_\mathrm{Fe}^\mathrm{srh}=n_\mathrm{Fe}^\mathrm{srh}-n_\mathrm{Fe}$ characterizes the influence of intrinsic recombination on ideality factor value.
In the case of $\mathrm{Fe}_i\mathrm{B}_s$ and $\mathrm{Fe}_i$ coexistence the
$n_\mathrm{FeB}^\mathrm{srh}$, $n_\mathrm{FeB}$, $\delta n_\mathrm{FeB}^\mathrm{srh}=n_\mathrm{FeB}^\mathrm{srh}-n_\mathrm{FeB}$ were calculated (indices had the same meaning).
Besides, the change of the ideality factor after $\mathrm{Fe}_i\mathrm{B}_s$ association $\delta n_\mathrm{Fe-FeB}=n_\mathrm{Fe}-n_\mathrm{FeB}$ was calculated as well.


\section{Results and discussion}



Secondly, the unpaired iron atom concentration can be big enough in the case of $\mathrm{Fe}_i\mathrm{B}_s$ and $\mathrm{Fe}_i$ coexistence as well
and it increases with temperature rise and decrease in doping level.
For example, the $\mathrm{Fe}_i$ concentration in the quasi--neutral region of base
reaches 23 (or 3) percent of $N_\mathrm{Fe}$ at $T=340$~K and $N_\mathrm{A}=10^{15}$~cm$^{-3}$ (or $10^{16}$~cm$^{-3}$).
That is, under these conditions, the concentration of unpaired iron atoms in the dark and $N_\mathrm{Fe}=10^{13}$~cm$^{-3}$ is larger
than ones under illumination and $N_\mathrm{Fe}=10^{11}$~cm$^{-3}$.
Finally,
as only ionized iron $\mathrm{Fe}_i^+$ (unlike to neutral iron $\mathrm{Fe}_i^0$) actively takes  part in the SRH recombination,
then these processes efficiently occur at $x\geq0.6W_p$ (where $W_p$ is the SCR depth).
And the area of processes, which determines ideality factor value,
shifts away from the $p-n$ junction with increase in doping level.

Several determinants must be taken into account when analyzing the dependencies of the ideality factor on the temperature and boron concentration of boron.
Namely.

i)~The occurring of hole on the $\mathrm{Fe}_i$ level, which determines the recombination efficiency.
Accordingly to the Fermi--Dirac statistics,
the probability of hole occupation in  a non--degenerate $p-$type semiconductor with full acceptor depletion
can be expressed as

\begin{equation}
\label{eqfp}
 f_p=\frac{1}{1+\frac{N_V(T)}{N_\mathrm{A}}\exp\left(\frac{E_V-E_{\mathrm{Fe}_i}}{kT}\right)}\,.
\end{equation}

It has been shown earlier \cite{Olikh2018SM}  the $f_p(T,N_\mathrm{A})$ dependence is generally similar to observed dependence of ideality factor dependence.
In particular, if $f_p$ is close to one (high $N_\mathrm{A}$ value and low temperature), this dependence changes slowly,
If $N_\mathrm{A}$ decreases or (and) $T$ increases,
the level is filled by electron in a sufficiently narrow range of arguments,
the SRH recombination ceases,


ii)~The balance of the defect related recombination and the intrinsic recombination.
SRH recombination generally causes increase in ideality factor value;
if the defect related recombination is dominant, the value often reported in publications is $n=2$.
The radiative band--to--band and Auger recombinations are enhanced by the increase in both free charge carrier concentration
(doping level) and temperature   \cite{Si_BtB,Si_Auger}).
In this case, the ideality factor reduces and the values $\delta n_\mathrm{Fe}^\mathrm{srh}$
and $\delta n_\mathrm{FeB}^\mathrm{srh}$ become nonzero.









In turn, the $\delta n_\mathrm{Fe-FeB}$ value depends as well on the iron concentration
at neighborhood of $n_\mathrm{FeB}>n_\mathrm{Fe}$.
As a result, $\delta n_\mathrm{Fe-FeB}$, along with $n_\mathrm{Fe}$ and $n_\mathrm{FeB}$,
can be used to estimate the impurity concentration by parameters of $I-V$ characteristic.


\section{Conclusion}

The diode ideality factor of silicon $n^+-p-p^+$ structure with iron contamination has been studied via computer simulation.
The data used in the simulations were the following.
The iron concentration ranged from $10^{10}$ to $10^{13}$~cm$^{-3}$,
the acceptor doping level --- from $10^{15}$ to $10^{17}$~cm$^{-3}$,
the temperature --- from $290$ to $340$~K,
and the base thickness --- from $150$ to $240$~$\mu$m.
It has been shown that the temperature and doping level dependencies of the ideality factor value
are mainly determined by a hole occurring on the $\mathrm{Fe}_i$ level.
The $n$ dependence on iron concentration is a monotonic function.
Additionally, not only defect concentration but also its location influences on  ideality factor value.
The intrinsic recombination causes the decrease in the ideality factor value at a high temperature and doping level as well as at a low iron concentration.
It has also been found that the base thickness influences on ideality factor if it exceeds the minority carrier diffusion length.
The increase in base thickness leads to decrease in $n$ value.
The investigation has revealed that the ideality factor in $\mathrm{Fe}_i\mathrm{B}_s$ and $\mathrm{Fe}_i$ coexistence case
can exceed ones in lone unpaired $\mathrm{Fe}_i$ case.
The ideality factor change after $\mathrm{Fe}_i\mathrm{B}_s$ dissociation can be used to contaminant concentration evaluation.

