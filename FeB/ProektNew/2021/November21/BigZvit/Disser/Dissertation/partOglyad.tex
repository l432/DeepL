\chapter{\MakeUppercase{Передумови та особливості використання активного ультразвука}}



\section{Ефекти впливу ультразвука на мікроелектронні структури та \\ матеріали \label{Oglyad}}
\cite{Olikh2019SM,Heide,Duan,n_CharGaN,n_CharSemic,n_CharPhysRevAppl,MachLean_RevModPhys,
MachLeanJAP,MachLeanPPV,SCRev2015,SCRev2020,FeB:Schmidt,IronSC,FeBLight2,FeB_Zong,
MurphyJAP2011,FeB:kinetic,FeBAssJAP2014,FeBKin2019,
ostapenko2002,Savkina2013,Olikh2018JAP,Davletova2008,Olikh2020JEM,Olikh:Ultras,OlikhJAP,Teterkin2009,
USImplant:JVacSci,RomanyukSST,Kalem2000,US:ZnOfilm,LaineIEEEPV2016,FeB:Vahanissi,Teimuraz2014JAP,
Ostapenko1995,Si:Absorb,GreenOptic}:

%\cite{Sachenko2016,TAYYIB201221,FeB_Wilson,FeB_Walz,FeB_Zong,Breitenstein2013,SINKE2019,Wijaranakula,Narland,Sakauchi,Si_Auger,FeBLight,FeBLight2,FeB:Schmidt,FeBLight3,FeBCoDop,FeBDecay}

In the literature, there are several models that describe the current--voltage ($I-V$) characteristics of the solar cells (SCs).
These models contain some parameters, which reflect the processes within the structures and are related to the main characteristics of the photovoltaic conversion.
So single diode model with three parameters has been used to represent the SC static characteristic because of simplicity:
\begin{equation}
\label{eqIVs}
    I=I_{0}\left[\exp\left(-\frac{qV}{nkT}\right)-1\right]-I_{ph}\,,
\end{equation}
were





