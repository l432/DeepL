\documentclass{WileyMSP-template}

\usepackage[square, super, comma]{natbib}
\usepackage{natmove}
\usepackage{setspace}

\begin{document}

\pagestyle{fancy}
\rhead{\includegraphics[width=2.5cm]{vch-logo.png}}


\title{Please insert your title here}

\maketitle

% Author: Please give full first and last names for authors and include * after the name of all corresponding authors

\author{Author One}
\author{Author Two}
\author{Author Three*}

% Dedication
\dedication{Optional dedication here. If no dedication is required, please leave blank}


% Affiliations: Please provide adacemic titles (Prof. or Dr.) for all authors where applicable, and include an institutional email address for all corresponding authors
\begin{affiliations}
A. N. Author, A. N. O. Author\\
Address\\
Email Address:

A. N. O. Author\\
Address

\end{affiliations}


% Keywords: Please provide a minimum of three and a maximum of seven keywords, separated by commas
\keywords{Keyword 1, Keyword 2, Keyword 3}


% Abstract should be written in the present tense and impersonal style (i.e., avoid we), and be at most 200 words long
\begin{abstract}

Please insert your abstract here

\end{abstract}

% Text: Please use section headings and subheadings as specified below. For communications, all section headings apart from Experimental Section should be removed
% Please make the first reference to a display item bold: \textbf{Figure 1}
% Do not abbreviate Figure, Equation, etc.; display items are always singular, i.e., Figure 1 and 2.
% Equations are always singular, i.e., Equation 1 and 2, and should be inserted using the {equation} environment, not as graphics
% Please do not use footnotes in the text, additional information can be added to the Reference list.


\section{First Section}


\subsection{First Subsection}


\subsubsection{First Sub Subsection}


\threesubsection{First lowest-level subsection}
Example of citations~\cite{geiger2012we,lees2010theoretical}. The authors~\cite{urmson2008autonomous} stated that...

\section{Conclusion}

% Experimental section

\section{Experimental Section}
\threesubsection{First part of experimental section}\\
\threesubsection{Second part of experimental section}\\


\medskip
\textbf{Supporting Information} \par %Please delete the Suppporting Information statement if it is not applicable. Please supply Supporting Information in another file. Supporting information should not be provided in .tex format
Supporting Information is available from the Wiley Online Library or from the author.



% Acknowledgements
\medskip
\textbf{Acknowledgements} \par %delete if not applicable))
Please insert your acknowledgements here

% References
\medskip

% Use the following code if you wish to generate your bibliography with BibTeX;
% replace the string "MSP-template" below with the name(s) of
% the BibTeX data base(s) you want to use.
% The resulting bibliography-output (the content of the .bbl file)
% must be pasted back into this file before submission.
% Please also include your BibTeX data base file(s) in your submission
% so that we can re-run BibTeX if necessary.
%
\bibliographystyle{MSP}
\bibliography{mybib}

%\begin{thebibliography}{1}
%	\providecommand{\url}[1]{\texttt{#1}}
%	\providecommand{\urlprefix}{URL }
%	
%	\bibitem{urmson2008autonomous}
%	C.~Urmson, J.~Anhalt, D.~Bagnell, C.~Baker, R.~Bittner, M.~Clark, J.~Dolan,
%	D.~Duggins, T.~Galatali, C.~Geyer, et~al.,
%	\newblock \emph{Journal of Field Robotics} \textbf{2008}, \emph{25}, 8 425.
%	
%	\bibitem{lees2010theoretical}
%	J.~Lees-Miller, J.~Hammersley, R.~Wilson,
%	\newblock \emph{Transportation Research Record: Journal of the Transportation
%		Research Board} \textbf{2010}, , 2146 76.
%	
%	\bibitem{geiger2012we}
%	A.~Geiger, P.~Lenz, R.~Urtasun,
%	\newblock In \emph{IEEE Conference on Computer Vision and Pattern Recognition
%		(CVPR) 2012}. IEEE, \textbf{2012} 3354--3361.
%	
%\end{thebibliography}


% Figures/tables and captions
% Permission statements are required for all figures reproduced or adapted from previously published articles/sources. Please also ensure that all necessary permissions to reproduce images have been received
% Please remove these statements for original figures


\begin{figure}
  \includegraphics[width=\linewidth]{placeholder-image.png}
  \caption{Figure 1 caption goes here. Reproduced with permission.\textsuperscript{[Ref.]} Copyright Year, Publisher. }
  \label{fig:boat1}
\end{figure}

\begin{figure}
  \includegraphics[width=\linewidth]{placeholder-image.png}
  \caption{Figure 2 caption goes here. Reproduced with permission.\textsuperscript{[Ref.]} Copyright Year, Publisher.}
  \label{fig:boat1}
\end{figure}

\begin{figure}
  \includegraphics[width=\linewidth]{placeholder-image.png}
  \caption{Figure 3 caption goes here. Reproduced with permission.\textsuperscript{[Ref.]} Copyright Year, Publisher.}
  \label{fig:boat1}
\end{figure}

\begin{table}
 \caption{Table 1 caption}
  \begin{tabular}[htbp]{@{}lll@{}}
    \hline
    Description 1 & Description 2 & Description 3 \\
    \hline
    Row 1, Col 1  & Row 1, Col 2  & Row 1, Col 3  \\
    Row 2, Col 1  & Row 2, Col 2  & Row 2, Col 3  \\
    \hline
  \end{tabular}
\end{table}


% Please provide Biographies and photos for Essays, Feature Articles, Progress Reports, Reviews, and Perspectives for those authors who should be highlighted  
% These should be at most 100 words long
% For other article types this section can be removed
% Photographs should be 40mm broad and 50 mm high

\begin{figure}
  \includegraphics{bio-placeholder.jpg}
  \caption*{Biography}
\end{figure}

\begin{figure}
  \includegraphics{bio-placeholder.jpg}
  \caption*{Biography}
\end{figure}

\begin{figure}
  \includegraphics{bio-placeholder.jpg}
  \caption*{Biography}
\end{figure}

\begin{figure}
  \includegraphics{bio-placeholder.jpg}
  \caption*{Biography}
\end{figure}


% Table of contents entry should be 50 - 60 words long
% Image should be 55 mm broad and 50 mm high or 110 mm broad and 20 mm high


\begin{figure}
\textbf{Table of Contents}\\
\medskip
  \includegraphics{toc-image.png}
  \medskip
  \caption*{ToC Entry}
\end{figure}


\end{document}
