

%\documentclass[aip,reprint]{revtex4-1}
\documentclass[aip,jap,preprint]{revtex4-1}
\usepackage{graphicx}% Include figure files
\usepackage{dcolumn}
\usepackage{color}
%\draft % marks overfull lines with a black rule on the right

\begin{document}
Dear editor,

We like to express our appreciation to the reviewers for their comments.
We are resubmitting the revised version of the paper number JAP21--AR--DIS2022--05281.
We have studied the comments of the reviewer carefully, and have changed the text according to the comments they
have listed.
The location of revisions is pointed by blue color in ``MarkedManuscript.pdf''.
Below we refer to each of the reviewer’s comments.


\subsection*{Response to Reviewer \#1 }

\noindent
\textcolor[rgb]{0.00,0.50,1.00}{\textbf{Comment~1.}}
\emph{The English expression is a little awkward in places, but not to the extent that it negatively affects the readability, in my opinion.}

\noindent
\textcolor[rgb]{0.51,0.00,0.00}{\textbf{Reply:}}
The text was revised.


\subsection*{Response to Reviewer \#2 }

\noindent
\textcolor[rgb]{0.00,0.50,1.00}{\textbf{Comment~1.}}
\emph{In Fig. 4, the numbers identifying the different curves are missing.}

\noindent
\textcolor[rgb]{0.51,0.00,0.00}{\textbf{Reply:}}
The curves' numbers were added.
We apologize for the inattention.


\vspace{1cm}
\noindent
\textcolor[rgb]{0.00,0.50,1.00}{\textbf{Comment~2.}}
\emph{Figure 6 is not so relevant, and could be given as supplementary information}

\noindent
\textcolor[rgb]{0.51,0.00,0.00}{\textbf{Reply:}}
Figure~6 was transferred to the Supplementary material,
the text was revised (page~1, column~1, paragraph~3).

\textcolor[rgb]{1.00,0.00,0.00}{\textbf{!!!!}}

\subsection*{Response to Reviewer \#3 }

\noindent
\textcolor[rgb]{0.00,0.50,1.00}{\textbf{Comment~1.}}
\emph{Page 1, column 2, line 1: The lattice deformation amplitude should have a unit.}

\noindent
\textcolor[rgb]{0.51,0.00,0.00}{\textbf{Reply:}}
The ``relative deformation'' was meant.
The term ``strain'' is used in the revised version.


\vspace{1cm}
\noindent
\textcolor[rgb]{0.00,0.50,1.00}{\textbf{Comment~2.}}
\emph{Page 1, column 2, paragraph 3: The origin of the iron in the diodes should be explained. How are the diodes contaminated with iron? }

\noindent
\textcolor[rgb]{0.51,0.00,0.00}{\textbf{Reply:}}
More

\vspace{1cm}
\noindent
\textcolor[rgb]{0.00,0.50,1.00}{\textbf{Comment~3.}}
\emph{Page 1, column 2, paragraph 5: How was the excess carrier density, which is induced by the LED illumination, estimated? This point should be explained. }

\noindent
\textcolor[rgb]{0.51,0.00,0.00}{\textbf{Reply:}}
The excess carrier density was estimated by using open-circuit voltage value $V_\mathrm{OC}$.
According to Sachenko \emph{et. al.}\cite{JAPSach}
\begin{equation}
  \Delta n=-\frac{n_0}{2}+\sqrt{\frac{n_0^2}{4}+n_i\exp\left(\frac{qV_\mathrm{OC}}{kT}\right)}
\end{equation}
where
$n_0$ is the equilibrium electron concentration,
$n_i$ is the intrinsic electron concentration.

The short information was added to the revised manuscript (page~1, last paragraph; page~2, first paragraph).


\vspace{1cm}
\noindent
\textcolor[rgb]{0.00,0.50,1.00}{\textbf{Comment~4.}}
\emph{Figure 2: Short circuit current has the unit $\mu$A not $\mu$m.}

\noindent
\textcolor[rgb]{0.51,0.00,0.00}{\textbf{Reply:}}
The reviewer is quite right.
The graph was corrected.


\vspace{1cm}
\noindent
\textcolor[rgb]{0.00,0.50,1.00}{\textbf{Comment~5.}}
\emph{Page 2, column 2, paragraph 1: The materials doping level is given on page 1 with 10 Ohm cm. This is about 1.4e15cm$^{-3}$ and not 1.4e16cm$^{-3}$ as stated here.}

\noindent
\textcolor[rgb]{0.51,0.00,0.00}{\textbf{Reply:}}
The reviewer is quite right.
It was a slip.
The correction is done.


\vspace{1cm}
\noindent
\textcolor[rgb]{0.00,0.50,1.00}{\textbf{Comment~6.}}
\emph{Equation 7: The unit of the pre-factor is missing. }

\noindent
\textcolor[rgb]{0.51,0.00,0.00}{\textbf{Reply:}}
The Equation was corrected.

\vspace{1cm}
\noindent
\textcolor[rgb]{0.00,0.50,1.00}{\textbf{Comment~7.}}
\emph{Page 3, column 1, paragraph 3: The obtained iron concentration is compared to results obtained from diffusion length measurements. There should be a reference to these measurements or more details should be given.}

\noindent
\textcolor[rgb]{0.51,0.00,0.00}{\textbf{Reply:}}
The diffusion length before ($L_{n0}$) and after ($L_{n1}$) pair dissociation was measured using spectral dependencies of short circuit current\cite{LnIscMethod}.
Then the iron concentration was determined by using Zoth and Bergholz\cite{FeB_Zong} equation:
\begin{equation}
  N_\mathrm{Fe}(\mathrm{cm}^{-3})=1.06\cdot10^{16}\left(\frac{1}{L_{n1}^2}-\frac{1}{L_{n0}^2}\right)(\mu\mathrm{m}^{-2})
\end{equation}

The short information was added to the revised manuscript.

\vspace{1cm}
\noindent
\textcolor[rgb]{0.00,0.50,1.00}{\textbf{Comment~8.}}
\emph{Figure 4: The numbers of the curves given in the caption are not included in the graphs. This must be improved otherwise the figures cannot be understood. Each graph should also be marked by a letter. What is "G" at the x-axis of the inset? This should be explained. }


\noindent
\textcolor[rgb]{0.51,0.00,0.00}{\textbf{Reply:}}
The graphs were revised.
The curves' numbers were added.
Each graph was marked by a letter.
``G'' was changed by ``Will'' (radiation intensity of halogen lamp).
We apologize for the inattention.


\vspace{1cm}
\noindent
\textcolor[rgb]{0.00,0.50,1.00}{\textbf{Comment~9.}}
\emph{Figure 5: This plot is confusing and must be revised. What is the main statement of this figure? What are the differences between the samples and why do the results change from sample to sample? The axes should have the same scaling. What are the light blue bars in (a)? }

\noindent
\textcolor[rgb]{0.51,0.00,0.00}{\textbf{Reply:}}
The

\vspace{1cm}
\noindent
\textcolor[rgb]{0.00,0.50,1.00}{\textbf{Comment~10.}}
\emph{Main Problem}
\emph{The impact of ultrasound on the iron-boron-pair reaction was first reported by Ostapenko and Bell in 1995 [1]. They found that the iron boron pairs dissociate due to an ultrasound treatment. This is in contradiction to the findings, which are reported herein. In the contribution under review it is found that the association of the FeB pairs is enhanced by the ultrasound treatment. This contradiction must absolutely be discussed by the authors otherwise the manuscript cannot be published.}
\emph{
[1] S. S. Ostapenko and R. E. Bell, "Ultrasound stimulated dissociation of Fe-B pairs in silicon," J. Appl. Phys., vol. 77, no. 10, p. 5458, 1995.
 }

\noindent
\textcolor[rgb]{0.51,0.00,0.00}{\textbf{Reply:}}
The


\bibliography{olikh}

\end{document}

