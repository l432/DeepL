

%\documentclass[aip,reprint]{revtex4-1}
\documentclass[sn-mathphys]{sn-jnl}
%\documentclass[aip,jap,preprint]{revtex4-1}
\usepackage{graphicx}% Include figure files
\usepackage{dcolumn}
\usepackage{color}
\usepackage{color,soul}
%\draft % marks overfull lines with a black rule on the right

\begin{document}
Dear Editor,

We like to express our appreciation to the reviewers for their comments.
We are resubmitting the revised version of the paper number JMSE--D--22--00483.
We have studied the comments of the reviewer carefully, and have changed the text according to the comments they
have listed.
The location of revisions is  highlighted by yellow in ``Marked--JMSE--D--22--00483.pdf''.
Below we refer to each of the reviewer’s comments.



\subsection*{Response to Reviewer \#1 }

\noindent
\textcolor[rgb]{0.00,0.50,1.00}{\textbf{Comment~1.}}
\emph{Since Ultrasound Stimulated (US) dissociation of FeB pairs in Silicon has been studied before, the author needs to give a short description of the work of reference 16 in the "Introduction" part.}

\noindent
\textcolor[rgb]{0.51,0.00,0.00}{\textbf{Reply:}}
In fact, the possibility of the ultrasound to  change the state of FeB was shown previously  \cite{Ostapenko1994APL,Ostapenko1995}.
In particular, the
FeB pair was revealed \cite{Ostapenko1995} to be dissociated in Cz–Si by the action of ultrasound with acoustic strain $\xi_\mathrm{US}=10^{-5}$--$10^{-4}$.
Furthermore, Ostapenko and Bell \cite{Ostapenko1995} regarded the resonance condition of pair dissociation and used 25--70~kHz.
Besides, it was asserted \cite{Ostapenko1994APL} that in the case of
predominant dissociated pairs, the ultrasound may promote the pairing reaction in contradistinction to the case of a
high fraction of paired iron.
But  empirical evidence  for this prediction  is absent.
In this work,
i)~the wave frequency $f_\mathrm{US}=$(0.3--30)~MHz and subthreshold strain $\xi_\mathrm{US}<2\times10^{-6}$ were used, which were deficient overcome the Coulombic attraction between Fe$_i^+$ and B$_s^-$;
ii)~the predominant  dissociation of FeB was realized by intense illumination.
Thus the association of FeB pair (the migration  of Fe$_i^+$) was firstly investigated in conditions of USL.


The additional information was added to the revised manuscript
(last paragraph in ``Introduction'').



\vspace{1cm}
\noindent
\textcolor[rgb]{0.00,0.50,1.00}{\textbf{Comment~2.}}
\emph{The font in Figure 4 is obviously too small.}

\noindent
\textcolor[rgb]{0.51,0.00,0.00}{\textbf{Reply:}}
The font was enlarged.

\vspace{1cm}
\noindent
\textcolor[rgb]{0.00,0.50,1.00}{\textbf{Comment~3.}}
\emph{The forms of the separation of axis titles and units are inconsistent in the Figures.}

\noindent
\textcolor[rgb]{0.51,0.00,0.00}{\textbf{Reply:}}
All axis titles were changed according to ``Axis title (unit)''.


\subsection*{Response to Reviewer \#2 }
\noindent
\textcolor[rgb]{0.00,0.50,1.00}{\textbf{Comment~1.}}
\emph{Abstract: This part is precise and concise.
It could be better to mention some points of the light induced degradation in solar cells.}

\noindent
\textcolor[rgb]{0.51,0.00,0.00}{\textbf{Reply:}}
The different light-induced degradation (LID) phenomena exist that affect the efficiency
of Cz-silicon solar cells due to a decrease in the lifetime of generated excess charge carriers.
The main reasons for this transformation are boron--oxygen complex formation (BO-LID) \cite{LIDRev} and
iron--boron pair dissociation (FeB-LID).
Besides, the light- and elevated-temperature-induced degradation (LeTID) is observed.
In recent studies, the occurrence of the LeTID defect is related
to the presence of hydrogen and metal impurities \cite{LeTID_H,LeTID_Me,LeTID_Me2}.
We study the  influence of ultrasound on LID recovery.
The complete recovery in the dark at near room temperature and the determined value
of activation energy (0.656 eV) evidenced the iron--boron pair LID in our case.

The text was revised (Abstract and page~4, paragraph~2).

%
%They  occurs due to interaction of boron and oxygen (BO-LID),
%
%metal impurities
%might be involved in the LeTID process
%LeTID defect could be related to the presence of hydrogen
%
%
%LeTID is
%different the well-known light induced degradation (LID) mechanism,
%such as B-O complex formation and Fe-B pair dissociation.
%
%light- and
%elevated-temperature-induced degradation (LeTID)
%One example is light-induced degradation (LID), which
%often occurs due to interaction of boron and oxygen, also referred
%to as BO-LID
%
%
%We study the mitigation of light- and elevated temperature-induced degradation (LeTID) with high-intensity
%illumination treatments, placing special emphasis on inline feasibility.
%
%In this paper, we study a light-induced degradation (LID) mechanism observed in commercial


\vspace{1cm}
\noindent
\textcolor[rgb]{0.00,0.50,1.00}{\textbf{Comment~2.}}
\emph{Introduction: Why not other doping elements ?
 why iron given more importance ? explanation required. Objective is not well framed.}

\noindent
\textcolor[rgb]{0.51,0.00,0.00}{\textbf{Reply:}}
The silicon solar cells (SCs) constitute about 90\% of current global photovoltaic production capacity.
Iron is one of the most relevant, omnipresent, and efficiency--limiting metallic impurities
in p--type silicon solar cells \cite{Istratov1999,IronSC}.
Therefore, the methods of defect engineering aimed at iron have practical importance.
On the one hand, shallow acceptors (B, Al, Ga, In) are effective trapping sites for iron around room temperature
and in darkness in p-Si
due to electrostatic attraction between the negatively charged
acceptors and the positively charged iron ions.
All Fe-acceptor pairs are similar:
complexes have two structural configurations
with trigonal and orthorhombic symmetry and can be broken by
intense illumination and/or annealing above 200$^\circ$C \cite{Istratov1999,FeBKinAPL2013}.
On the other hand, the back surface field (BSF) cell and passivated emitter and rear cell (PERC)
are the most popular designs that have been used in the mass-production of Si SCs,
and both BSF and PERC are mainly  based on boron-doped silicon wafers \cite{SCRev2020,GreenRew2019}.
Therefore the iron--boron pair is one of the most relevant complexes to the defect engineering in real SCs.
The aim of this work is to investigate experimentally how acoustic waves (AWs) influence the ability of iron to diffuse in silicon SCs.

The text was revised
(second and third paragraphs in ``Introduction'').

\vspace{1cm}
\noindent
\textcolor[rgb]{0.00,0.50,1.00}{\textbf{Comment~3.}}
\emph{Experimental part: How much doping of phosphorus? How do they measure thickness?}

\noindent
\textcolor[rgb]{0.51,0.00,0.00}{\textbf{Reply:}}
A diffusion from the gas phase (POCl$_3$) at 940$^\circ$C was performed on wafers resulting in a $n^+$--emitter layer on
the front side (sheet resistance of about $20-30$~$\Omega/\Box$, thickness of $0.7$~$\mu$m).
In addition, to reduce recombination losses and increase the conductivity of the contact layer,
a $p^+$ layer ($10-20$~$\Omega/\Box$, $0.6$~$\mu$m) was formed by boron diffusion from
the gas phase (BCl$_3$) at 985$^\circ$C on the rear surface.
The solid and grid aluminum contacts were formed by magnetron sputtering on the rear and front surfaces, respectively.

The layer thicknesses were evaluated by using an annealing  temperature and duration as well as an partial pressure gas.

More detailed information about samples was added
(first paragraph in ``Experimental and Calculation Details'').


\vspace{1cm}
\noindent
\textcolor[rgb]{0.00,0.50,1.00}{\textbf{Comment~4.}}
\emph{Why were AWs excited in the samples of 2.4; 4.1; 5.4; 9.0; 14; 18; 31 MHz (longitudinal) or 0.3 MHz (transverse) with frequency fUS ?. Give a reason. Why not selected substitution atoms?}

\noindent
\textcolor[rgb]{0.51,0.00,0.00}{\textbf{Reply:}}
We have previously shown \cite{Olikh2018SM,Olikh2018JAP,OlikhJAP,GORB2020,Olikh:Ultras,Olikh:Ultras2016}
that waves with megahertz frequency can be an effective tool for defect engineering in silicon structures.
Thus in this work, we tried to use such waves to control iron impurities in silicon SCs.

On the other hand,
it is widely known that the efficiency of ultrasound influence on  defects in semiconductors depends on acoustic wave frequency.
Moreover, the type of frequency dependence is determined by the mechanism of acousto-defect
interaction \cite{Brailsford,Pavlovich,PeleshchakUJF2016}.
The set of frequencies (2.4, 4.1, 5.4, 9.0, 14, 18, 31 MHz, longitudinal waves)
was used to establish features of
ultrasound influence on iron migration.
Besides,  the effect of the increase in the carrier capture coefficient for defects in silicon SC
is shown \cite{Olikh2018SM} to be intensified in the case of the transverse acoustic waves using.
Accordingly, the transverse waves (0.3~MHz) were used as well.
In the case of acoustically-induced (AI) change of capture coefficient, the
USL causes the variation in the complex components' distance \cite{Olikh2018SM}.
 In our case, the AI effect is the opposite and is weakened for transverse waves using.
Therefore, the acceleration of the FeB pair association does not deal with iron-boron distance change.

The text was revised
(second paragraph in ``Experimental and Calculation Details'';
page~9, paragraph~1).


\vspace{1cm}
\noindent
\textcolor[rgb]{0.00,0.50,1.00}{\textbf{Comment~5.}}
\emph{Result and discussion: In figure 5, x axis when will it reach minimum? Is there any particular reason for selecting 300 to 304K?}

\noindent
\textcolor[rgb]{0.51,0.00,0.00}{\textbf{Reply:}}
The investigations have shown that the dependence of $\tau_\mathrm{US}/\tau_{0}$
on ultrasound intensity is close to linear at low $W_\mathrm{US}$.
The increase in $W_\mathrm{US}$ leads to saturation, which correspond to
$\tau_\mathrm{US}\simeq0.7\tau_{0}$ at $T=340$~K.
Therefore the minimum (saturation) of curves in Fig.~5 is expected at about $0.3$~W/cm$^2$.
It should be noted that according to Ostapenko and Bell \cite{Ostapenko1995},
the further increase in acoustic strain up to $\xi_\mathrm{US}=10^{-5}$ may lead to
FeB pair dissociation and increase in $\tau_\mathrm{US}$.

The temperature range is limited for the following reasons.
On the one hand, the investigations have shown that the ultrasound influence
on FeB pair association time weakened with temperature decrease.
 In fact, the effect value is about 5\% only for $T=300$~K --- see Fig.~6.
Therefore the experiments seem useless at $T<300$.
On the other hand, the portion of interstitial iron atoms that
remain unpaired in the equilibrium state encreases with temperature rise.
According to Wijaranakula \cite{FeB:kinetic}, the concentration of
unpaired iron atom $N_\mathrm{Fe_i,eq}$
depends on temperature, doping level, and concentration of
all iron impurity atoms $N_\mathrm{Fe,0}$.
The estimations show that at 340~K $N_\mathrm{Fe_i,eq}\simeq0.1N_\mathrm{Fe,0}$.
The further temperature decrease leads to an increase in
$N_\mathrm{Fe_i,eq}/N_\mathrm{Fe,0}$ value and
to a significant decrease in change of short circuit current after intense illumination
and measurement accuracy of association time.



\subsection*{Response to Reviewer \#3 }
\noindent
\textcolor[rgb]{0.00,0.50,1.00}{\textbf{Comment~1.}}
\emph{The results are new and may be considered for publication, however, the presentation of the contents is very poor and hence it is unacceptable in its present form.
I guess, the main problem is with the English language used;  many unconventional or unusual technical words, which make it difficult to understand what some descriptions mean. }

\noindent
\textcolor[rgb]{0.51,0.00,0.00}{\textbf{Reply:}}
The text has been revised. We hope for the language improvement.


\vspace{1cm}
\noindent
\textcolor[rgb]{0.00,0.50,1.00}{\textbf{Comment~2.}}
\emph{1.	Page 3, line 28, “Fig. 1 Scheme of the sample…” should be changed to, “Fig. 1 Schematic structure of the sample…”. And also in lines 33 -34, this should be changed as, “The schematic structure of SC …”.}

\noindent
\textcolor[rgb]{0.51,0.00,0.00}{\textbf{Reply:}}
The Reviewer is absolutely right. We have revised the text accordingly.

\vspace{1cm}
\noindent
\textcolor[rgb]{0.00,0.50,1.00}{\textbf{Comment~3.}}
\emph{Fig. 2, presents the measured Isc as a function of time of illumination t.
Thus the y-axis of Fig. 2 should be labelled as Isc ($\mu$A) and the x-axis as t ($10^3$ s).
Also it is not explained what the red and green curves represent?
The should explain Red with US and Green without US?
Thus, the figure captions of Fig. 2 in lines 26-27 on page 4, should be changed to, “Measured Isc plotted as a function of the illumination time t; red curve- with US and green curve without US. ….”}

\noindent
\textcolor[rgb]{0.51,0.00,0.00}{\textbf{Reply:}}
The figure captions was revised.


\vspace{1cm}
\noindent
\textcolor[rgb]{0.00,0.50,1.00}{\textbf{Comment~4.}}
\emph{3.	The second paragraph, on page 4, lines 36-40, does not make any sense to me.
It needs to be rewritten.
It should emphasize how Isc in Eq. (1) is obtained to represent the measure Isc in Fig. 2.}

\noindent
\textcolor[rgb]{0.51,0.00,0.00}{\textbf{Reply:}}
The FeB pair association in the dark was accompanied by the $\tau$ increase and
was monitored by measuring the $I_\mathrm{SC}$ under LED illumination.
The LED illumination induced excess carrier density $<10^{12}$~cm$^{-3})$,
had duty cycle 0.5\% while $I_\mathrm{SC}(t)$ measuring,
and did not cause FeB dissociation.
Moreover, the fitting of the measured dependencies $I_\mathrm{SC}(t)$ after high-intensive illumination
allows determining  the pair concentration and the characteristic time of the FeB complex formation.


The text was revised.
(page~5, paragraph~2).

\vspace{1cm}
\noindent
\textcolor[rgb]{0.00,0.50,1.00}{\textbf{Comment~5.}}
\emph{4.	In Eq. (1) , please define all the symbols used. For example, what is Pph ?
Also if Eq. (1) represents the measured Isc, why does it not depend on the time t?
This needs to be carefully explained.}

\noindent
\textcolor[rgb]{0.51,0.00,0.00}{\textbf{Reply:}}
The Eq.~(1) was modified,
all symbols were defined.
The parameters, which depend on time, were pointed in Eqs.(2),(3),(6) as well.
In initial manuscript the $P_{ph}$ (LED light power) was introduced on page 3, line 57.


\vspace{1cm}
\noindent
\textcolor[rgb]{0.00,0.50,1.00}{\textbf{Comment~6.}}
\emph{Page 5, line 53, “… where $\tau_{ass}$ is the characteristic time of the complex association.”
This does not make sense to me.
Do you mean, ““… where $\tau_{ass}$ is the characteristic time of the formation of Fe-B complex ?” This needs to be fixed.}

\noindent
\textcolor[rgb]{0.51,0.00,0.00}{\textbf{Reply:}}
We have revised the text accordingly.


\vspace{1cm}
\noindent
\textcolor[rgb]{0.00,0.50,1.00}{\textbf{Comment~7.}}
\emph{On page 6, lines 33-37, “ … the values of $\tau_{ass}$ …$1380 \pm20$  for T = 330 K … $1.26 \pm0.02) 10^4$  for T = 300 K …”.
Something is wrong here?
How can a 30 degree difference in temperature can give such a big value of $\tau_{ass}$ from Eq. (10).}

\noindent
\textcolor[rgb]{0.51,0.00,0.00}{\textbf{Reply:}}
According to \cite{FeBKin2019,FeBAssJAP2014,FeBAssSST2011}
\begin{equation}
\label{eqTass}
\tau_\mathrm{ass}=5.7\times10^5\,\frac{\mathrm{s}}{\mathrm{K}\;\mathrm{cm}^3}\times\frac{T}{N_A}\exp\left(\frac{E_m}{kT}\right)\,,
\end{equation}
where
$E_m$ is the energy of Fe$_i^+$ migration, $E_m=0.66$~eV \cite{FeBAssJAP2014,FeBkinAPL2008,FeBKin2019,FeBAssSST2011}.
Hence
\begin{equation}
\frac{\tau_\mathrm{ass}(T=300 \mathrm{K})}{\tau_\mathrm{ass}(T=330 \mathrm{K})}=
\frac{300}{330}\times \exp\left[\frac{0.66}{8.625\times10^{-5}}\left(\frac{1}{300}-\frac{1}{330}\right)\right]\simeq9.2\,.
\end{equation}
At the same time
\begin{equation}
\frac{1.26\times10^4}{1380}\simeq9.1\,.
\end{equation}

$N_A=1.4\times10^{15}$~cm$^{-3}$ for the samples under investigation.
Therefore the experimentally determined times are close to ones expected from Eq.~(\ref{eqTass}).




\vspace{1cm}
\noindent
\textcolor[rgb]{0.00,0.50,1.00}{\textbf{Comment~8.}}
\emph{Page, 6 lines 53-57, the first sentence of Results and Discussion does not make any sense.
Please rewrite it.}

\noindent
\textcolor[rgb]{0.51,0.00,0.00}{\textbf{Reply:}}
The sentence was rewrite in form
``The experiments have shown that the US loading leads to
speed up of recovery of short circuit current after high-intensive illumination.
Therefore, the  FeB association is intensified under AW action.''


\vspace{1cm}
\noindent
\textcolor[rgb]{0.00,0.50,1.00}{\textbf{Comment~9.}}
\emph{Page 6, lines 60-61, “ The figure also shows $\tau_{ass}$ values…” but figure 3 does not show any $\tau_{ass}$?
Please check it and fix it carefully.}

\noindent
\textcolor[rgb]{0.51,0.00,0.00}{\textbf{Reply:}}
Figure~3 presents the measured short circuit current (marks), the fitting curves (lines), and the pair formation time constants  
determined by the fitting (labels).
The text (first paragraph in ``Results and Discussion'') was revised.



\bibliography{olikh}

\end{document}

