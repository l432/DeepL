

%\documentclass[aip,reprint]{revtex4-1}
\documentclass[sn-mathphys]{sn-jnl}
%\documentclass[aip,jap,preprint]{revtex4-1}
\usepackage{graphicx}% Include figure files
\usepackage{dcolumn}
\usepackage{color}
\usepackage{color,soul}
%\draft % marks overfull lines with a black rule on the right

\begin{document}
Dear Editor,

We like to express our appreciation to the reviewers for their comments.
We are resubmitting the revised version of the paper number JMSE--D--22--00483.
We have studied the comments of the reviewer carefully, and have changed the text according to the comments they
have listed.
The location of revisions is  highlighted by yellow in ``Marked--JMSE--D--22--00483.pdf''.
Below we refer to each of the reviewer’s comments.



\subsection*{Response to Reviewer \#1 }

\noindent
\textcolor[rgb]{0.00,0.50,1.00}{\textbf{Comment~1.}}
\emph{Since Ultrasound Stimulated (US) dissociation of FeB pairs in Silicon has been studied before, the author needs to give a short description of the work of reference 16 in the "Introduction" part.}

\noindent
\textcolor[rgb]{0.51,0.00,0.00}{\textbf{Reply:}}
In fact, the possibility of the ultrasound to  change the state of FeB was shown previously  \cite{Ostapenko1994APL,Ostapenko1995}.
In particular, the
FeB pair was revealed \cite{Ostapenko1995} to be dissociated in Cz–Si by the action of ultrasound with acoustic strain $\xi_\mathrm{US}=10^{-5}$--$10^{-4}$.
Furthermore, Ostapenko and Bell \cite{Ostapenko1995} regarded the resonance condition of pair dissociation and used 25--70~kHz.
Besides, it was asserted \cite{Ostapenko1994APL} that in the case of
predominant dissociated pairs, the ultrasound may promote the pairing reaction in contradistinction to the case of a
high fraction of paired iron.
But  empirical evidence  for this prediction  is absent.
In this work,
i)~the $f_\mathrm{US}=$2--30~MHz and subthreshold strain $\xi_\mathrm{US}<2\times10^{-6}$ were used, which were deficient to effectively overcome the Coulombic attraction between Fe$_i^+$ and B$_s^-$;
ii)~the predominant  dissociation of FeB was realized by intense illumination.
Thus the association of FeB pair (the migration  of Fe$_i^+$) was firstly investigated experimentally in conditions of USL.


The additional information was added to the revised manuscript
(last paragraph in ``Introduction'').



\vspace{1cm}
\noindent
\textcolor[rgb]{0.00,0.50,1.00}{\textbf{Comment~2.}}
\emph{The font in Figure 4 is obviously too small.}

\noindent
\textcolor[rgb]{0.51,0.00,0.00}{\textbf{Reply:}}
The font was enlarged.

\vspace{1cm}
\noindent
\textcolor[rgb]{0.00,0.50,1.00}{\textbf{Comment~3.}}
\emph{The forms of the separation of axis titles and units are inconsistent in the Figures.}

\noindent
\textcolor[rgb]{0.51,0.00,0.00}{\textbf{Reply:}}
All axis titles were changed according to ``Axis title (unit)''.


\subsection*{Response to Reviewer \#2 }
\noindent
\textcolor[rgb]{0.00,0.50,1.00}{\textbf{Comment~1.}}
\emph{Abstract: This part is precise and concise.
It could be better to mention some points of the light induced degradation in solar cells.}

\noindent
\textcolor[rgb]{0.51,0.00,0.00}{\textbf{Reply:}}
%***************************
The text was revised.


\vspace{1cm}
\noindent
\textcolor[rgb]{0.00,0.50,1.00}{\textbf{Comment~2.}}
\emph{Introduction: Why not other doping elements ?
 why iron given more importance ? explanation required. Objective is not well framed.}

\noindent
\textcolor[rgb]{0.51,0.00,0.00}{\textbf{Reply:}}
%***************************
The text was revised.


\vspace{1cm}
\noindent
\textcolor[rgb]{0.00,0.50,1.00}{\textbf{Comment~3.}}
\emph{Experimental part: How much doping of phosphorus? How do they measure thickness?}

\noindent
\textcolor[rgb]{0.51,0.00,0.00}{\textbf{Reply:}}
%***************************
The text was revised.


\vspace{1cm}
\noindent
\textcolor[rgb]{0.00,0.50,1.00}{\textbf{Comment~4.}}
\emph{Why were AWs excited in the samples of 2.4; 4.1; 5.4; 9.0; 14; 18; 31 MHz (longitudinal) or 0.3 MHz (transverse) with frequency fUS ?. Give a reason. Why not selected substitution atoms?}

\noindent
\textcolor[rgb]{0.51,0.00,0.00}{\textbf{Reply:}}
%***************************
The text was revised.


\vspace{1cm}
\noindent
\textcolor[rgb]{0.00,0.50,1.00}{\textbf{Comment~5.}}
\emph{Result and discussion: In figure 5, x axis when will it reach minimum? Is there any particular reason for selecting 300 to 304K?}

\noindent
\textcolor[rgb]{0.51,0.00,0.00}{\textbf{Reply:}}
%***************************
The text was revised.


\subsection*{Response to Reviewer \#3 }
\noindent
\textcolor[rgb]{0.00,0.50,1.00}{\textbf{Comment~1.}}
\emph{The results are new and may be considered for publication, however, the presentation of the contents is very poor and hence it is unacceptable in its present form.
I guess, the main problem is with the English language used;  many unconventional or unusual technical words, which make it difficult to understand what some descriptions mean. }

\noindent
\textcolor[rgb]{0.51,0.00,0.00}{\textbf{Reply:}}
%***************************
The text was revised.


\vspace{1cm}
\noindent
\textcolor[rgb]{0.00,0.50,1.00}{\textbf{Comment~2.}}
\emph{1.	Page 3, line 28, “Fig. 1 Scheme of the sample…” should be changed to, “Fig. 1 Schematic structure of the sample…”. And also in lines 33 -34, this should be changed as, “The schematic structure of SC …”.}

\noindent
\textcolor[rgb]{0.51,0.00,0.00}{\textbf{Reply:}}
%***************************
The text was revised.


\vspace{1cm}
\noindent
\textcolor[rgb]{0.00,0.50,1.00}{\textbf{Comment~3.}}
\emph{Fig. 2, presents the measured Isc as a function of time of illumination t.
Thus the y-axis of Fig. 2 should be labelled as Isc ($\mu$A) and the x-axis as t ($10^3$ s).
Also it is not explained what the red and green curves represent?
The should explain Red with US and Green without US?
Thus, the figure captions of Fig. 2 in lines 26-27 on page 4, should be changed to, “Measured Isc plotted as a function of the illumination time t; red curve- with US and green curve without US. ….”}

\noindent
\textcolor[rgb]{0.51,0.00,0.00}{\textbf{Reply:}}
%***************************
The text was revised.


\vspace{1cm}
\noindent
\textcolor[rgb]{0.00,0.50,1.00}{\textbf{Comment~4.}}
\emph{3.	The second paragraph, on page 4, lines 36-40, does not make any sense to me.
It needs to be rewritten.
It should emphasize how Isc in Eq. (1) is obtained to represent the measure Isc in Fig. 2.}

\noindent
\textcolor[rgb]{0.51,0.00,0.00}{\textbf{Reply:}}
%***************************
The text was revised.


\vspace{1cm}
\noindent
\textcolor[rgb]{0.00,0.50,1.00}{\textbf{Comment~5.}}
\emph{4.	In Eq. (1) , please define all the symbols used. For example, what is Pph ?
Also if Eq. (1) represents the measured Isc, why does it not depend on the time t?
This needs to be carefully explained.}

\noindent
\textcolor[rgb]{0.51,0.00,0.00}{\textbf{Reply:}}
%***************************
The text was revised.


\vspace{1cm}
\noindent
\textcolor[rgb]{0.00,0.50,1.00}{\textbf{Comment~6.}}
\emph{Page 5, line 53, “… where $\tau_{ass}$ is the characteristic time of the complex association.”
This does not make sense to me.
Do you mean, ““… where $\tau_{ass}$ is the characteristic time of the formation of Fe-B complex ?” This needs to be fixed.}

\noindent
\textcolor[rgb]{0.51,0.00,0.00}{\textbf{Reply:}}
%***************************
The text was revised.


\vspace{1cm}
\noindent
\textcolor[rgb]{0.00,0.50,1.00}{\textbf{Comment~7.}}
\emph{On page 6, lines 33-37, “ … the values of $\tau_{ass}$ …$1380 \pm20$  for T = 330 K … $1.26 \pm0.02) 10^4$  for T = 300 K …”.
Something is wrong here?
How can a 30 degree difference in temperature can give such a big value of $\tau_{ass}$ from Eq. (10).}

\noindent
\textcolor[rgb]{0.51,0.00,0.00}{\textbf{Reply:}}
%***************************
The text was revised.


\vspace{1cm}
\noindent
\textcolor[rgb]{0.00,0.50,1.00}{\textbf{Comment~8.}}
\emph{Page, 6 lines 53-57, the first sentence of Results and Discussion does not make any sense.
Please rewrite it.}

\noindent
\textcolor[rgb]{0.51,0.00,0.00}{\textbf{Reply:}}
%***************************
The text was revised.



\vspace{1cm}
\noindent
\textcolor[rgb]{0.00,0.50,1.00}{\textbf{Comment~9.}}
\emph{Page 6, lines 60-61, “ The figure also shows $\tau_{ass}$ values…” but figure 3 does not show any $\tau_{ass}$?
Please check it and fix it carefully.}

\noindent
\textcolor[rgb]{0.51,0.00,0.00}{\textbf{Reply:}}
%***************************
The text was revised.



%-----------------------------------------------------------------
\noindent
\textcolor[rgb]{0.00,0.50,1.00}{\textbf{Comment~1.}}
\emph{The English expression is a little awkward in places, but not to the extent that it negatively affects the readability, in my opinion.}

\noindent
\textcolor[rgb]{0.51,0.00,0.00}{\textbf{Reply:}}
The text was \hl{revised}.


\subsection*{Response to Reviewer \#2 }

\noindent
\textcolor[rgb]{0.00,0.50,1.00}{\textbf{Comment~1.}}
\emph{In Fig. 4, the numbers identifying the different curves are missing.}

\noindent
\textcolor[rgb]{0.51,0.00,0.00}{\textbf{Reply:}}
The curves' numbers were added.
We apologize for the inattention.


\vspace{1cm}
\noindent
\textcolor[rgb]{0.00,0.50,1.00}{\textbf{Comment~2.}}
\emph{Figure 6 is not so relevant, and could be given as supplementary information}

\noindent
\textcolor[rgb]{0.51,0.00,0.00}{\textbf{Reply:}}
Figure~6 was transferred to the Supplementary material,
the text was revised (page~5, column~1, paragraph~2).


\subsection*{Response to Reviewer \#3 }

\noindent
\textcolor[rgb]{0.00,0.50,1.00}{\textbf{Comment~1.}}
\emph{Page 1, column 2, line 1: The lattice deformation amplitude should have a unit.}

\noindent
\textcolor[rgb]{0.51,0.00,0.00}{\textbf{Reply:}}
The ``relative deformation'' was meant.
The term ``strain'' is used in the revised version.


\vspace{1cm}
\noindent
\textcolor[rgb]{0.00,0.50,1.00}{\textbf{Comment~2.}}
\emph{Page 1, column 2, paragraph 3: The origin of the iron in the diodes should be explained. How are the diodes contaminated with iron? }

\noindent
\textcolor[rgb]{0.51,0.00,0.00}{\textbf{Reply:}}
The technological process of solar cells (SCs) manufacturing from Cz-p-Si wafers included the formation of separating and isotype barriers ($n^+$-$p$ and $p$-$p^+$ junctions) by diffusion of phosphorus (POCl$_3$) and boron (BCl$_3$) from the gas phase, respectively; thermal oxidation; thermal annealing; photolithography; etching the dividing groove; chemical treatments; magnetron sputtering of aluminum contacts to the front and back sides.

It has been found that some SC lots have significantly worse parameters compared to typical solar cells for this technological process.
In particular, the photoconversion efficiency was almost halved.
The analysis showed that the reason for such a deterioration in the SC parameters is a sharp drop in the diffusion length of minority charge carriers (electrons) $L_n$ in the SC base.
Additional experiments with thermal annealing at temperatures of 200$^\circ$C and 90$^\circ$C
(the procedure is described by Tayyib~\emph{et al.}\cite{TAYYIB201221})
showed that the decrease in $L_n$ value is caused by iron impurities available in the SC base at concentrations up to $4\cdot10^{13}$~cm$^{-3}$.
It has also been found that the source of iron impurity is insufficiently pure chemicals that were used for chemical treatments in the technological process, obtained from another supplier.
These reagents were the source of contamination in the process of manufacturing experimental SC samples.

To study the effect of ultrasonic loading on the kinetics of the transformation  of iron-boron pairs, samples with varying degrees of iron contamination (iron concentration $2\cdot10^{12}$-$4\cdot10^{13}$~cm$^{-3}$) were taken.

More detailed information about iron sources was added (page~1, column~2, paragraphs~3 and 4).


\vspace{1cm}
\noindent
\textcolor[rgb]{0.00,0.50,1.00}{\textbf{Comment~3.}}
\emph{Page 1, column 2, paragraph 5: How was the excess carrier density, which is induced by the LED illumination, estimated? This point should be explained. }

\noindent
\textcolor[rgb]{0.51,0.00,0.00}{\textbf{Reply:}}
The excess carrier density was estimated by using open-circuit voltage value $V_\mathrm{OC}$.
According to Sachenko \emph{et. al.}\cite{JAPSach}
\begin{equation}
  \Delta n=-\frac{n_0}{2}+\sqrt{\frac{n_0^2}{4}+n_i\exp\left(\frac{qV_\mathrm{OC}}{kT}\right)}
\end{equation}
where
$n_0$ is the equilibrium electron concentration,
$n_i$ is the intrinsic electron concentration.

The short information was added to the revised manuscript (page~2, column~1, paragraph~3).


\vspace{1cm}
\noindent
\textcolor[rgb]{0.00,0.50,1.00}{\textbf{Comment~4.}}
\emph{Figure 2: Short circuit current has the unit $\mu$A not $\mu$m.}

\noindent
\textcolor[rgb]{0.51,0.00,0.00}{\textbf{Reply:}}
The reviewer is quite right.
The graph was corrected.


\vspace{1cm}
\noindent
\textcolor[rgb]{0.00,0.50,1.00}{\textbf{Comment~5.}}
\emph{Page 2, column 2, paragraph 1: The materials doping level is given on page 1 with 10 Ohm cm. This is about 1.4e15cm$^{-3}$ and not 1.4e16cm$^{-3}$ as stated here.}

\noindent
\textcolor[rgb]{0.51,0.00,0.00}{\textbf{Reply:}}
The reviewer is quite right.
It was a slip.
The correction is done.


\vspace{1cm}
\noindent
\textcolor[rgb]{0.00,0.50,1.00}{\textbf{Comment~6.}}
\emph{Equation 7: The unit of the pre-factor is missing. }

\noindent
\textcolor[rgb]{0.51,0.00,0.00}{\textbf{Reply:}}
The Equation was corrected.

\vspace{1cm}
\noindent
\textcolor[rgb]{0.00,0.50,1.00}{\textbf{Comment~7.}}
\emph{Page 3, column 1, paragraph 3: The obtained iron concentration is compared to results obtained from diffusion length measurements. There should be a reference to these measurements or more details should be given.}

\noindent
\textcolor[rgb]{0.51,0.00,0.00}{\textbf{Reply:}}
The diffusion length before ($L_{n0}$) and after ($L_{n1}$) pair dissociation was measured using spectral dependencies of short circuit current
Then the iron concentration was determined by using Zoth and Bergholz\cite{FeB_Zong} equation:
\begin{equation}
  N_\mathrm{Fe}(\mathrm{cm}^{-3})=1.06\cdot10^{16}\left(\frac{1}{L_{n1}^2}-\frac{1}{L_{n0}^2}\right)(\mu\mathrm{m}^{-2})
\end{equation}

The short information was added to the revised manuscript.

\vspace{1cm}
\noindent
\textcolor[rgb]{0.00,0.50,1.00}{\textbf{Comment~8.}}
\emph{Figure 4: The numbers of the curves given in the caption are not included in the graphs. This must be improved otherwise the figures cannot be understood. Each graph should also be marked by a letter. What is ``G'' at the x-axis of the inset? This should be explained. }


\noindent
\textcolor[rgb]{0.51,0.00,0.00}{\textbf{Reply:}}
The graphs were revised.
The curves' numbers were added.
Each graph was marked by a letter.
``G'' was changed by ``Will'' (radiation intensity of halogen lamp).
We apologize for the inattention.


\vspace{1cm}
\noindent
\textcolor[rgb]{0.00,0.50,1.00}{\textbf{Comment~9.}}
\emph{Figure 5: This plot is confusing and must be revised. What is the main statement of this figure? What are the differences between the samples and why do the results change from sample to sample? The axes should have the same scaling. What are the light blue bars in (a)? }

\noindent
\textcolor[rgb]{0.51,0.00,0.00}{\textbf{Reply:}}
Figure~5 was changed by Table~I.
The main assignment of this data is an illustration of

\noindent
i)~USL actually does not influence the $\tau_\mathrm{dis}$  magnitude;

\noindent
ii)~some pairs do not dissociate under illumination in the USL case when light--induced pair dissociation is close to saturation.
The decrease in illumination intensity reduces the last effect at the given illumination times.

The main differences between the samples are the iron concentrations.
However, data show that
the effect of ultrasound does not qualitatively change with the iron concentration alteration
(from sample to sample).
The value of acousto-induced change in $N_\mathrm{Fe,fit}$ value depends on ultrasound intensity
(see data for sample SC350-1, $W_\mathrm{ill}=0.16$~W/cm$^2$) and frequency.


The dissociation rate of FeB pairs depends on iron concentration, light intensity, and
temperature\cite{Schmidt2019,FeBLight2,FeBKin2019,Lagowskii1993}.
There is a reason of $\tau_\mathrm{dis}$ value change from panel to panel in Fig.~5 in the unrevised manuscript (from row to row in Table~I in the revised manuscript)

The light blue bars in Fig.~5(a) (the unrevised manuscript) corresponded to another ultrasound intensity
(0.6~W/cm$^2$).

\vspace{1cm}
\noindent
\textcolor[rgb]{0.00,0.50,1.00}{\textbf{Comment~10.}}
\emph{Main Problem}
\emph{The impact of ultrasound on the iron-boron-pair reaction was first reported by Ostapenko and Bell in 1995 [1]. They found that the iron boron pairs dissociate due to an ultrasound treatment. This is in contradiction to the findings, which are reported herein. In the contribution under review it is found that the association of the FeB pairs is enhanced by the ultrasound treatment. This contradiction must absolutely be discussed by the authors otherwise the manuscript cannot be published.}

\emph{
[1] S. S. Ostapenko and R. E. Bell, "Ultrasound stimulated dissociation of Fe-B pairs in silicon," J. Appl. Phys., vol. 77, no. 10, p. 5458, 1995.
 }

\noindent
\textcolor[rgb]{0.51,0.00,0.00}{\textbf{Reply:}}
We do not keep silent about reports Ostapenko \emph{et al.} --- see Refs.26,27;
page~3, column~2, paragraph~1; page~7, column~2, paragraph~1 in the unrevised manuscript.
But Reviewer is right and more attention should be paid to the comparison of the findings.

In our opinion,
the reported herein and previous results have a lot in common.
According to Ostapenko \emph{et al.}\cite{Ostapenko1995SST}
Fe$_i$ has to ``jump'' to the next nearest interstitial under ultrasound action;
we state about the decrease in Fe$_i$ migration energy value
as well as enhance of Fe$_i$ diffusivity
in case of ultrasound loading.
The effectiveness of ultrasound influence increase with the rise of temperature in both cases:
according to Ostapenko \emph{et al.},\cite{Ostapenko1994APL,Ostapenko1995SST}
the diffusion length increases from 10.5~$\mu$m to 13~$\mu$m at 20$^\circ$C and up to
22~$\mu$m at 70$^\circ$C;
$\Delta E_\mathrm{US}$ increases from 3~meV to 8~meV with temperature change
from 300~K to 340~K at  $f_\mathrm{US}=0.3$~MHz and $W_\mathrm{US}=0.76$~W/cm$^2$
(Fig.6 in the revised manuscript).
The difference in the action of ultrasound
(the iron-boron pairs dissociation or the enhancing of pairing)
is associated with the difference in the intensity of the acoustic influence.
For instance, the researchers used acoustic strain $\xi_\mathrm{US}=10^{-5}$-$10^{-4}$
was used\cite{Ostapenko1995}
to dissociate FeB pairs in Cz--Si.
Furthermore,
Ostapenko and Bell\cite{Ostapenko1995} regarded the resonance condition of
pair reorientation (first step of dissociation) and used $25-70$~kHz.
In our case,  $\xi_\mathrm{US}<2\cdot10^{-6}$ and $f_\mathrm{US}=2-30$~MHz are deficient to effectively overcome the Coulombic attraction between Fe$_i^+$ and B$_s^-$.
Additionally, the presented data show that the effectiveness of acoustically--induced change
decreases as the ultrasound frequency increases.

It should be noted that Ostapenko \emph{et al.}\cite{Ostapenko1994APL} asserted
``in the  case  of predominant  dissociated  pairs,  it [ultrasound treatment] may promote  the
pairing reaction'' (page~1557, column~1, paragraph~2).
In our case, the predominant  dissociation was realized by intense illumination, and then
the ultrasound loading accelerated pairing.
Thus, the reported results provided empirical evidence  for the above-mentioned prediction.


On the other hand, a certain manifestation of partial acoustically-induced FeB dissociation could be
a decrease in the concentration of pairs, which dissociate under illumination in the USL case.
In fact, if some pairs were dissociated by  ultrasound waves, which had been pre-exited
in the sample with a high initial  fraction of paired Fe,
they can not be dissociated under illumination.
The decrease in the temperature   or photon quantity leads to reduce in the set of FeB, which was modified by ultrasound, or in the set of FeB,
which was modified by the capture of light-induced electrons, respectively.
Sets cease to overlap and $N_\mathrm{Fe,fit}(W_\mathrm{US}>0)\simeq N_\mathrm{Fe,fit}(W_\mathrm{US}=0)$ --- see Table~I in the revised manuscript.


Finally, let's summarize the differences between the manuscript and previously reported results.
Ostapenko \emph{et al.}\cite{Ostapenko1995,Ostapenko1994APL,Ostapenko1995SST}
investigated the change in silicon properties (diffusion length) \textbf{after}
the action of ultrasound with high (above threshold) intensity.
This is the so-called ultrasound treatment mode of operation
and results  can be used to develop a new approach to semiconductor properties modification.
Our manuscript is devoted to the modification of the process
(short circuit current kinetics or FeB transformation),
\textbf{during} subthreshold acoustic wave propagation.
This is ultrasound loading mode,
which can be used for acoustically controlled tuning of the technological processes.

The additional information was added to the revised manuscript
(page~7, column~1, last paragraph; page~7, column~2, paragraph~1).

\bibliography{olikh}

\end{document}

