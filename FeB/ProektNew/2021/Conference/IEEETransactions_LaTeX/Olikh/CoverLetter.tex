


\documentclass[preprint]{elsarticle}

\begin{document}


To:
IEEE Journal of Photovoltaics Editorial Board


Subject:
Article Submit

\vspace{5mm}
Dear Editors,

\vspace{3mm}
%Enclosed is a manuscript entitled “XXX XXXX XXXXX” by sb, which we are submitting for publication in the journal of …. We have chosen this journal because it deals with …
Enclosed with this letter you will find en electronic submission of manuscript entitled ``An Evaluation for Iron Contamination in Silicon Solar Cell Using Ideality Factor and Machine Learning'' by O.~Olikh, O.~Lozitsky, and O.Zavhorodnii.
This is an origin paper which has not
%neither previously nor
simultaneously in whole or in part been submitted anywhere else.
%Both authors have read and approved the final version submitted.
%No part of this paper has published or submitted elsewhere.
No conflict of interest exits in the submission of this manuscript.

%It is known that the electrical properties of semiconductor devices are determined by the crystal microstructure.
%The present manuscript focused on silicon MOS structure, one of the most common forms of electronic devices used in application.
%It has been experimentally observed that ultrasound treatment leads to radiation defect annealing and recovering of $\gamma$--degraded silicon MOS structure characteristics.
%We believe that using ultrasound for defect engineering would be of interest to the journal’s readers.

It is well known that impurities are crucial for the solar cells performance.
There are many experimental methods for impurity evaluation, such as the infrared spectroscopy, deep level transient spectroscopy, photoluminescence, thermally stimulated capacitance and current, secondary ion mass spectrometry etc. 
These methods are complicated enough and demand a special setup. 
At the same time, there is a simpler and commonly used technique, which
is the analysis of the solar cell current--voltage characteristics. 
The present manuscript describes the method of contaminant concentration evaluation by
using the ideality factor value, which extracted from current-voltage curve. 
The method is based on results of numerical simulation of solar cells and use a deep neural network. 
We believe that such way of defect characterisation would be of interest to the readers

%Mechano--sorptive is sometimes denoted as accelerated creep.
%It has been experimentally observed that the creep of parer accelerates if it is subjected to a cyclic moisture content.
%This is of large practical importance for the paper industry.
%The present manuscript describes a micromechanical model on the fibre network level that is able to capture the experimentally observed behaviour.
%In particular, the difference between mechano--sorptive creep in tension is analysed.


%The paper demonstrates [significant finding and its significance]. As such this paper should be of interest to a broad readership including those interested in [what kinds of research, topics, techniques – should be those targeted by the journal]
%We believe that sth would be of interest to the journal’s readers.

%Three potential independent reviewers who have excellent  knowledge about  this paper are
%1.Dr. Chen, University of Maine, email1@university.com

We would  very much appreciate if you would consider the manuscript for publication in the \emph{IEEE Journal of Photovoltaics}.
%We appreciate your consideration of our manuscript, and we look forward to receiving comments from the reviewers.

\vspace{3mm}

Sincerely yours,

Oleg~Olikh and co-authors


Taras Shevchenko National University of Kyiv


Kyiv 01601, Ukraine

E-mail: olegolikh@knu.ua


%Dear Editors
%It is more than 12 weeks since I submitted our manuscript (No: ) for possible publication in your journal. I have not yet received a reply and am wondering whether you have reached a decision. I should appreciated your letting me know what you have decided as soon as possible.








\end{document}

