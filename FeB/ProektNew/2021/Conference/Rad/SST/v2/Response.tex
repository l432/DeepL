

%\documentclass[aip,reprint]{revtex4-1}
%\documentclass[sn-mathphys]{sn-jnl}
\documentclass[10pt]{iopart}
%\documentclass[aip,jap,preprint]{revtex4-1}
\usepackage{graphicx}% Include figure files
\usepackage{dcolumn}
\usepackage{color}
\usepackage{color,soul}
%\draft % marks overfull lines with a black rule on the right

\begin{document}
Dear Editor,

We like to express our appreciation to the reviewers for their comments.
We are resubmitting the revised version of the paper number SST--108576.
We have studied the comments of the reviewer carefully, and have changed the text according to the comments they
have listed.
The location of revisions is  highlighted by yellow in ``Marked--SST--108576.pdf''.
Below we refer to each of the reviewer’s comments.



\subsection*{Response to Reviewer \#1 }

\noindent
\textcolor[rgb]{0.00,0.50,1.00}{\textbf{Comment~1.}}
\emph{Details that how the microwave treatments affect the semiconductors are not clear.
The author claimed that it was not only heating up, and several possibilities were suggested.
But still the physics of the microwave treatment is very unclear for the reviewer.}

\noindent
\textcolor[rgb]{0.51,0.00,0.00}{\textbf{Reply:}}

Let's consider the possible ways microwave treatments affect defects in structures under investigation. 
The primary process, which determines defect production under irradiation, 
is a displacement of the atoms from the sites of the lattice. 
But the energy of a photon with frequency 2.45~GHz is about $10^{-5}$~eV only;
the threshold displacement energies in GaAs and SiC are $8-28$~eV \cite{Ed:GaAs} and $20-35$~eV \cite{Ed:SiC}, respectively. 
Therefore such a channel of microwave-induced modification of defect subsystem is unreal.

Microwave energy is known to transform into heat inside the material.
Processes based on microwave heating find many industrial applications.
But the used experimental procedure 
(pulsed microwave radiation with a period of 500~s and a duty cycle of 1\%)
allowed prevention of essential heating. 
The calculation according to Bacherikov~\emph{et al}. \cite{Bacherikov2008En} shows that 
the maximum possible heating temperature of the sample $\Delta T$ is about 1~K.
The $\Delta T$ magnitude was confirmed by measurements using a T--type thermocouple.
As a result, the influence of microwave heating can be neglected.




The additional information was added to the revised manuscript
(last paragraph in ``Introduction'').



\vspace{1cm}
\noindent
\textcolor[rgb]{0.00,0.50,1.00}{\textbf{Comment~2.}}
\emph{The author's suggested modifications of the defects are too drastic.
Such the defect modifications change the deep level parameters more.
For example, deep level parameters for V\_Si V\_C and V\_C in silicon carbide have been
identified by both the experiments and theoretical calculations,
and their parameters are very different.
The small changes in deep level parameters should be other reasons.
They should be experimental errors or very small modifications of the defect structures.}

\noindent
\textcolor[rgb]{0.51,0.00,0.00}{\textbf{Reply:}}
The font was enlarged.

\vspace{1cm}
\noindent
\textcolor[rgb]{0.00,0.50,1.00}{\textbf{Comment~3.}}
\emph{The adopted deep level observation techniuqe is not so common,
and it is difficult to judge measurement accuracy.
For confirmation of the results, conventional techniques such as DLTS should be employed simultaneously.}

\noindent
\textcolor[rgb]{0.51,0.00,0.00}{\textbf{Reply:}}
The font was enlarged.


\vspace{1cm}
\noindent
\textcolor[rgb]{0.00,0.50,1.00}{\textbf{Comment~4.}}
\emph{The materials observed are not the state of the art materials in industries.
Therefore, even if the results and physics are truth,
the impacts of this manuscript are limited.
If the microwave treatment has surely advantages compared
with conventional processes, the author can apply it to industry important materials, such as 4H-SiC or GaN.}


\noindent
\textcolor[rgb]{0.51,0.00,0.00}{\textbf{Reply:}}
The font was enlarged.


\subsection*{Response to Reviewer \#2 }
\noindent
\textcolor[rgb]{0.00,0.50,1.00}{\textbf{Comment~1.}}
\emph{I suggest the authors concentrating on a specific semiconductor,
digging deeply on the evolution of defects after MWT,
and discussing the effect of doping on the process.
The current work relates to the type of the host,
the single crystal/epitaxial layers, doping concentrations.
Readers lost easily during the introduction of results and discussions.}

\noindent
\textcolor[rgb]{0.51,0.00,0.00}{\textbf{Reply:}}
The text was revised.


\vspace{1cm}
\noindent
\textcolor[rgb]{0.00,0.50,1.00}{\textbf{Comment~2.}}
\emph{The authors carried out MWT experiments on differently doped SiC or GaAs.
Did the concentration of dopants change the evolution of point defects?
The Fermi level is different for differently doped hosts,
which charge states of intrinsic defects and MWT-induced defects, and thus the interaction, may be different.}

\noindent
\textcolor[rgb]{0.51,0.00,0.00}{\textbf{Reply:}}
The text was revised.


\vspace{1cm}
\noindent
\textcolor[rgb]{0.00,0.50,1.00}{\textbf{Comment~3.}}
\emph{The Pool-Frenkel effect is related to defect states of dislocations.
Yes, some works dealt the interaction between irradiation-induced point defects and dislocations,
such as Appl. Phys. Lett. 117, 023501 (2020) and J. Mater. Chem. C 9, 3177 (2021).
But I didn’t see any detailed discussion in this work.}

\noindent
\textcolor[rgb]{0.51,0.00,0.00}{\textbf{Reply:}}
The text was revised.


\vspace{1cm}
\noindent
\textcolor[rgb]{0.00,0.50,1.00}{\textbf{Comment~4.}}
\emph{Typos throughout the manuscript should be corrected, including but not limited to:}

\emph{(1) ``and semiconducting compounds including [6, 8]''}

\emph{(2) ``doping degree''}

\emph{(3) ``$0:31\div0:33$''}

\noindent
\textcolor[rgb]{0.51,0.00,0.00}{\textbf{Reply:}}
The text was revised.


\subsection*{Response to Reviewer \#3 }
\noindent
\textcolor[rgb]{0.00,0.50,1.00}{\textbf{Comment~1.}}
\emph{The reviewed manuscript presents results on investigation of deep levels
in various materials before and after microwave treatment with different durations (doses).
Therefore, the investigation is not well focused, since the nature of defects is different
for different materials, and they not necessarily should be governed by the same trends. }

\noindent
\textcolor[rgb]{0.51,0.00,0.00}{\textbf{Reply:}}
The text was revised.


\vspace{1cm}
\noindent
\textcolor[rgb]{0.00,0.50,1.00}{\textbf{Comment~2.}}
\emph{The results could present interest, even though the investigated materials are not the state of the art ones in industries.
However, the used experimental method is not widely approved,
and the identification of the microscopic nature of defects is not an easy
task even with widely used investigation methods such as DLTS or DLOS.
As a result, the discussions about the microscopic nature of the observed defects
and their reconfiguration during the microwave treatment
as well as the made conclusions are not enough convincing
(especially in cases when the deep level activation
energies do not change very much under microwave treatment,
as well as in cases when there is a large difference
between the experimental results and those coming from theoretical consideration).}

\noindent
\textcolor[rgb]{0.51,0.00,0.00}{\textbf{Reply:}}
The text was revised.

\vspace{1cm}
\noindent
\textcolor[rgb]{0.00,0.50,1.00}{\textbf{Comment~3.}}
\emph{I consider that the manuscript is not suitable for publication in the present form.
However, the reliability of the obtained results may be significantly improved,
if the study is complimented with other methods of investigation such as DLTS or DLOS,
at least for some of the investigated materials.
Moreover, the DLTS method also gives information about the concentration of deep levels.
In such a case, a comparison of parameters of deep levels obtained by TAV and DLTS,
combined with an analysis of the results of previously performed investigations,
may provide enough arguments confirming the
proposed microscopic nature of defects and their transformations.}

\noindent
\textcolor[rgb]{0.51,0.00,0.00}{\textbf{Reply:}}
The text was revised.

\cite{OstrovskiiSST}

\section*{References}

\bibliographystyle{iopart-num}
\bibliography{olikh}

\end{document}

