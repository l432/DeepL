

%\documentclass[aip,reprint]{revtex4-1}
%\documentclass[sn-mathphys]{sn-jnl}
\documentclass[10pt]{iopart}
%\documentclass[aip,jap,preprint]{revtex4-1}
\usepackage{graphicx}% Include figure files
\usepackage{dcolumn}
\usepackage{color}
\usepackage{color,soul}
%\draft % marks overfull lines with a black rule on the right

\begin{document}
Dear Editor,

We like to express our appreciation to the reviewers for their comments.
We are resubmitting the revised version of the paper number SST--108576.
We have studied the comments of the reviewer carefully, and have changed the text according to the comments they
have listed.
The location of revisions is  highlighted by yellow in ``Marked--SST--108576.pdf''.
Below we refer to each of the reviewer’s comments.



\subsection*{Response to Reviewer \#1 }

\noindent
\textcolor[rgb]{0.00,0.50,1.00}{\textbf{Comment~1.}}
\emph{Details that how the microwave treatments affect the semiconductors are not clear.
The author claimed that it was not only heating up, and several possibilities were suggested.
But still the physics of the microwave treatment is very unclear for the reviewer.}

\noindent
\textcolor[rgb]{0.51,0.00,0.00}{\textbf{Reply:}}

Let's consider the possible ways microwave treatments affect defects in structures under investigation.
The primary process, which determines defect production under irradiation,
is a displacement of the atoms from the sites of the lattice.
But the energy of a photon with frequency 2.45~GHz is about $10^{-5}$~eV only;
the threshold displacement energies in GaAs and SiC are $8-28$~eV \cite{Ed:GaAs} and $20-35$~eV \cite{Ed:SiC}, respectively.
Therefore such a channel of microwave-induced modification of defect subsystem is unreal.

Microwave energy is known to transform into heat inside the material.
Processes based on microwave heating find many industrial applications.
But the used experimental procedure
(pulsed microwave radiation with a period of 500~s and a duty cycle of 1\%)
allowed prevention of essential heating.
The calculation according to Bacherikov~\emph{et al}. \cite{Bacherikov2008En} shows that
the maximum possible heating temperature of the sample $\Delta T$ is about 1~K.
The $\Delta T$ magnitude was confirmed by measurements using a T--type thermocouple.
As a result, the influence of microwave heating can be neglected.

There are many experimental observations that suggest
non-thermal influence of microwave fields \cite{MW:Si2018,MWT:Rew2001}.
These phenomena can be related to various physical reasons.
First, it is known \cite{MW:Force,Milenin:SPQEO2020} that a free charged particle in an electromagnetic field
performs a drift in parallel to the electric component.
The drift velocity magnitude is given by $\upsilon_{\bot}\propto (E_0/m \nu)$
(where
$E_0$ is the amplitude of the electric field,
$m$ is the mass of the particle)
and velocity direction depends on the phase
of the field at the initial time.
Consequently, along with the systematic drift of individual charged
particles, directional movement of the entire
set of particles is absent.
Besides, the particle drifts in the direction of
wave propagation with the velocity $\upsilon_\|\propto(E_0/m \nu)^2$.
However, the charged point defects in
semiconductor crystal are not free and
have to overcome potential barriers when moving.
It can be taken into account by using effective
mass $m_{\rm eff}$, which exponentially depends on barrier height \cite{Milenin:SPQEO2020}.
In our opinion, the mentioned features testify that
such MW-induced movement is not responsible for revealed effects.

Second, the ponderomotive forces can arise under MW action.
Under inhomogeneous microwave electromagnetic field conditions, the
induced oscillatory defect fluxes are rectified,
leading to directional, macroscopic mass transport \cite{MWT:Rew2001,MWT:PandForce97,MWT:PandForce92}.
The ponderomotive force can be described as follows \cite{MWT:PandForce97,Milenin:SPQEO2020}:
\begin{equation}\label{eqFpan}
  F_p(x)=\frac{q^2\beta E_0^2}{8 m_{\rm eff} \pi^2 \nu^2} \exp(-2\beta x)\,,
\end{equation}
where
$q$ is the charge of the defect,
and $\beta$ is the coefficient of electromagnetic wave absorption,
the axis $x$ is along the direction of wave propagation.
On the one hand, the both the MW attenuation and ponderomotive force are essential
when $d\geq\beta^{-1}$
(where $d$ is the thickness of the semiconductor crystal).
On the other hand, the following expression can be used to estimate $\beta$ \cite{Milenin:SPQEO2020}:
\begin{equation}\label{eqBeta}
  \beta=\frac{1}{c}\left(\frac{\sigma\pi\nu}{\varepsilon_0}\right)^{\frac{1}{2}}\,,
\end{equation}
where
$\sigma=e n \mu_n$ is crystal conductivity;
$\mu_n$  is the electron mobility,
8500~cm$^2$/sV for GaAs, 400~cm$^2$/sV for SiC.
Accordind to \cite{Milenin:SPQEO2020},
the Eq.~(\ref{eqBeta}) is correct in the case of
$(\sigma/2\pi\varepsilon_0\varepsilon\nu)\gg1$
(where
$\varepsilon$ is the dielectric perminity;
12.9 for GaAs, 10.03 for SiC),
which corresponds to the samples under investigation.
The calculations show that
$\beta^{-1}$ is $(57-90)$~$\mu$m for SiC crystals,
138~$\mu$m for GAS2,
20~$\mu$m for GAS1 and substrate of epitaxial structures,
$(100-470)$~$\mu$m for epi-layers.
Thus the ponderomotive forces are able to cause the movement of the
charged point defects both in the single-crystal samples and
substrate of epitaxial structures, and the effect is maximal in the near--surface region.
Though ponderomotive influence can  not be the only reason for observed effects.
In fact $\beta(\mathrm{GAS1})>\beta(\mathrm{GAS2})$,
however, MWT with $t_\mathtt{MWT}=20$~s does not lead to defect transformation
in GAS1, unlike in GAS1 --- see table~1.
Similar results are observed for SIC1(2) and SIC3.

Third, it was shown \cite{MWT:JLumin,Konakova2007JTFEn,Milenin:SPQEO2019} that under resonance conditions
(the coincidence of eigenfrequencies of the dislocation segment vibrations and electrical component of the microwave radiation),
multiple dislocation loops occur.
Besides, the MWT causes the  movement of dislocations.
In particular, at $\nu=2.45$~GHz for GaAs, resonant detachment
of numerous dislocations with the length $L\leq4$~$\mu$m becomes possible \cite{Milenin:SPQEO2019}.
The dislocation climb is accompanied by intrinsic defect generation.
In addition, the behavior of the dislocation segment in a MW electric field may be strongly affected by
impurity atoms, which decorate dislocations.
Having accumulated at dislocations, they, on the one hand,
decrease resonance frequency;
on the other hand, impurity atoms may detach from dislocations at
high oscillation amplitudes, and free impurity atoms may appear in the crystal \cite{MWT:JLumin,Konakova2007JTFEn}.
In turn, the appearance of free doping atoms can result in an intrinsic defect concentration increase.
In our case, the dislocation generation is confirmed by a change in both curvature radius and
deformation of the near-surface crystallographic planes after MWT.

Fourth, the MW-induced destruction of impurity complexes, united in clusters,
is described \cite{MWT:JLumin,Konakova2007JTFEn,Milenin:SPQEO2019}.
This is resonant phenomenon as well, and it is expected if the
irradiation frequency is close to ion--plasma frequency
$\nu_r=\sqrt{e^2N_\mathtt{com}/4\pi^2\varepsilon_0\varepsilon\mu}$
(where
$N_\mathtt{com}$ is the complex concentration in the cluster,
$\mu$ is the reduced mass of complex ions).
For example, $\nu_r$ equals to 2.01~GHz for
Te$^+$--Cu$^-$ complex with $N_\mathtt{com}=5\cdot10^{16}$~cm$^{-3}$
in GaAs \cite{MWT:JLumin}.
But since the revealed transformation of deep levels
relates to intrinsic defects, this mechanism seems unlikely.
Besides, data about defect clusters in the samples under investigation are absent.

In our opinion, the process of MWT influence was two-stage.
Initially, the resonant movement of the dislocation segment
has caused an increase in the concentration of interstitial atoms.
The approach of primary vacancy-related defects and secondary defects under
the ponderomotive forces action and subsequent defect reactions occurred at the final stage.

The additional information was added to the revised manuscript
(page ).


\vspace{1cm}
\noindent
\textcolor[rgb]{0.00,0.50,1.00}{\textbf{Comment~2.}}
\emph{The author's suggested modifications of the defects are too drastic.
Such the defect modifications change the deep level parameters more.
For example, deep level parameters for V\_Si V\_C and V\_C in silicon carbide have been
identified by both the experiments and theoretical calculations,
and their parameters are very different.
The small changes in deep level parameters should be other reasons.
They should be experimental errors or very small modifications of the defect structures.}

\noindent
\textcolor[rgb]{0.51,0.00,0.00}{\textbf{Reply:}}

The reviewer is quite right and vacancy--related defects in silicon carbide are investigated extensively ---
see, for instance, \cite{6HSiC:Vsi,4HSiC:Vc,6HSiC:VV2019,4HSiC:Vacan,SiC:defEPR,6HSiC:VPAS,4HSiC:VV,SiC:bookCh6,
6HSiC:Vsi2021,6HSiC:vac2021,4HSiC:NV,SiC:NV}.
In particular, it is known \cite{6HSiC:VV2019} that
6H--SiC gives rise to 3 configurations for carbon vacancy
and 6 configurations for the divacancy.
The further increase in deep levels, which are relevant to one defect, are connected to the different charge states.

The defect configurations were identified by using level location in the gap.
The primary indicator was the maximum agreement between determined values of $(E_c-E_t)$ and data given in 
previous publications (see References).
If the change in $(E_c-E_t)$ after MWT has exceeded the experimental errors limit,
we suggested about configuration change.
The level locations were determined from the dependency of the TAV relaxation time 
versus inverse temperature with enough high precision  (about 10 meV).
The typical dependencies are shown in Fig.~3.
In our opinion, the MWT-induced changes in deep level parameters ware not so small in most cases.
For example, the $(E_c-E_t)$ changes were 70~meV and 90~meV for SIC1-3 and GAT samples, respectively.
The minimal level shift, which was used as evidence of defect transformation, was 40~meV.




%\hl{ }


\cite{LightNeuIrrad:1,LightNeuIrrad:2,SiC:bookCh17}



Single crystalline SiC is a suitable material for realization of high power, high frequency and high
temperature devices

The font was enlarged.

\vspace{1cm}
\noindent
\textcolor[rgb]{0.00,0.50,1.00}{\textbf{Comment~3.}}
\emph{The adopted deep level observation techniuqe is not so common,
and it is difficult to judge measurement accuracy.
For confirmation of the results, conventional techniques such as DLTS should be employed simultaneously.}

\noindent
\textcolor[rgb]{0.51,0.00,0.00}{\textbf{Reply:}}
The font was enlarged.


\vspace{1cm}
\noindent
\textcolor[rgb]{0.00,0.50,1.00}{\textbf{Comment~4.}}
\emph{The materials observed are not the state of the art materials in industries.
Therefore, even if the results and physics are truth,
the impacts of this manuscript are limited.
If the microwave treatment has surely advantages compared
with conventional processes, the author can apply it to industry important materials, such as 4H-SiC or GaN.}


\noindent
\textcolor[rgb]{0.51,0.00,0.00}{\textbf{Reply:}}
The font was enlarged.


\subsection*{Response to Reviewer \#2 }
\noindent
\textcolor[rgb]{0.00,0.50,1.00}{\textbf{Comment~1.}}
\emph{I suggest the authors concentrating on a specific semiconductor,
digging deeply on the evolution of defects after MWT,
and discussing the effect of doping on the process.
The current work relates to the type of the host,
the single crystal/epitaxial layers, doping concentrations.
Readers lost easily during the introduction of results and discussions.}

\noindent
\textcolor[rgb]{0.51,0.00,0.00}{\textbf{Reply:}}
The text was revised.


\vspace{1cm}
\noindent
\textcolor[rgb]{0.00,0.50,1.00}{\textbf{Comment~2.}}
\emph{The authors carried out MWT experiments on differently doped SiC or GaAs.
Did the concentration of dopants change the evolution of point defects?
The Fermi level is different for differently doped hosts,
which charge states of intrinsic defects and MWT-induced defects, and thus the interaction, may be different.}

\noindent
\textcolor[rgb]{0.51,0.00,0.00}{\textbf{Reply:}}
The text was revised.


\vspace{1cm}
\noindent
\textcolor[rgb]{0.00,0.50,1.00}{\textbf{Comment~3.}}
\emph{The Pool-Frenkel effect is related to defect states of dislocations.
Yes, some works dealt the interaction between irradiation-induced point defects and dislocations,
such as Appl. Phys. Lett. 117, 023501 (2020) and J. Mater. Chem. C 9, 3177 (2021).
But I didn’t see any detailed discussion in this work.}

\noindent
\textcolor[rgb]{0.51,0.00,0.00}{\textbf{Reply:}}
The text was revised.


\vspace{1cm}
\noindent
\textcolor[rgb]{0.00,0.50,1.00}{\textbf{Comment~4.}}
\emph{Typos throughout the manuscript should be corrected, including but not limited to:}

\emph{(1) ``and semiconducting compounds including [6, 8]''}

\emph{(2) ``doping degree''}

\emph{(3) ``$0:31\div0:33$''}

\noindent
\textcolor[rgb]{0.51,0.00,0.00}{\textbf{Reply:}}

The text was revised.


\subsection*{Response to Reviewer \#3 }
\noindent
\textcolor[rgb]{0.00,0.50,1.00}{\textbf{Comment~1.}}
\emph{The reviewed manuscript presents results on investigation of deep levels
in various materials before and after microwave treatment with different durations (doses).
Therefore, the investigation is not well focused, since the nature of defects is different
for different materials, and they not necessarily should be governed by the same trends. }

\noindent
\textcolor[rgb]{0.51,0.00,0.00}{\textbf{Reply:}}
The text was revised.


\vspace{1cm}
\noindent
\textcolor[rgb]{0.00,0.50,1.00}{\textbf{Comment~2.}}
\emph{The results could present interest, even though the investigated materials are not the state of the art ones in industries.
However, the used experimental method is not widely approved,
and the identification of the microscopic nature of defects is not an easy
task even with widely used investigation methods such as DLTS or DLOS.
As a result, the discussions about the microscopic nature of the observed defects
and their reconfiguration during the microwave treatment
as well as the made conclusions are not enough convincing
(especially in cases when the deep level activation
energies do not change very much under microwave treatment,
as well as in cases when there is a large difference
between the experimental results and those coming from theoretical consideration).}

\noindent
\textcolor[rgb]{0.51,0.00,0.00}{\textbf{Reply:}}
The text was revised.

\vspace{1cm}
\noindent
\textcolor[rgb]{0.00,0.50,1.00}{\textbf{Comment~3.}}
\emph{I consider that the manuscript is not suitable for publication in the present form.
However, the reliability of the obtained results may be significantly improved,
if the study is complimented with other methods of investigation such as DLTS or DLOS,
at least for some of the investigated materials.
Moreover, the DLTS method also gives information about the concentration of deep levels.
In such a case, a comparison of parameters of deep levels obtained by TAV and DLTS,
combined with an analysis of the results of previously performed investigations,
may provide enough arguments confirming the
proposed microscopic nature of defects and their transformations.}

\noindent
\textcolor[rgb]{0.51,0.00,0.00}{\textbf{Reply:}}
The text was revised.

\cite{OstrovskiiSST}

\section*{References}

\bibliographystyle{iopart-num}
\bibliography{olikh}

\end{document}

