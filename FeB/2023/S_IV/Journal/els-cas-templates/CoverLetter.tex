


\documentclass[preprint]{elsarticle}

\begin{document}


To:
Expert Systems with Applications Editorial Board


Subject:
Article Submit

\vspace{5mm}
Dear Editors,

\vspace{3mm}
%Enclosed is a manuscript entitled “XXX XXXX XXXXX” by sb, which we are submitting for publication in the journal of …. We have chosen this journal because it deals with …
Enclosed with this letter you will find the electronic submission of manuscript entitled
``A test of meta-heuristic algorithms for parameter extraction of next-generation solar cells with S-shaped current-voltage curves'' O.~Olikh.



It is widely recognized that accurately determining the parameters of photovoltaic (PV) models
based on measured current-voltage (IV) characteristic curves is vital for the simulation, evaluation, and control of PV systems.
The commonly used traditional solar cell lumped-parameter models include the single-diode model, double-diode model, and three-diode model.
Many studies have investigated the use of meta-heuristic algorithms to find these parameters within these models.

However, the IV characteristics of new-generation PV devices, such as thin-film, organic, perovskite, and quantum dot solar cells, often exhibit an S-shaped deformation, requiring different models for accurate description.
The No Free Lunch theory suggests that no single meta-heuristic algorithm is universally effective for all optimization tasks.
Therefore, it is essential to accurately estimate parameters from S-shaped IV curves.
Surprisingly, to our knowledge, there are no studies that have identified the most suitable approach to solve these problems.
This study aims to compare the effectiveness of parameter estimation using the two-diode model, employing 14 different meta-heuristic algorithms, and determine the best-performing algorithm among them.
We have compared the results obtained from these algorithms using various nonparametric statistical methods.
%We strongly believe that this study, which involves testing and comparative analysis of different meta-heuristic algorithms for estimating solar cell parameters,
%will be of great interest to the readers.
%We strongly believe that this study, which involves testing and comparative analysis of different meta-heuristic algorithms,
%can provide a foundation for creating expert systems aimed at characterizing future generations of solar cells and
%will be of great interest to the readers.
We are confident that this research, involving the testing and comparison of various meta-heuristic algorithms, will lay the groundwork for developing expert systems to assess next-generation solar cells and will capture the readers' interest.

This is an original paper which has not been simultaneously submitted as a whole or in part anywhere else.
No elements of the work have been published in any form.
No conflict of interest exits in the submission of this manuscript.


We would  very much appreciate if you would consider the manuscript for publication in the \emph{Expert Systems with Applications}.
%We appreciate your consideration of our manuscript, and we look forward to receiving comments from the reviewers.

%Possible reviewers are the following.
%
%\begin{itemize}
%  \item Yimin Zhang,
%Shenyang University of Chemical Technology,
%Equipment Reliability Institute, Shenyang 110142, China,
%ymzhang@mail.neu.edu.cn
%  \item Kang Li,
%University of Leeds,
%School of Electronic and Electrical Engineering,  Leeds, LS2 9JT, UK,
%k.li1@leeds.ac.uk
%  \item Rebecca Saive,
%University of Twente Institute for Nanotechnology, Enschede 7522 NB, The Netherlands,
%r.saive@utwente.nl
%  \item Ripon Chakrabortty,
%University of New South Wales,
%School of Engineering and IT, Canberra, Australia,
%r.chakrabortty@adfa.edu.au
%%  \item Belen Arredondo,
%%Universidad Rey Juan Carlos
%%Área de Tecnología Electrónica, C/ Tulipán s/n, 28933 Móstoles, Spain,
%%belen.arredondo@urjc.es
%\end{itemize}


\vspace{3mm}

Sincerely yours,

Oleg~Olikh 


Taras Shevchenko National University of Kyiv


Kyiv 01601, Ukraine

E-mail: olegolikh@knu.ua


%Dear Editors
%It is more than 12 weeks since I submitted our manuscript (No: ) for possible publication in your journal. I have not yet received a reply and am wondering whether you have reached a decision. I should appreciated your letting me know what you have decided as soon as possible.








\end{document}

