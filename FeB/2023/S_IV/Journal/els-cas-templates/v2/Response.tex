

%\documentclass[aip,reprint]{revtex4-1}
%\documentclass[aip,jap,preprint]{revtex4-1}
\documentclass[a4paper,fleqn]{cas-sc}

\usepackage[numbers]{natbib}

\usepackage{graphicx}% Include figure files
\usepackage{dcolumn}
\usepackage{color}
%\draft % marks overfull lines with a black rule on the right

\begin{document}
\shorttitle{}


Dear Editor,

We like to express our appreciation to the reviewers for their comments.
We are resubmitting the revised version of the paper number MSB--D--24--00484.
We have studied the comments of the reviewer carefully, and have changed the text according to the comments they
have listed.
%The location of revisions is pointed by blue color in ``MarkedManuscript.pdf''.
Below we refer to each of the reviewer’s comments.


\subsection*{Response to Reviewer \#2 }

\noindent
\textcolor[rgb]{0.00,0.50,1.00}{\textbf{Comment~1.}}
\emph{The author should add the aim of the research in the introduction part.}

\noindent
\textcolor[rgb]{0.51,0.00,0.00}{\textbf{Reply:}}
This study aimed to compare the effectiveness of meta-heuristic algorithms in extracting the parameters of NG-SCs from the S-shaped \emph{IV} curves
and to determine the most suitable ones for addressing this optimization problem.

The aim of the research was added in the introduction part (page~2, last paragraph in the Introduction).


\vspace{1cm}
\noindent
\textcolor[rgb]{0.00,0.50,1.00}{\textbf{Comment~2.}}
\emph{Figure captions are poorly written. Pay attention to them, besides i couldn't find Fig.~3 in the texts.}

\noindent
\textcolor[rgb]{0.51,0.00,0.00}{\textbf{Reply:}}
Figure captions were revised.

The Fig.~3 is referenced on page 5 (last line in Subsection~2.2.2) and on page 16 (first line in Subsection~3.3.2)
in the revised manuscript.


\vspace{1cm}
\noindent
\textcolor[rgb]{0.00,0.50,1.00}{\textbf{Comment~3.}}
\emph{Figure~4, 5 and 6 are hard to understand for authors, try to make it clear.}

\noindent
\textcolor[rgb]{0.51,0.00,0.00}{\textbf{Reply:}}
The text was revised.

\vspace{1cm}
\noindent
\textcolor[rgb]{0.00,0.50,1.00}{\textbf{Comment~4.}}
\emph{Details about equivalent circuit model of a solar cell is insufficient.
How it is useful and novel compared with one diode model.}



\noindent
\textcolor[rgb]{0.51,0.00,0.00}{\textbf{Reply:}}
Details about equivalent circuit model were added.
Additionally, the study included a comparison between elements of
the opposed two--diode model and the single-diode model.
The relevant information can be found on page 3, paragraphs 2 and 3 from the top.


\vspace{1cm}
\noindent
\textcolor[rgb]{0.00,0.50,1.00}{\textbf{Comment~5.}}
\emph{The artwork quality in this manuscript could be improved thoroughly.
Figures should be placed in a uniform location with a uniform font, size, figures are unclear.}

\noindent
\textcolor[rgb]{0.51,0.00,0.00}{\textbf{Reply:}}
The text was revised.

\vspace{1cm}
\noindent
\textcolor[rgb]{0.00,0.50,1.00}{\textbf{Comment~6.}}
\emph{Table~3 need to modify, text size and structure should be similar in whole manuscript.}

\noindent
\textcolor[rgb]{0.51,0.00,0.00}{\textbf{Reply:}}
The text was revised.

\vspace{1cm}
\noindent
\textcolor[rgb]{0.00,0.50,1.00}{\textbf{Comment~7.}}
\emph{Please rectify all the English language mistakes.}

\noindent
\textcolor[rgb]{0.51,0.00,0.00}{\textbf{Reply:}}
We are sorry for our English.
The text has been revised by a bilingual speaker, and we hope for the language improvement.


\vspace{1cm}
\noindent
\textcolor[rgb]{0.00,0.50,1.00}{\textbf{Comment~8.}}
\emph{Please check the grammar.}

\noindent
\textcolor[rgb]{0.51,0.00,0.00}{\textbf{Reply:}}
The text was revised.
We are really hoping to see a significant decrease in the number of grammatical errors.

\vspace{1cm}
\subsection*{Response to Reviewer \#4 }


\noindent
\textcolor[rgb]{0.00,0.50,1.00}{\textbf{Comment~1.}}
\emph{The paper would benefit from more detailed descriptions of the individual meta-heuristic algorithms used,
especially the lesser-known STLBO and ADELI.
Explaining their mechanisms, strengths, and why they are particularly suited to solving the parameter estimation problem
for S-shaped IV curves would help readers understand their effective application in this
context added recent LR}

\emph{
``1. Zhu, J., Chaturvedi, R., Fouad, Y., Albaijan, I., Juraev, N., Alzubaidi, L. H.,... Garalleh, H. A. L. (2024).
A numerical modeling of battery thermal management system using nano-enhanced phase
change material in hot climate conditions. Case Studies in Thermal Engineering, 58, 104372.
doi: https://doi.org/10.1016/j.csite.2024.104372
}

\emph{
2. Zhang, X., Tang, Y., Zhang, F., \& Lee, C. (2016).
A Novel Aluminum-Graphite Dual-Ion Battery. Advanced energy materials, 6(11), 1502588.
doi: 10.1002/aenm.201502588
}

\emph{
3. Wang, M., Jiang, C., Zhang, S., Song, X., Tang, Y.,... Cheng, H. (2018).
Reversible calcium alloying enables a practical room-temperature rechargeable
calcium-ion battery with a high discharge voltage. Nature chemistry, 10(6), 667-672.
doi: 10.1038/s41557-018-0045-4
}

\emph{
4. Su, Y., Shang, J., Liu, X., Li, J., Pan, Q.,... Tang, Y. (2024).
Constructing p-p Superposition Effect of Tetralithium Naphthalenetetracarboxylate
with Electron Delocalization for Robust Dual-ion Batteries.
Angewandte Chemie International Edition, e202403775.
doi: https://doi.org/10.1002/anie.202403775
}

\emph{
5. Zhang, M., Zhang, W., Zhang, F., Lee, C., \& Tang, Y. (2024).
Anion-hosting cathodes for current and late-stage dual-ion batteries.
Science China Chemistry.
doi: 10.1007/s11426-023-1957-3
}

\emph{
6. Liu, Q., Liu, L., Zheng, Y., Li, M., Ding, B., Diao, X.,... Tang, Y. (2024).
On-demand engineerable visible spectrum by fine control of electrochemical reactions.
National Science Review, 11(3), nwad323.
doi: https://doi.org/10.1093/nsr/nwad323
}

\emph{
7. Zhu, C. (2023).
Optimizing and using AI to study of the cross-section of finned
tubes for nanofluid-conveying in solar panel cooling with phase change materials.
Engineering Analysis with Boundary Elements, 157, 71-81.
doi: https://doi.org/10.1016/j.enganabound.2023.08.018''.
}

\noindent
\textcolor[rgb]{0.51,0.00,0.00}{\textbf{Reply:}}

ADELI improved version of DE incorporates three main elements:
local search using Lagrange interpolation, self-adaptive DE control parameter settings,
and an adaptive mutational strategy \cite{ADELI}.
The first element involves interpolating three potential solutions using a
polynomial function to calculate the local minimum value.
The self-adaptive control parameter settings include randomly altering the
scaling factor and crossover rate values in each iteration.
The adaptive mutational strategy determines the probability of
employing Lagrange interpolation in each generation based on the best fitness function value.
These incorporations aim to enhance exploitation capability and speed up the convergence.

The information was added to the revised manuscript (page~6, paragraph~2).

In STLBO, the teacher phase is redefined and simplified, while the learner phase remains unchanged \cite{STLBO}.
In the redefined teacher phase, the mutation of potential solutions is possible,
with the mutation probability decreasing as the iteration number increases.
During the early stages, a higher mutation probability helps explore a larger solution space
and approach the optimum quickly.
However, in the latter stages of optimization, when the teacher (best solution) is near the global optimum,
a lower mutation probability serves as a fine-tuning mechanism
to enhance local search capability, proving to be an effective strategy.
To enrich the mutation behavior, a chaotic sequence is introduced to generate values for the mutation parameters.
A chaotic sequence is a deterministic, random-like process found in nonlinear dynamic systems,
which is non-periodic, non-converging, and bounded \cite{May1976}.
Additionally, the elite strategy replaces the worst solutions in the current population with new solutions based on objective function values.

The information was added to the revised manuscript (page~7, paragraph~3).

Unfortunately, providing a detailed description of each algorithm requires too much space.
However, the paper includes references that allow obtaining additional information.

Furthermore, some speculation about possible modification of meta--heuristic algorithms
to successful solving the parameter estimation problem for S-shaped \emph{IV} curves was added (page 17, last paragraph before Conclusion)

We would also like to express our gratitude to the Reviewer for providing exceptionally insightful references.
We hope the Reviewer will not object to their inclusion in the revised manuscript (references 1, 2, 3).

Additionally, the idea of employing meta-heuristic algorithms during the optimization of dual-ion batteries appears to be particularly intriguing.



\vspace{1cm}
\noindent
\textcolor[rgb]{0.00,0.50,1.00}{\textbf{Comment~2.}}
\emph{While the use of synthetic IV curves is mentioned, the process for generating
these curves could be elaborated upon.
Details such as the assumptions made, the variability in parameters used to generate the curves,
and how these synthetic models compare to actual solar cell data would greatly enhance the transparency and reproducibility of the results.}

\noindent
\textcolor[rgb]{0.51,0.00,0.00}{\textbf{Reply:}}
In the first part of the study, the performance of meta--heuristic algorithms for parameter estimation was evaluated using a single \emph{IV} curve.
This curve represents the experimental data of bulk heterojunction photocells prepared
using a composite of $p$--DTS(FBTTh$_2$)$_2$ and neat C$_{70}$ \cite{Tada2015Organic}.
The information was added to the revised manuscript (page~4, paragraph~1).

In the second part, we simulated a set of \emph{IV} characteristics.
These curves correspond to the temperature range from 260~K to 350~K.
In the simulation process, we considered the temperature dependencies of the parameters.
We based our approach on known physical mechanisms of current flow in NG-SCs and
used the reported temperature dependencies of saturation current, ideality factor, shunt resistance, and series resistance.
However, the main focus was on achieving a diversity of parameter ratios rather
than attempting to precisely replicate real--life PV converters of a specific type.
Furthermore, an S--shaped \emph{IV} curve is observed in various types of solar cells,
and diverse charge transport mechanisms significantly complicate the selection of a single possible temperature dependence for each of the eight model parameters.
The specific temperature dependencies and parameter values utilized are detailed in Subsection~2.2.2.

The \emph{IV} curves ware simulated using Eqs.~(1)--(2) over a voltage range of 0-0.8~V (in ‘‘Single--\emph{IV} case’’) or
0-1.0~V (in ‘‘\emph{IV}--set case’’) with a 10~mV step.
The information was added to the revised manuscript (last paragraphs of Subsection~2.2.1 and Subsection~2.2.2).


\vspace{1cm}
\noindent
\textcolor[rgb]{0.00,0.50,1.00}{\textbf{Comment~3.}}
\emph{The conclusion hints at future research into the impact of noisy data on algorithm performance.
It would strengthen the current study to include preliminary investigations or simulations on how noise influences the effectiveness of the meta-heuristic algorithms.
Even basic insights could provide significant value in understanding the robustness of these algorithms under practical conditions.}

\noindent
\textcolor[rgb]{0.51,0.00,0.00}{\textbf{Reply:}}
The text was revised.


\vspace{1cm}
\noindent
\textcolor[rgb]{0.00,0.50,1.00}{\textbf{Comment~4.}}
\emph{The paper mentions the use of nonparametric statistical methods but could include more comprehensive details about these analyses.
For instance, discussing why particular tests were chosen, the statistical significance levels,
 and the interpretation of these tests would help in understanding the comparative analysis more deeply.}

\noindent
\textcolor[rgb]{0.51,0.00,0.00}{\textbf{Reply:}}

Wilcoxon test is used to assess whether there are statistically significant differences between pairs of algorithms.
Meanwhile, the Friedman, Friedman Aligned, and Quade tests are employed when it's necessary to compare three or more related groups of results (algorithms).
Friedman test evaluates whether there are statistically significant differences between the medians of the ranks of these algorithms.
Friedman Aligned Ranks test addresses the issue of rank correlation in the original Friedman test, providing more precise results.
Finally, the Quade test helps account for the effects of observed factors, such as random variations,
to more accurately determine the statistical differences between groups.

The main drawback of the Friedman, Friedman Aligned, and Quade tests is that
they can only detect significant differences over the whole set of multiple comparisons,
making it difficult to establish proper comparisons between specific algorithms \cite{Derrac2011}.
To address these issues, it is necessary to employ post-hoc procedures.
Post-hoc methods are applied after the initial analysis and allow for controlling the overall error rate
when comparing multiple algorithms, thereby reducing the likelihood of randomly identifying statistically significant differences.

$1\times N$ designs help determine if there are statistically significant differences between one algorithm
(the control algorithm) and each of the other algorithms.
Multidimensional comparisons $N\times N$ designs involve analyzing statistical differences between all possible pairs of algorithms.
Typically, different post-hoc procedures are used for $1\times N$ and $N\times N$ comparisons.
Description of all the used post-hoc procedures can be found in Derrac \emph{et al.} \cite{Derrac2011}.

All mentioned methods are standard for comparing metaheuristic algorithms \cite{Derrac2011}.
Their comprehensive application enables making the most well-founded conclusions.

In most cases, the study used the widely accepted significance level of 0.05.

The information was added to the revised manuscript (page~8, last 4 paragraphs).



\vspace{1cm}
\noindent
\textcolor[rgb]{0.00,0.50,1.00}{\textbf{Comment~5.}}
\emph{ Adding visual representations such as scatter plots, error distributions,
or convergence graphs of the algorithms' performance over iterations could provide intuitive insights into their behavior.
This would also help in visually comparing the efficiency and accuracy of the algorithms beyond the tabular or textual data presented.}

\noindent
\textcolor[rgb]{0.51,0.00,0.00}{\textbf{Reply:}}

The convergence graphs for  algorithms were added (see Fig.S1 in the supplementary materials).
Information about the existence of convergence graphs has been added to the revised manuscript (first paragraph in Subsection 3.1).

Besides the scatter plots for parameter determination results are represented in Figs.~S15-S22 (supplementary materials).

Simultaneously, the interquartile range (IQR) for estimated parameter values deals with error distributions.
IQR data is represented in Figs.~4, 9, and S6--S14.
Therefore, we believe that the separate error distribution graphs are somewhat redundant.


\vspace{1cm}
\noindent
\textcolor[rgb]{0.00,0.50,1.00}{\textbf{Comment~6.}}
\emph{ the paper could discuss the practical implications of implementing these algorithms
in real-world solar cell testing and production environments.
Suggestions on how to integrate these algorithms into existing systems, the expected improvements
in parameter estimation accuracy, and potential challenges in deployment would be beneficial for practitioners in the field.}

\noindent
\textcolor[rgb]{0.51,0.00,0.00}{\textbf{Reply:}}
The text was revised.
the text was revised (page~5, column~1, paragraph~2).

The ``relative deformation'' was meant.
The term ``strain'' is used in the revised version.

\bibliographystyle{model1-num-names}
\bibliography{olikh_Methods}


\end{document}

