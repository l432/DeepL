

%\documentclass[aip,reprint]{revtex4-1}
%\documentclass[aip,jap,preprint]{revtex4-1}
\documentclass[a4paper,fleqn]{cas-sc}

\usepackage[numbers]{natbib}

\usepackage{graphicx}% Include figure files
\usepackage{dcolumn}
\usepackage{color}
%\draft % marks overfull lines with a black rule on the right

\begin{document}
\shorttitle{}


Dear editor,

We like to express our appreciation to the reviewers for their comments.
We are resubmitting the revised version of the paper number MSB--D--24--00484.
We have studied the comments of the reviewer carefully, and have changed the text according to the comments they
have listed.
%The location of revisions is pointed by blue color in ``MarkedManuscript.pdf''.
Below we refer to each of the reviewer’s comments.


\subsection*{Response to Reviewer \#2 }

\noindent
\textcolor[rgb]{0.00,0.50,1.00}{\textbf{Comment~1.}}
\emph{The author should add the aim of the research in the introduction part.}

\noindent
\textcolor[rgb]{0.51,0.00,0.00}{\textbf{Reply:}}
The text was revised.

\vspace{1cm}
\noindent
\textcolor[rgb]{0.00,0.50,1.00}{\textbf{Comment~2.}}
\emph{Figure captions are poorly written. Pay attention to them, besides i couldn't find Fig.~3 in the texts.}

\noindent
\textcolor[rgb]{0.51,0.00,0.00}{\textbf{Reply:}}
The text was revised.

\vspace{1cm}
\noindent
\textcolor[rgb]{0.00,0.50,1.00}{\textbf{Comment~3.}}
\emph{Figure~4, 5 and 6 are hard to understand for authors, try to make it clear.}

\noindent
\textcolor[rgb]{0.51,0.00,0.00}{\textbf{Reply:}}
The text was revised.

\vspace{1cm}
\noindent
\textcolor[rgb]{0.00,0.50,1.00}{\textbf{Comment~4.}}
\emph{Details about equivalent circuit model of a solar cell is insufficient.}

\noindent
\textcolor[rgb]{0.51,0.00,0.00}{\textbf{Reply:}}
The text was revised.

\vspace{1cm}
\noindent
\textcolor[rgb]{0.00,0.50,1.00}{\textbf{Comment~5.}}
\emph{The artwork quality in this manuscript could be improved thoroughly. 
Figures should be placed in a uniform location with a uniform font, size, figures are unclear.}

\noindent
\textcolor[rgb]{0.51,0.00,0.00}{\textbf{Reply:}}
The text was revised.

\vspace{1cm}
\noindent
\textcolor[rgb]{0.00,0.50,1.00}{\textbf{Comment~6.}}
\emph{How it is useful and novel compared with one diode model. 
Table~3 need to modify, text size and structure should be similar in whole manuscript.}

\noindent
\textcolor[rgb]{0.51,0.00,0.00}{\textbf{Reply:}}
The text was revised.

\vspace{1cm}
\noindent
\textcolor[rgb]{0.00,0.50,1.00}{\textbf{Comment~7.}}
\emph{Please rectify all the English language mistakes.}

\noindent
\textcolor[rgb]{0.51,0.00,0.00}{\textbf{Reply:}}
The text was revised.

\vspace{1cm}
\noindent
\textcolor[rgb]{0.00,0.50,1.00}{\textbf{Comment~8.}}
\emph{Please check the grammar.}

\noindent
\textcolor[rgb]{0.51,0.00,0.00}{\textbf{Reply:}}
The text was revised.


\vspace{1cm}
\subsection*{Response to Reviewer \#4 }


\noindent
\textcolor[rgb]{0.00,0.50,1.00}{\textbf{Comment~1.}}
\emph{The paper would benefit from more detailed descriptions of the individual meta-heuristic algorithms used, 
especially the lesser-known STLBO and ADELI. 
Explaining their mechanisms, strengths, and why they are particularly suited to solving the parameter estimation problem 
for S-shaped IV curves would help readers understand their effective application in this 
context added recent LR} 

\emph{
``1. Zhu, J., Chaturvedi, R., Fouad, Y., Albaijan, I., Juraev, N., Alzubaidi, L. H.,... Garalleh, H. A. L. (2024). 
A numerical modeling of battery thermal management system using nano-enhanced phase 
change material in hot climate conditions. Case Studies in Thermal Engineering, 58, 104372. 
doi: https://doi.org/10.1016/j.csite.2024.104372 
}

\emph{
2. Zhang, X., Tang, Y., Zhang, F., \& Lee, C. (2016). 
A Novel Aluminum-Graphite Dual-Ion Battery. Advanced energy materials, 6(11), 1502588. 
doi: 10.1002/aenm.201502588 
}

\emph{
3. Wang, M., Jiang, C., Zhang, S., Song, X., Tang, Y.,... Cheng, H. (2018). 
Reversible calcium alloying enables a practical room-temperature rechargeable 
calcium-ion battery with a high discharge voltage. Nature chemistry, 10(6), 667-672. 
doi: 10.1038/s41557-018-0045-4 
}

\emph{
4. Su, Y., Shang, J., Liu, X., Li, J., Pan, Q.,... Tang, Y. (2024). 
Constructing p-p Superposition Effect of Tetralithium Naphthalenetetracarboxylate 
with Electron Delocalization for Robust Dual-ion Batteries. 
Angewandte Chemie International Edition, e202403775. 
doi: https://doi.org/10.1002/anie.202403775 
}

\emph{
5. Zhang, M., Zhang, W., Zhang, F., Lee, C., \& Tang, Y. (2024). 
Anion-hosting cathodes for current and late-stage dual-ion batteries. 
Science China Chemistry. 
doi: 10.1007/s11426-023-1957-3 
}

\emph{
6. Liu, Q., Liu, L., Zheng, Y., Li, M., Ding, B., Diao, X.,... Tang, Y. (2024). 
On-demand engineerable visible spectrum by fine control of electrochemical reactions. 
National Science Review, 11(3), nwad323. 
doi: https://doi.org/10.1093/nsr/nwad323 
}

\emph{
7. Zhu, C. (2023). 
Optimizing and using AI to study of the cross-section of finned 
tubes for nanofluid-conveying in solar panel cooling with phase change materials. 
Engineering Analysis with Boundary Elements, 157, 71-81. 
doi: https://doi.org/10.1016/j.enganabound.2023.08.018''.
}

\noindent
\textcolor[rgb]{0.51,0.00,0.00}{\textbf{Reply:}}
The text was revised.


\vspace{1cm}
\noindent
\textcolor[rgb]{0.00,0.50,1.00}{\textbf{Comment~2.}}
\emph{While the use of synthetic IV curves is mentioned, the process for generating 
these curves could be elaborated upon. 
Details such as the assumptions made, the variability in parameters used to generate the curves, 
and how these synthetic models compare to actual solar cell data would greatly enhance the transparency and reproducibility of the results.}

\noindent
\textcolor[rgb]{0.51,0.00,0.00}{\textbf{Reply:}}
The text was revised.


\vspace{1cm}
\noindent
\textcolor[rgb]{0.00,0.50,1.00}{\textbf{Comment~3.}}
\emph{The conclusion hints at future research into the impact of noisy data on algorithm performance. 
It would strengthen the current study to include preliminary investigations or simulations on how noise influences the effectiveness of the meta-heuristic algorithms. 
Even basic insights could provide significant value in understanding the robustness of these algorithms under practical conditions.}

\noindent
\textcolor[rgb]{0.51,0.00,0.00}{\textbf{Reply:}}
The text was revised.


\vspace{1cm}
\noindent
\textcolor[rgb]{0.00,0.50,1.00}{\textbf{Comment~4.}}
\emph{The paper mentions the use of nonparametric statistical methods but could include more comprehensive details about these analyses. 
For instance, discussing why particular tests were chosen, the statistical significance levels,
 and the interpretation of these tests would help in understanding the comparative analysis more deeply.}

\noindent
\textcolor[rgb]{0.51,0.00,0.00}{\textbf{Reply:}}
The text was revised.


\vspace{1cm}
\noindent
\textcolor[rgb]{0.00,0.50,1.00}{\textbf{Comment~5.}}
\emph{ Adding visual representations such as scatter plots, error distributions, 
or convergence graphs of the algorithms' performance over iterations could provide intuitive insights into their behavior. 
This would also help in visually comparing the efficiency and accuracy of the algorithms beyond the tabular or textual data presented.}

\noindent
\textcolor[rgb]{0.51,0.00,0.00}{\textbf{Reply:}}
The text was revised.


\vspace{1cm}
\noindent
\textcolor[rgb]{0.00,0.50,1.00}{\textbf{Comment~6.}}
\emph{ the paper could discuss the practical implications of implementing these algorithms 
in real-world solar cell testing and production environments. 
Suggestions on how to integrate these algorithms into existing systems, the expected improvements 
in parameter estimation accuracy, and potential challenges in deployment would be beneficial for practitioners in the field.}

\noindent
\textcolor[rgb]{0.51,0.00,0.00}{\textbf{Reply:}}
The text was revised.
the text was revised (page~5, column~1, paragraph~2).

The ``relative deformation'' was meant.
The term ``strain'' is used in the revised version.

\bibliographystyle{model1-num-names}
\bibliography{olikh_Methods}
%\bibliography{olikh}

\end{document}

