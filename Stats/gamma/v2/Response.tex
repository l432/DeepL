

%\documentclass[aip,graphicx]{revtex4-1}
\documentclass[aip,jap,preprint]{revtex4-1}
\usepackage{graphicx}% Include figure files
\usepackage{dcolumn}
\usepackage{color}
%\draft % marks overfull lines with a black rule on the right
\bibliographystyle{elsarticle-num}
\begin{document}
Dear editor,

We like to express our appreciation to the reviewers for their comments.
We are resubmitting the revised version of the paper number  SSE-2019-316.
We have studied the comments of the reviewer carefully, and have changed the text according to the comments they
have listed.
Below we refer to each of the reviewer’s comments.



\subsection*{Response to Reviewer \#1 }

\noindent
\textcolor[rgb]{0.00,0.50,1.00}{\textbf{Comment~1.}}
\emph{The main conclusions (and the highlights) were in fact drawn before and according to more consistent studies.}

\noindent
\textcolor[rgb]{0.51,0.00,0.00}{\textbf{Reply:}}
The correction was done.



\noindent
\textcolor[rgb]{0.00,0.50,1.00}{\textbf{Comment~2.}}
\emph{The claimed study of carrier transport is very poor, the investigations being reduced to the measurement and fit of a few Current-Voltage (I-V) characteristics recorded on Au-SiO2-Si MOS capacitors at room temperature (295K?!, for all the cases the same?): as grown, as irradiated with Co60-gamma and irradiated followed by ultrasound treatments of 30 and 60 minutes.}

\noindent
\textcolor[rgb]{0.51,0.00,0.00}{\textbf{Reply:}}
The text was revised.

\noindent
\textcolor[rgb]{0.00,0.50,1.00}{\textbf{Comment~3.}}
\emph{The temperature dependence was measured only in a tiny range and only for fw biases (between 304K and 320K in Fig.3, although the fit of the I-V, shown in Fig.1, seems to be for 295K!).}


\noindent
\textcolor[rgb]{0.51,0.00,0.00}{\textbf{Reply:}}
The parameters were defined in the Table 1 caption.



\noindent
\textcolor[rgb]{0.00,0.50,1.00}{\textbf{Comment~4.}}
\emph{The possible dependencies on the oxide thickness are neglected (the thickness of the oxide and of the depletion layer in Si are not even mentioned in the manuscript) as well as on the defect distribution (mentioned as nonhomogeneous for E’-center at page 8). A positive charge in the oxide would move the flat band voltage, shifting and disturbing, when nothomogeneous, the shape of the I-V curves.}

\emph{Thus, it is hard to understand how relevant are the fit formulas for the device I-V characteristics.}


\noindent
\textcolor[rgb]{0.51,0.00,0.00}{\textbf{Reply:}}
The parameters were defined in the Table 1 caption.




\noindent
\textcolor[rgb]{0.00,0.50,1.00}{\textbf{Comment~5.}}
\emph{Also, the English is poor.}


\noindent
\textcolor[rgb]{0.51,0.00,0.00}{\textbf{Reply:}}
The text was revised.


\subsection*{Response to Reviewer \#2 }

\noindent
\textcolor[rgb]{0.00,0.50,1.00}{\textbf{Comment~1.}}
\emph{ The paper is about influence of gamma-irradiation on carrier transport in Au-SiO2-Si structure. However, gamma-irradiation mainly influences on bulk Si rather than its surface, which is too thin. I feel that the discussions regarding the Si surface passivation are not relevant to main focus of the article.}

\noindent
\textcolor[rgb]{0.51,0.00,0.00}{\textbf{Reply:}}
The reviewer is correct and gamma-irradiation mainly influences on bulk Si.
But irradiation leads to creation of defects both at Si/SiO$_2$  interface and in thin oxide layer as well.
In fact, total concentrations of $P_b$--centers and $E'$--centers are about $10^{18}$~cm$^{-3}$ 
in the case of 10~Mrad dose of ionizing radiation \cite{Fleetwood,PersenkovBook}. 
Our work focuses on current mechanisms in MOS structure.
These mechanisms mainly deals with a defects, located at the near interface region. 


it  is generally accepted that  they  anneal by  the 
trapping  of  some diffusing molecular species  such as 02 
[20], H2 [21], H20 [221,




\noindent
\textcolor[rgb]{0.00,0.50,1.00}{\textbf{Comment~2.}}
\emph{ The idea about ultrasound-induced hydrogen diffusion is interesting, however, there is no discussion about hydrogen source. The only source could be the hydrogen used in surface treatment, but that concentration of that hydrogen is not enough to cause drastic influence on electrical properties of bulk Si irradiated by gamma particles. This part of the discussions is speculative.}

\noindent
\textcolor[rgb]{0.51,0.00,0.00}{\textbf{Reply:}}
More detailed information about parameter calculation was added (the first paragraph in page 6).



\noindent
\textcolor[rgb]{0.00,0.50,1.00}{\textbf{Comment~3.}}
\emph{ Some important References are not cited, such as, e.g., A Davletova, et. al. J. Phys. Chem. Solids 70 (6), 989-992; J. Phys. D: Appl. Phys. 41 (16), 165107.}

\noindent
\textcolor[rgb]{0.51,0.00,0.00}{\textbf{Reply:}}
References list was expanded.


\bibliography{olikh}

\end{document}

