

%\documentclass[aip,graphicx]{revtex4-1}
\documentclass[aip,jap,preprint]{revtex4-1}
\usepackage{graphicx}% Include figure files
\usepackage{dcolumn}
\usepackage{color}
%\draft % marks overfull lines with a black rule on the right
\bibliographystyle{elsarticle-num}
\begin{document}
Dear editor,

We like to express our appreciation to the reviewers for their comments.
We are resubmitting the revised version of the paper number  SSE-2019-316.
We have studied the comments of the reviewer carefully, and have changed the text according to the comments they
have listed.
Below we refer to each of the reviewer’s comments.



\subsection*{Response to Reviewer \#1 }

\noindent
\textcolor[rgb]{0.00,0.50,1.00}{\textbf{Comment~1.}}
\emph{The main conclusions (and the highlights) were in fact drawn before and according to more consistent studies.}

\noindent
\textcolor[rgb]{0.51,0.00,0.00}{\textbf{Reply:}}

The reviewer is correct and our work is not pioneering observations of acoustically induced annealing of defects.
In particular, a whole number of References are cited in Introduction.
However, not all radiation defects are acoustically active and annealed by ultrasound treatment (UST).
To the best of our knowledge, only Parchinskii \emph{et al}. \cite{Parchinskii2000,Parchinskii2006} investigated UST influence on
$\gamma$--modified Si--SiO$_2$ structure and there is no published paper which describes an UST effect on charge transfer in irradiated Si--SiO$_2$.
In our case using of high--resistance substrate allowed to clearly mark out both current components, concerned with radiation defect, and UST effect on them.
Basing on known typical radiation defects in Si--SiO$_2$ and obtained results it was concluded about low temperature annealing of $P_b$--centers and $E'$--centers.
It is not previously reported about acousto--activity of  $P_b$--centers and $E'$--centers.

In addition, it is well known that irradiation leads to increase in leakage current.
It is well known that the majority of leakage current mechanisms in MOS structures
deal with a defects, located at the near interface region.
But there is a wide variety of such mechanisms:
the thermionic trap-assisted tunneling,
the space--charge limited current,
the thermally--assisted variable--range--hopping conduction,
the phonon--assisted tunneling etc.
The global decision about nature of irradiation induced current is absent 
and the concrete charge transfer mechanism depend on type of irradiating particles, doze,  properties of semiconductor and oxide layer. 
This work has decided such task for the case of $5\cdot10^7$~rad $\gamma$--irradiation of Si--SiO$_2$ with high--resistance substrate and native oxide layer. 
The main conclusions and the highlights were corrected in accordance with aforesaid.

To conclude, we believe strongly  that the paper is an important addition to the literature and is
not a variation but rather expansion of preceding studies.


\noindent
\textcolor[rgb]{0.00,0.50,1.00}{\textbf{Comment~2.}}
\emph{The claimed study of carrier transport is very poor, the investigations being reduced to the measurement and fit of a few Current-Voltage (I-V) characteristics recorded on Au-SiO2-Si MOS capacitors at room temperature (295K?!, for all the cases the same?): as grown, as irradiated with Co60-gamma and irradiated followed by ultrasound treatments of 30 and 60 minutes.}

\noindent
\textcolor[rgb]{0.51,0.00,0.00}{\textbf{Reply:}}
The text was revised.



\noindent
\textcolor[rgb]{0.00,0.50,1.00}{\textbf{Comment~3.}}
\emph{The temperature dependence was measured only in a tiny range and only for fw biases (between 304K and 320K in Fig.3, although the fit of the I-V, shown in Fig.1, seems to be for 295K!).}


\noindent
\textcolor[rgb]{0.51,0.00,0.00}{\textbf{Reply:}}
The fit of the $I-V$ curves, shown in Fig.~1, was done for 300~K.
We apologise for mistake in figure caption.
The correction was done.
The revised Fig.3 includes data over a temperature range of 300-340~K.
This range is not huge as well, but following defence should be taken into account if possible.
On the one hand, the lower temperature limit is restricted by setup sensitivity,
the upper temperature limit is restricted by desire to avoid an annealing of defect.
On the other hand, the current value at 340~K is 5 times as much as one at 300~K
% has increased five-fold over a used range
%changed by five times (see Fig.3)
and the activation energy can be estimated precisely enough.

The temperature dependence of current in the case of  trap--assisted tunneling is tangled sufficiently  \cite{TAT:Gilmore,TAT:GopalSST,TAT:Gopal} and
measured reverse biased curves do not used to illustrate  results.


\noindent
\textcolor[rgb]{0.00,0.50,1.00}{\textbf{Comment~4.}}
\emph{The possible dependencies on the oxide thickness are neglected (the thickness of the oxide and of the depletion layer in Si are not even mentioned in the manuscript) as well as on the defect distribution (mentioned as nonhomogeneous for E’-center at page 8). A positive charge in the oxide would move the flat band voltage, shifting and disturbing, when nothomogeneous, the shape of the I-V curves.}

\emph{Thus, it is hard to understand how relevant are the fit formulas for the device I-V characteristics.}


\noindent
\textcolor[rgb]{0.51,0.00,0.00}{\textbf{Reply:}}
The parameters were defined in the Table 1 caption.
3.5e12 cm-3
1e-5m=1e-3cm=10mkm



\noindent
\textcolor[rgb]{0.00,0.50,1.00}{\textbf{Comment~5.}}
\emph{Also, the English is poor.}


\noindent
\textcolor[rgb]{0.51,0.00,0.00}{\textbf{Reply:}}
The text was revised.


\subsection*{Response to Reviewer \#2 }

\noindent
\textcolor[rgb]{0.00,0.50,1.00}{\textbf{Comment~1.}}
\emph{ The paper is about influence of gamma-irradiation on carrier transport in Au-SiO2-Si structure. However, gamma-irradiation mainly influences on bulk Si rather than its surface, which is too thin. I feel that the discussions regarding the Si surface passivation are not relevant to main focus of the article.}

\noindent
\textcolor[rgb]{0.51,0.00,0.00}{\textbf{Reply:}}
The reviewer is correct and gamma-irradiation mainly influences on bulk Si.
But irradiation leads to creation of defects both at Si/SiO$_2$  interface and in thin oxide layer as well.
In fact, total concentrations of $P_b$--centers and $E'$--centers are about $10^{18}$~cm$^{-3}$
in the case of 10~Mrad dose of ionizing radiation \cite{Fleetwood,PersenkovBook}.
Our work focuses on current mechanisms in MOS structure.
These mechanisms are mainly determined by near interface region.
Our results testify to the annealing  of defects, located at interface region.
In turn, it  is generally accepted \cite{SiO2:Devine,SiO2:Mahapatra} that $P_b$--centers (broken interfacial $\equiv\!\mathrm{Si}\!-\!\mathrm{H}$ bonds) and $E'$--centers (broken $\equiv\!\mathrm{Si}\!-\!\mathrm{O}$ bonds) anneal by  the
trapping  of  some  diffusing molecular species  such as  $\text{O}_2$, $\text{H}_2$, ...
Therefore we are forced to discuss regarding passivation of near interface dangling bonds.




\noindent
\textcolor[rgb]{0.00,0.50,1.00}{\textbf{Comment~2.}}
\emph{ The idea about ultrasound-induced hydrogen diffusion is interesting, however, there is no discussion about hydrogen source. The only source could be the hydrogen used in surface treatment, but that concentration of that hydrogen is not enough to cause drastic influence on electrical properties of bulk Si irradiated by gamma particles. This part of the discussions is speculative.}

\noindent
\textcolor[rgb]{0.51,0.00,0.00}{\textbf{Reply:}}

Indeed, the native oxidation of Si surface is a hydrogen source.
The both dry and wet oxidation processes take place in this case:
\begin{eqnarray}
\text{Si} + \text{O}_2&\rightarrow&\text{SiO}_2\,, \nonumber \\
\text{Si} + 2\text{H}_2\text{O}&\rightarrow&\text{SiO}_2+2\text{H}_2\,.\nonumber
\end{eqnarray}
Second reaction leads to rather high concentration of hydrogen in oxide layer.
For instance, it is shown \cite{angermann2016} that the appearance of the first monolayer of
silicon oxide causes a strong increase in both interface states density $D_{it}$ and surface charge.
The further native oxide growth is characterized by significant decrease of $D_{it}$ (down to about $10^{12}$~cm$^{-2}$ in 1~nm layer).
On the other hand, it is demonstrated \cite{Fleetwood} a rough one--to--one correspondence between interface trap density and $P_b$--center density.
Therefore significant decrease of $D_{it}$ indicates about passivation of broken interfacial $\equiv\!\mathrm{Si}\!-\!\mathrm{H}$ bonds by hydrogen.
In addition, according to Pershenkov \emph{et al.} \cite{PersenkovBook},  the SiO$_2$ layers, which are grown by non--chemical oxidation, are rich in an atomic hydrogen.
Thus hydrogen is enough to cause drastic influence on processes, occurring in interface region.

According to Pintilie \emph{et al}. \cite{FZSi:Rad}, the electrical properties of bulk gamma--irradiated Si are mainly influenced by interstitial defect $I^{0/-}$.
In our case, the electrical properties of Si bulk have an effect on series resistance value predominantly.
And increase in $R_s$ value was detected after gamma-irradiation.






\noindent
\textcolor[rgb]{0.00,0.50,1.00}{\textbf{Comment~3.}}
\emph{ Some important References are not cited, such as, e.g., A Davletova, et. al. J. Phys. Chem. Solids 70 (6), 989-992; J. Phys. D: Appl. Phys. 41 (16), 165107.}

\noindent
\textcolor[rgb]{0.51,0.00,0.00}{\textbf{Reply:}}
The list of references was expanded.


\bibliography{olikh}

\end{document}

