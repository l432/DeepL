


\documentclass[preprint]{elsarticle}

\begin{document}


To:
Current Applied Physics Editorial Board


Subject:
Article Submit

\vspace{5mm}
Dear Editors,

\vspace{3mm}
%Enclosed is a manuscript entitled “XXX XXXX XXXXX” by sb, which we are submitting for publication in the journal of …. We have chosen this journal because it deals with …
Enclosed with this letter you will find the electronic submission of manuscript entitled
``Modeling the impact of iron defect variability on silicon solar cell performance across different scenarios''
by Oleg Olikh and Oleksii Zavhorodnii.

It is widely recognized that defects can significantly alter the properties of semiconductors,
and understanding the mechanisms behind these alterations is essential for optimizing solar cell efficiency.
Over the years, researchers have accumulated considerable knowledge about various defects.
Our investigation aims to examine the impact of iron-related defect variability
on the performance of silicon-based solar cells, integrating both modeling and experimental validation.

Our research reveals that changes in short-circuit current provide the most reliable method for estimating iron impurity concentrations,
significantly improving over traditional techniques.
We also found that the behavior of open-circuit voltage is non-monotonic at low boron doping levels,
indicating a complex interaction between doping and defect dynamics.
We believe that our research not only enhances the understanding of iron-related defects
but also has practical implications for developing methods to characterize impurity concentration in 
silicon solar cells from current-voltage characteristic.
Such approach involves quantifying photovoltaic energy conversion parameters 
(short-circuit current, open-circuit voltage, solar cell efficiency, fill factor) 
and then employing these values as input features for machine learning models. 
One of the finding of this research is the expedience of principal component analysis in this context.
Given the growing interest in renewable energy technologies and machine learning,
we are confident that our findings will resonate with your readership and stimulate further exploration in this area.


This is an original paper which has not been simultaneously submitted as a whole or in part anywhere else.
No elements of the work have been published in any form.
No conflict of interest exits in the submission of this manuscript.


We would  very much appreciate if you would consider the manuscript for publication in the \emph{Current Applied Physics}.
%We appreciate your consideration of our manuscript, and we look forward to receiving comments from the reviewers.

%Possible reviewers are the following.
%
%\begin{itemize}
%  \item Yimin Zhang,
%Shenyang University of Chemical Technology,
%Equipment Reliability Institute, Shenyang 110142, China,
%ymzhang@mail.neu.edu.cn
%  \item Kang Li,
%University of Leeds,
%School of Electronic and Electrical Engineering,  Leeds, LS2 9JT, UK,
%k.li1@leeds.ac.uk
%  \item Rebecca Saive,
%University of Twente Institute for Nanotechnology, Enschede 7522 NB, The Netherlands,
%r.saive@utwente.nl
%  \item Ripon Chakrabortty,
%University of New South Wales,
%School of Engineering and IT, Canberra, Australia,
%r.chakrabortty@adfa.edu.au
%%  \item Belen Arredondo,
%%Universidad Rey Juan Carlos
%%Área de Tecnología Electrónica, C/ Tulipán s/n, 28933 Móstoles, Spain,
%%belen.arredondo@urjc.es
%\end{itemize}


\vspace{3mm}

Sincerely yours,

Oleg~Olikh and Oleksii Zavhorodnii


Taras Shevchenko National University of Kyiv


Kyiv 01601, Ukraine

E-mail: olegolikh@knu.ua





\end{document}

