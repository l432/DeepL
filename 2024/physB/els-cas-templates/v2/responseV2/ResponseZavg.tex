

%\documentclass[aip,reprint]{revtex4-1}
%\documentclass[aip,jap,preprint]{revtex4-1}
\documentclass[a4paper,fleqn]{cas-sc}

\usepackage[numbers]{natbib}


\begin{document}
\shorttitle{}


Dear Editor,

We like to express our appreciation to the reviewers for their comments.
We are resubmitting the revised version of the paper number MSB-S-24-02710.
We have studied the comments of the reviewer carefully, and have changed the text according to the comments they
have listed.
%The location of revisions is pointed by blue color in ``MarkedManuscript.pdf''.
Below we refer to each of the reviewer’s comments.


\subsection*{Response to Reviewer \#1 }

\noindent
\textcolor[rgb]{0.00,0.50,1.00}{\textbf{Comment~1.}}
\emph{The author must explain the specific mechanisms you propose for the interaction between iron-related defects and other impurities in silicon solar cells? How might these interactions complicate the interpretation of photovoltaic performance metrics?}

\noindent
\textcolor[rgb]{0.51,0.00,0.00}{\textbf{Reply:}}
In our study, we primarily focus on the impact of iron-related defects, particularly $\mathrm{Fe}_i$ and $\mathrm{Fe}_i\mathrm{B}_s$ pairs, on the photovoltaic parameters of silicon solar cells. However, we acknowledge that the interaction of iron with other impurities (such as oxygen, carbon, or transition metals) can introduce additional complexities in interpreting photovoltaic performance.

Mechanisms of Interaction:
\begin{itemize}
    \item \textbf{iron-oxygen complexes}: oxygen is a common impurity in Czochralski-grown silicon and can form stable complexes with interstitial iron ($\mathrm{Fe}_i$-O). These complexes alter the recombination activity of $\mathrm{Fe}_i$ and may reduce the effectiveness of $\mathrm{Fe}_i\mathrm{B}_s$ pair dissociation as a means to estimate iron concentration \cite{istratov1999};
    \item \textbf{iron-carbon interactions}: carbon-related defects, such as Fe-C pairs, may modify the charge state of $\mathrm{Fe}_i$, influencing its recombination properties and affecting the carrier lifetime \cite{kustov1992}. This interaction can lead to deviations in the expected trends of $I_\mathrm{SC}$ and $V_\mathrm{OC}$;
    \item \textbf{transition metal contamination}: if other transition metals (Cu , Ni, etc.) \cite{coletti2011} are present, they can introduce additional deep-level traps, competing with iron-related recombination centers. This may lead to non-monotonic behavior in efficiency trends, complicating the interpretation of photovoltaic responses .
\end{itemize}


Impact on Photovoltaic Metrics:
\begin{itemize}
    \item the presence of multiple recombination centers can lead to overlapping effects in $V_\mathrm{OC}$ and $I_\mathrm{SC}$, making it difficult to isolate the influence of Fe-related defects;
    \item some impurity interactions may cause non-monotonic trends in efficiency and fill factor changes, particularly under varying injection conditions;
    \item the temperature dependence of defect interactions may further introduce variability in photovoltaic responses, as different impurities have different activation energies.
\end{itemize}


$\mathrm{Fe}_i\mathrm{B}_s$ pairs introduce deep-level recombination centers that strongly impact minority carrier lifetime, leading to reduced $I_\mathrm{SC}$ and $V_\mathrm{OC}$. When $\mathrm{Fe}_i\mathrm{B}_s$ pairs dissociate, the more mobile $\mathrm{Fe}_i$ leads to different recombination dynamics, which is particularly evident under monochromatic (940~nm) illumination. Our study demonstrates that changes in $I_\mathrm{SC}$ after $\mathrm{Fe}_i\mathrm{B}_s$ dissociation can serve as a reliable method to estimate Fe concentration. 
%The aim of the research was added in the introduction part 
%(page~2, paragraph~3 (last paragraph in the Introduction), line~1-2).

\vspace{1cm}
\noindent
\textcolor[rgb]{0.00,0.50,1.00}{\textbf{Comment~2.}}
\emph{ The author should give the transient nature of defect dynamics, how do you assess the long-term stability of silicon solar cells in the presence of iron contamination? What experimental approaches would you recommend to study the aging effects of these defects over time?}

\noindent
\textcolor[rgb]{0.51,0.00,0.00}{\textbf{Reply:}}

\textbf{Transient nature of iron defects and long-term stability}

The reviewer is correct in noting that iron impurities in silicon do not exhibit long-term stability in the traditional sense. Instead, their behavior is highly transient and reversible, depending on illumination and temperature conditions:

\begin{itemize}
    \item \textbf{under Illumination}: FeB pairs dissociate into interstitial iron (Fe) and substitutional boron (B). This process is almost instantaneous under strong illumination;
    \item \textbf{in the dark}: $\mathrm{Fe}_i$ and $\mathrm{B}_s$ recombine to form FeB pairs again. The rate of this re-pairing depends on the temperature of the solar cell, with higher temperatures accelerating the process. Full recovery of all FeB pairs may take minutes to hours, depending on the temperature and doping concentration.
\end{itemize}

This reversible behavior means that iron impurities do not cause permanent degradation in the same way as other defects (e.g., light-induced degradation due to boron-oxygen complexes). Instead, their impact is dynamic and can be partially mitigated by controlling the operating conditions of the solar cell.


\vspace{1cm}
\noindent
\textcolor[rgb]{0.00,0.50,1.00}{\textbf{Comment~3.}}
\emph{An Author should provide a theoretical framework for understanding how the temperature range you studied (290 K to 340 K) influences the activation energy of iron-related defects? How might this affect the performance of solar cells in varying environmental conditions?}

\noindent
\textcolor[rgb]{0.51,0.00,0.00}{\textbf{Reply:}}

The temperature range studied in our work (290 K to 340 K) was chosen to reflect realistic operating conditions of silicon solar cells, particularly those exposed to moderate climate variations. The behavior of iron-boron ($\mathrm{Fe}_i\mathrm{B}_s$) pairs and interstitial iron ($\mathrm{Fe}_i$) is strongly temperature-dependent due to thermal excitation and defect migration effects. The key theoretical considerations include:

\begin{itemize}
    \item Thermal dissociation of $\mathrm{Fe}_i\mathrm{B}_s$ pairs:
    \begin{itemize}
        \item the equilibrium between $\mathrm{Fe}_i\mathrm{B}_s$ pairs and free $\mathrm{Fe}_i$ follows an Arrhenius-type relation, governed by the dissociation rate:
        \begin{equation}
        \label{eq1}
            k_\mathrm{d}  = k_\mathrm{0} \exp(-\frac{E_\mathrm{a}}{k_\mathrm{b}T}),
        \end{equation}
        where $k_\mathrm{d}$ is the dissociation rate, $E_\mathrm{a}$ is the activation energy of $\mathrm{Fe}_i\mathrm{B}_s$ dissociation, $k_\mathrm{b}$ is the Boltzmann constant, and $T$ is the absolute temperature;
        \item prior studies estimate Ea $\approx$ 0.7 $-$ 0.8 eV for $\mathrm{Fe}_i\mathrm{B}_s$ dissociation in silicon;
        \item as temperature increases, $\mathrm{Fe}_i\mathrm{B}_s$ pairs dissociate more readily, increasing $\mathrm{Fe}_i$ concentration, which enhances recombination activity and reduces carrier lifetimes.
    \end{itemize}
    \item Shockley-Read-Hall recombination dependence:
    \begin{itemize}
        \item the capture cross-section ($\sigma$) of $\mathrm{Fe}_i$ defects is temperature-dependent, influencing the recombination rate $U$ given by:
        \begin{equation}
        \label{eq2}
            U  = \frac{\sigma v_\mathrm{th} N_\mathrm{d}}{1 + \exp(\frac{E_\mathrm{t} - E_\mathrm{F}}{k_\mathrm{b}T})},
        \end{equation}
        where $v_\mathrm{th}$ is the thermal velocity of carriers, $N_\mathrm{d}$ is defect concentration, and $E_\mathrm{t}$ is the defect energy level;
        \item as temperature rises, carrier thermal velocity increases, altering the recombination dynamics and further impacting photovoltaic parameters.
    \end{itemize}
\end{itemize}

The temperature-dependent behavior of iron-related defects has direct consequences on solar cell operation under different environmental conditions:
\begin{itemize}
    \item Reduction in open-circuit voltage ($V_\mathrm{OC}$):
    \begin{itemize}
        \item increased FeB dissociation at high temperatures raises $\mathrm{Fe}_i$ concentration, enhancing Shockley-Read-Hall recombination. Since $V_\mathrm{OC}$ is governed by the equation:
        \begin{equation}
        \label{eq3}
             V_\mathrm{OC} = \frac{k_\mathrm{b}T}{q} log\left(\frac{I_\mathrm{SC}}{I_\mathrm{0}} + 1\right),
        \end{equation}
        where $I_\mathrm{0}$ is the saturation current, higher recombination rates increase $I_\mathrm{0}$, leading to a reduction in $V_\mathrm{OC}$;
        \item the observed temperature coefficient of ($V_\mathrm{OC} \approx -2.3~\frac{mV}{K}$) aligns with known iron-related degradation effects.
    \end{itemize}
\end{itemize}


\textbf{Impact on defect populations}:

\begin{itemize}
    \item \textbf{FeB dissociation}: at elevated temperatures, increased  $\mathrm{Fe}_i$ concentration enhances Shockley-Read-Hall recombination, reducing minority carrier lifetime ($\tau$) and degrading $I_\mathrm{SC}$, $V_\mathrm{OC}$, and efficiency ($\eta$).;
    \item \textbf{re-pairing dynamics}: In the dark, $\mathrm{Fe}_i$ and $\mathrm{B}_s$ recombine to form FeB pairs, but the rate depends on T. This reversibility explains the transient nature of iron-related efficiency losses.
\end{itemize}

At very high temperatures (>340 K), the dissociation of FeB pairs may reach a saturation point, where further increases in temperature have a diminishing effect on the dissociation rate. This is due to the limited availability of FeB pairs to dissociate at extreme temperatures.


In our study, we used well-established temperature dependencies for silicon parameters, such as bandgap narrowing, carrier mobilities, and intrinsic recombination coefficients, as reported in line 62 on page 2 and lines 39-48 on page 3. These dependencies ensure the accuracy of our simulations across the studied temperature range.

\vspace{1cm}
\noindent
\textcolor[rgb]{0.00,0.50,1.00}{\textbf{Comment~4.}}
\emph{An Author mentioned the application of principal component analysis in evaluating impurity levels. Could you elaborate on how this technique was implemented in your study and its effectiveness in distinguishing between different types of impurities?}

\noindent
\textcolor[rgb]{0.51,0.00,0.00}{\textbf{Reply:}}

PCA was applied to reduce redundancy in the dataset of photovoltaic parameters influenced by iron-related defect restructuring. Here is how it was implemented:

\begin{itemize}
    \item \textbf{Input parameters}: the dataset included relative changes in four key photovoltaic parameters due to FeB pair dissociation: $\epsilon I_\mathrm{SC}$, $\epsilon V_\mathrm{OC}$, $\epsilon \eta$, $\epsilon FF$. Additional variables included solar cell properties: $T$, $N_\mathrm{B}$, $d_\mathrm{p}$;
    \item \textbf{Standardization}: all features were standardized to zero mean and unit variance to ensure equal weighting;
    \item \textbf{Covariance matrix and eigenanalysis}: PCA transformed the standardized data into orthogonal principal components (PCs) by computing eigenvalues/eigenvectors of the covariance matrix;
    \item \textbf{Component selection}: Components were ranked by explained variance. For example, under monochromatic illumination, the first three PCs captured >95~\% of the variance (Table 2 in the manuscript);
    \item \textbf{Interpretation}: 
    \begin{itemize}
    \item \textbf{PC1}: dominantly correlated with $\epsilon I_\mathrm{SC}$ and $\epsilon \eta$, reflecting iron-induced recombination;
    \item \textbf{PC2}: linked to $\epsilon V_\mathrm{OC}$ and temperature dependencies.
    \end{itemize}
\end{itemize}

Our study specifically addressed iron impurities (FeB pairs and interstitial $\mathrm{Fe}_i$), and PCA was used to:

\begin{itemize}
    \item identify the most informative parameters for estimating $N_\mathrm{Fe}$;
    \item reduce redundancy between correlated variables (e.g., $\epsilon I_\mathrm{SC}$ and $\epsilon \eta$ share similar trends).
\end{itemize}

Although PCA effectively streamlined the dataset for iron analysis, distinguishing between different impurities (e.g., Fe vs. Cu/Ni) was not within the scope of this work. However, the utility of PCA in multi-impurity systems can be inferred: if multiple impurities are present, PCA could isolate distinct variance patterns tied to specific defects (e.g., Fe-related recombination vs. Cu-induced traps). This would require including impurity-specific parameters (e.g. carrier lifetime and photoluminescence spectra) and advanced feature engineering.


\vspace{1cm}
\noindent
\textcolor[rgb]{0.00,0.50,1.00}{\textbf{Comment~5.}}
\emph{Your paper discusses the influence of doping levels on the response of solar cells to iron contamination. Author should explain how varying the doping concentration affects the sensitivity of photovoltaic parameters to iron presence through the reference:
Augmented photovoltaic performance of Cu/Ce-(Sn: Cd)/n-Si Schottky barrier diode utilizing dual-doped Ce-(Sn: Cd) thin films.}

\noindent
\textcolor[rgb]{0.51,0.00,0.00}{\textbf{Reply:}}

The doping level ($N_\mathrm{B}$) in the silicon base directly influences the position of the Fermi level and the dynamics of carrier recombination, modulating the sensitivity of the solar cell to iron contamination. Key mechanisms include:

a) Fermi level and defect activity

\begin{itemize}
    \item low doping ($N_\mathrm{B}$ $\sim$ $10^{15}$ cm$^{-3}$): the Fermi level ($E_\mathrm{F}$) lies closer to the valence band (Fig. 4 in the manuscript). FeB pairs dominate, as interstitial iron remains passivated. Dissociating FeB pairs (e.g., under illumination) releases $\mathrm{Fe}_i$, which introduces a deep donor level near mid-gap. However, at low $N_\mathrm{B}$,  $\mathrm{Fe}_i$ is in a neutral charge state ($\mathrm{Fe}_i^0$), showing weaker recombination activity. As a result: a small increase in recombination causes a positive  $\epsilon I_\mathrm{SC}$ (higher $I_\mathrm{SC}$ post-dissociation due to reduced trapping);
    \item high doping ($N_\mathrm{B}$ $\sim$ $10^{17}$ cm$^{-3}$): $E_\mathrm{F}$ shifts toward the conduction band (Fig. 4). $\mathrm{Fe}_i$ adopts a positive charge state ($\mathrm{Fe}_i^+$), becoming a strong SRH recombination center. Dissociating FeB pairs significantly increases $\mathrm{Fe}_i^+$, drastically reducing the lifetime of minority carriers. As a result: negative $\epsilon I_\mathrm{SC}$ and $\epsilon \eta$ due to enhanced recombination.
\end{itemize}

b) Electric field and carrier collection

Higher $N_\mathrm{B}$ narrows the depletion region, reducing the electric field strength. Carriers generated outside the depletion region (quasi-neutral base) are more susceptible to recombination at $\mathrm{Fe}_i$ defects. This amplifies the sensitivity of $I_\mathrm{SC}$ and $V_\mathrm{OC}$ to iron contamination at high $N_\mathrm{B}$.

\textbf{Connection to the referenced study}:

The referenced work on Cu/Ce-(Sn: Cd)/n-Si Schottky diodes demonstrates how dual doping (Sn and Cd in Ce thin films) modifies defect states and band alignment to enhance photovoltaic performance. This aligns with our findings:

1) dual doping suppresses recombination by tailoring defect energy levels and carrier transport pathways. In our study, adjusting $N_\mathrm{B}$ similarly modulates defect activity (FeB vs. $\mathrm{Fe}_i$) to control recombination.
2) our paper and this paper reveal that doping strategies can lead to non-linear responses in photovoltaic parameters. For example: Sn/Cd co-doping optimizes band alignment, minimizing interfacial recombination. In our work,  $N_\mathrm{B}$ $\sim$ $10^{16}$ cm$^{-3}$ represents a "blind spot" where $\epsilon I_\mathrm{SC}$ sensitivity to iron is minimal (Fig. 3).

\vspace{1cm}
\noindent
\textcolor[rgb]{0.00,0.50,1.00}{\textbf{Comment~6.}}
\emph{In your findings, Author mentions that changes in short-circuit current under monochromatic illumination are the most reliable for estimating iron concentration. An author should provide more details on the methodology used to derive this conclusion and any potential limitations of this approach?}

\noindent
\textcolor[rgb]{0.51,0.00,0.00}{\textbf{Reply:}}

Key metrics for reliability:

\begin{itemize}
    \item \textbf{Monotonicity}: $\epsilon I_\mathrm{SC}$ under monochromatic light exhibited a linear, monotonic relationship with $N_\mathrm{Fe}$ across all $N_\mathrm{B}$ and $T$, unlike $V_\mathrm{OC}$ or $FF$, which showed non-monotonic trends;
    \item \textbf{Sensitivity}: at $N_\mathrm{B}$ = $10^{17}$ cm$^{-3}$, $\epsilon I_\mathrm{SC}$ under 940~nm changed by >100~\% for $N_\mathrm{Fe}$ = $10^{14}$ cm$^{-3}$, vs. <30~\% under AM1.5;
    \item \textbf{Correlation strength}: Pearson correlation coefficient ?
\end{itemize}
 
Validation:

\begin{itemize}
    \item \textbf{Experimental agreement}: measured $\epsilon I_\mathrm{SC}$ (940 nm) matched simulations after applying a correction factor $C_\mathrm{cor}$ = 1.4 to account for unmodeled effects (e.g. shunt resistance, surface passivation).
\end{itemize}

Limitations of the approach:

1) Real solar cells operate under AM1.5, so monochromatic-based $N_\mathrm{Fe}$ estimates can not predict performance under standard conditions.


2) Co-existing impurities (e.g. Cu, Ni) could introduce overlapping recombination signals, though this study focused exclusively on iron.


\vspace{1cm}
\noindent
\textcolor[rgb]{0.00,0.50,1.00}{\textbf{Comment~7.}}
\emph{An Author must improve the introduction section in the application part through the recent referencs
CuO-La2O3 Composite-Enabled MIS Schottky Barrier Diodes: A Novel Approach to Optoelectronic Device Diversification; Enhancing photovoltaic applications through precipitating agents in ITO/CIS/CeO2/Al heterojunction solar cell; Manifestation on the choice of a suitable combination of MIS for proficient Schottky diodes for optoelectronics applications: A comprehensive review.}

\noindent
\textcolor[rgb]{0.51,0.00,0.00}{\textbf{Reply:}}



Recent developments in MIS and heterojunction technologies have opened new pathways for optoelectronic device diversification and enhanced photovoltaic performance. For example, a novel approach employing a CuO–La2O3 composite has been shown to enable MIS Schottky barrier diodes with improved rectification and tailored optoelectronic responses. In this work, Paul et al. demonstrated that the incorporation of La2O3 into CuO can optimize the interfacial properties and barrier heights, thereby enhancing device performance.
% https://doi.org/10.1007/s10904-024-03277-z

Moreover, precipitating agents have recently been applied in heterojunction solar cells to improve the interfacial quality and charge separation in ITO/CIS/CeO2/Al structures. Gayathri et al. reported that the use of specific precipitating agents not only refines the microstructure but also improves the overall photovoltaic efficiency by facilitating better charge transport.  
% https://doi.org/10.1016/j.inoche.2024.112866

In addition, a comprehensive review by Sibu et al. has recently detailed the critical role of selecting a suitable MIS combination for the fabrication of proficient Schottky diodes for optoelectronic applications. This review underscores how the careful engineering of the insulator and metal interfaces can lead to significant improvements in device stability and performance, providing a robust framework for future photovoltaic and sensor applications 
% https://doi.org/10.1016/j.nanoen.2024.109534



\subsection*{Response to Reviewer \#2 }

\noindent
\textcolor[rgb]{0.00,0.50,1.00}{\textbf{Comment~1.}}
\emph{The abstract section should be more informative.}

\noindent
\textcolor[rgb]{0.51,0.00,0.00}{\textbf{Reply:}}

New:

Silicon solar cells are critical for renewable energy but are often degraded by iron impurities that alter defect states and carrier recombination. This study investigates the impact of iron-related defect variability on photovoltaic performance via comprehensive SCAPS simulations and experimental measurements. Simulations span temperatures from 290 to 340~K, base thicknesses from 180 to 380~$\mu$m, boron doping levels from $10^{15}$ to $10^{17}$~cm$^{-3}$, and iron concentrations from $10^{10}$ to $10^{14}$~cm$^{-3}$ under AM1.5 and monochromatic (940~nm) illumination. The dissociation of iron–boron pairs significantly affects short-circuit current, open-circuit voltage, fill factor, and efficiency. Experimental results validate simulation predictions, identifying short-circuit current changes under monochromatic illumination as a sensitive indicator of iron contamination. Principal component analysis reveals strong correlations among photovoltaic parameters, supporting their potential use in non-invasive iron quantification. This integrated approach advances our understanding of impurity effects and paves the way for improved device reliability.


\noindent
\textcolor[rgb]{0.00,0.50,1.00}{\textbf{Comment~2.}}
\emph{The novelty of the work is missing in the introduction. Authors should explain what are the key advantages of iron's impact on silicon solar cell?}

\noindent
\textcolor[rgb]{0.51,0.00,0.00}{\textbf{Reply:}}

We have revised the introduction to clearly articulate the novelty of our work. In our study, we do not simply regard iron contamination as a detrimental effect; rather, we demonstrate that the changes induced by iron‐related defects—particularly the dissociation of iron–boron (FeB) pairs—can be leveraged as a diagnostic tool. The key advantages and novel aspects are as follows:


1) By systematically correlating photovoltaic parameters (e.g. short-circuit current, open-circuit voltage) with iron concentration under controlled conditions, our approach provides a direct, non-destructive method for quantifying iron contamination in silicon solar cells. This diagnostic strategy can facilitate real-time quality control in manufacturing.

2) The sensitivity of the photovoltaic parameters—especially under monochromatic illumination—to the state of iron-related defects lays the groundwork for integrating machine learning algorithms. This enables advanced, automated impurity evaluation and process optimization.

3) Our work spans a broad range of operational conditions (temperature, doping levels, base thickness, and illumination types). This extensive mapping not only deepens the understanding of defect physics in silicon but also makes the findings robust and widely applicable in industrial settings. Most of the previous studies that investigate iron contamination in silicon solar cells have focused on multicrystalline silicon, where inherent processing and grain boundary effects typically lead to iron concentrations above $10^{13}$ cm$^{-3}$ \cite{abbott2014}.


\noindent
\textcolor[rgb]{0.00,0.50,1.00}{\textbf{Comment~3.}}
\emph{Authors should improve the image quality of all figures.}

\noindent
\textcolor[rgb]{0.51,0.00,0.00}{\textbf{Reply:}}

improvement options in the folder "Pictures"


\noindent
\textcolor[rgb]{0.00,0.50,1.00}{\textbf{Comment~4.}}
\emph{Author should explain how does the band alignment affect the overall performance of the solar cell?}

\noindent
\textcolor[rgb]{0.51,0.00,0.00}{\textbf{Reply:}}

1) The relative positions of the conduction and valence bands across the junction establish the built-in potential. An optimal band alignment ensures that the built-in field is sufficiently strong to separate photogenerated electron–hole pairs effectively. For example, a well‑engineered conduction band offset can facilitate the extraction of electrons while blocking holes, reducing recombination losses.

2) The conduction band offset ($\Delta E_\mathrm{C}$) and valence band offset $\Delta E_\mathrm{V}$) determine the energy barriers for minority and majority carrier injection. A too-large offset may impede carrier transport (lowering short-circuit current) or increase the dark current, adversely affecting the open-circuit voltage and fill factor. Conversely, if the offset is too small, carriers may not be efficiently confined, leading to increased recombination at the interface.

3) The band alignment affects the density of interface states and the rate at which carriers recombine at the junction. A favorable alignment minimizes the formation of recombination centers, thereby enhancing the quasi-Fermi level splitting under illumination and ultimately improving cell efficiency.


\noindent
\textcolor[rgb]{0.00,0.50,1.00}{\textbf{Comment~5.}}
\emph{What are the primary sources of iron contamination in silicon used for solar cells?}

\noindent
\textcolor[rgb]{0.51,0.00,0.00}{\textbf{Reply:}}

In our work—where we use high‑purity, boron‑doped Czochralski silicon (100), which is monocrystalline—the primary sources of iron contamination are predominantly extrinsic. Unlike multicrystalline silicon, where iron can be inherent in the feedstock and introduced via diamond wire sawing, monocrystalline silicon is grown in highly controlled environments. Nonetheless, trace iron can still be introduced during processing. The main sources include:

1) Even in well‐controlled Czochralski processes, slight contamination can occur from the crucible, susceptor, or ambient atmosphere during growth. However, this is generally minimized in high‑purity monocrystalline ingots \cite{Goncalo2017}. 

2) Iron may be introduced during subsequent wafer processing steps—such as cleaning, dopant diffusion, and high‑temperature annealing—via contaminated process chemicals, deposition chambers, or etching tools. For example, residual metal particles on equipment surfaces or impurities in the chemical baths can transfer trace amounts of iron onto the wafer surface \cite{abbott2014}.

3) During wafer handling and transportation, inadvertent contact with metal fixtures or contaminated tools can also introduce iron.


\noindent
\textcolor[rgb]{0.00,0.50,1.00}{\textbf{Comment~6.}}
\emph{What role do recombination centers created by iron play in the modeling of solar cell performance?}

\noindent
\textcolor[rgb]{0.51,0.00,0.00}{\textbf{Reply:}}

Recombination centers created by iron—such as interstitial iron ($\mathrm{Fe}_i$) and iron–boron (FeB) pairs—play a critical role in determining the performance of silicon solar cells. In our modeling, these defect centers are incorporated using the Shockley–Read–Hall formalism. Their presence is characterized by specific parameters such as defect energy level, electron and hole capture cross sections, and defect concentration. These parameters directly affect the minority carrier lifetime, increasing the recombination current and dark saturation current, which in turn reduce the open‑circuit voltage ($V_\mathrm{OC}$), fill factor ($FF$), and overall efficiency ($\eta$).


\noindent
\textcolor[rgb]{0.00,0.50,1.00}{\textbf{Comment~7.}}
\emph{Author should discuss and cited recent Si based solar cell in the revised manuscript: $DOI:10.1016/j.mseb.2024.117360, DOI:10.1016/j.mseb.2023.117141, DOI:10.1016/j.mseb.2024.117817, DOI:10.1007/s42247-024-00821-y, DOI:10.1016/j.inoche.2024.112785$}

\noindent
\textcolor[rgb]{0.51,0.00,0.00}{\textbf{Reply:}}

Recent advances in Si-based solar cell design and processing have provided novel insights into interface engineering, defect control, and alternative device architectures. For example, Sultana et al. [DOI:10.1016/j.mseb.2024.117360] present a novel design and optimization of a Si-based high-performance double absorber heterojunction solar cell. Their work demonstrates that precise control of the absorber layers and interface quality can significantly enhance device efficiency by reducing recombination losses.


In addition, Behera et al. [DOI:10.1016/j.mseb.2023.117141] report the prediction of new MAX phase compounds, specifically Zr2MSiC2 (M = Ti, V), which offer unique structural and electronic properties. These compounds, when integrated with Si, may serve as effective passivation layers or gettering agents to mitigate the detrimental effects of metallic impurities.


% як на мене це зайве DOI:10.1016/j.mseb.2024.117817


A recent study [DOI:10.1007/s42247-024-00821-y] explores innovative device architectures and advanced gettering techniques in Si solar cells. This work emphasizes how interface modifications and impurity mitigation can be combined to boost performance, providing an important reference point for our approach.


Islam et al. [DOI:10.1016/j.inoche.2024.112785] examine novel charge transport materials in optoelectronic devices. While their work primarily addresses perovskite solar cells, the insights regarding band alignment and defect passivation can be directly applied to Si-based solar cells to enhance carrier extraction and reduce recombination.

\noindent
\textcolor[rgb]{0.00,0.50,1.00}{\textbf{Comment~8.}}
\emph{How does the modeling in this study contribute to the design of processes for impurity control in silicon?}

\noindent
\textcolor[rgb]{0.51,0.00,0.00}{\textbf{Reply:}}

The modeling in our study contributes to impurity control process design by providing a predictive framework that relates key process parameters (e.g. temperature, doping concentration, illumination and base thickness) to the behavior of iron‐related defect states in silicon. In our SCAPS simulations, we incorporate defect parameters for interstitial iron and Fe–boron pairs using a Shockley–Read–Hall formalism. By reproducing experimental trends in photovoltaic performance (e.g. open‑circuit voltage, fill factor), our model allows us to predict how varying process conditions—such as gettering anneals or light-soaking steps—affect the dissociation/recombination dynamics of FeB pairs and the overall impurity profile.

This quantitative insight enables process engineers to:

1) Identify the ideal temperatures and annealing durations that maximize gettering efficiency and reduce active impurity levels.


2) Fine-tune phosphorus diffusion gettering and cooling profiles to minimize the residual concentration of recombination-active iron.


3) Use the simulation results as guidelines to rapidly screen and optimize process parameters, thus accelerating the development of impurity control strategies.


\noindent
\textcolor[rgb]{0.00,0.50,1.00}{\textbf{Comment~9.}}
\emph{State the main findings in the conclusions.}

\noindent
\textcolor[rgb]{0.51,0.00,0.00}{\textbf{Reply:}}

already???

\bibliographystyle{model1-num-names}
\bibliography{olikh_Methods}


\end{document}

%Будь-ласка, покращ англійську в наступних реченнях, які я буду пропонувати. Зокрема, треба буде виправити граматичні та стилістичні помилки
%якщо можна, окрім виправленого речення, наводь інформацію про зміни, які зроблені