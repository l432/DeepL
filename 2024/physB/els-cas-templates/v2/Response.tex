

%\documentclass[aip,reprint]{revtex4-1}
%\documentclass[aip,jap,preprint]{revtex4-1}
\documentclass[a4paper,fleqn]{cas-sc}

\usepackage[numbers]{natbib}
\usepackage{mdframed}
\usepackage{color}
\usepackage{enumitem}


\begin{document}
\shorttitle{}


%Dear Editor,
%
%We like to express our appreciation to the reviewers for their comments.
%We are resubmitting the revised version of the paper number MSB-S-24-02710.
%We have studied the comments of the reviewer carefully, and have changed the text according to the comments they
%have listed.
%%The location of revisions is pointed by blue color in ``MarkedManuscript.pdf''.
%Below we refer to each of the reviewer’s comments.


Dear Editor and Reviewers,

We sincerely thank you for taking the time to review our manuscript
``Iron's impact on silicon solar cell execution: comprehensive modeling across diverse scenarios''
(Ms. Ref. No.: MSB-S-24-02710).
Your insightful comments and constructive suggestions have greatly helped us improve
the quality of our work.
We particularly appreciate your careful reading and thoughtful feedback, which have
led to significant improvements in both the technical content and presentation clarity of our manuscript.
We have carefully addressed all the comments and made corresponding revisions to the manuscript.
%All changes are marked in red in the revised version.
%The location of revisions is pointed by blue color in ``MarkedManuscript.pdf''.
Below we provide our detailed point-by-point responses to
each comment.
We hope the revised manuscript better meets your expectations and standards for publication in Materials Science and Engineering: B.


\subsection*{Response to Reviewer \#1 }

\noindent
\textcolor[rgb]{0.00,0.50,1.00}{\textbf{Comment~1.}}
\emph{The author must explain the specific mechanisms you propose for the interaction between iron-related defects and other impurities in silicon solar cells? How might these interactions complicate the interpretation of photovoltaic performance metrics?}


%Автор повинен пояснити конкретні механізми, які ви пропонуєте для взаємодії між дефектами, пов'язаними з залізом, та іншими домішками в кремнієвих сонячних елементах? Як ці взаємодії можуть ускладнити інтерпретацію показників фотоелектричної продуктивності?

\noindent
\textcolor[rgb]{0.51,0.00,0.00}{\textbf{Reply:}}


In general, interstitial iron is already mobile at room temperature and can interact with various impurities.
Over 30 iron-related complexes have been identified in silicon,
and nearly 25 energy levels associated with iron-related defects have been reported in the silicon bandgap across different studies \cite{Istratov1999,Gwozdz2022,JYOTHI2015}.
For example, iron forms complexes with carbon, oxygen, hydrogen, sulfur, and other metal impurities, as well as with native defects in silicon.
However, in terms of their impact on photovoltaic performance, two primary factors must be considered.
First, the concentration of the relevant defects is crucial.
It is widely recognized that in $p$--type silicon under equilibrium conditions, most iron impurity atoms pair with acceptors
(e.g., boron in Si:B) \cite{Kimerling1983,Istratov1999}.
The concentration of other complexes is significantly lower,
and detailed investigation of these complexes requires specialized sample preparation,
such as additional doping with other expected complex components
or irradiation to increase the concentration of native defects \cite{Tang2013}.
Further supporting this, several methods used to evaluate the total iron concentration in $p$--Si rely on breaking the FeB pair \cite{Zoth1990,FeMethod2012,Olikh2021JAP}.
Second, the recombination activity of these centers must be considered,
as it depends on the defect energy level relative to the Fermi level and the charge carrier capture cross-section.
Interstitial iron and FeB pairs are among the most detrimental recombination-active impurities in $p$--Si \cite{Istratov1999,TeimurazJAP} in contrast to other iron-related defects.
Notably, iron gettering by oxygen precipitates or structural defects increases carrier lifetime \cite{Schoen2011,FeB:Vahanissi}.
In $n$--Si, Fe$_i$ is not a recombination-active impurity due to the filling of its energy level.
Reducing the impact of iron through gettering during phosphorus diffusion \cite{Schoen2011,FeB:Vahanissi,Vaehaenissi2017} or the formation of a
passivating layer \cite{Teimuraz2014JAP} is well-established and used in the solar cell industry.

Thus, our assumption that the primary influence of iron impurities in the $p$--regions of a solar cell on photoelectric parameters is due to
Fe$_i$B$_\mathrm{Si}$ pairs and Fe$_i$ is fully justified.
Furthermore, we emphasize that this paper focuses on the relative changes in parameters during
Fe$_i$B$_\mathrm{Si}$ $\rightleftarrows$ Fe$_i$ + B$_\mathrm{Si}$ transformation, which enables the isolation of these specific defects' contributions.

This answer is incorporated in the revised text on page 5, two last paragraph in Section 2.1. Simulation Details.




%In general, interstitial iron is already mobile at room temperature and it can interact with different impurities.
%More than 30 Fe-related complexes were detected in Si and almost 25 energy levels of Fe-related defects were reported in
%the bandgap of Si in different studies.
%Наприклад, залізо утворює комплекси з атомами вуглецю, кисню, водню, сірки, власними дефектами кремнію, іншими металевими домішками.
%Проте з точки зору впливу подібних комплексів на photovoltaic performance необхідно врахувати два основних фактора.
%По-перше, концентрацію відповідних дефектів.
%Загально прийнятою є думка, що в p-Si у рівноважному стані переважна кількість домішкових атомів заліза утворює пари з акцептором (наприклад з бором у Si:B).
%Концентрація інших комплексів набагато менша і для їх детального дослідження необхідно проводити спеціальну підготовку зразків на кшталт додаткового легування іншими компонентами очікуваного комплексу чи радіаційне опромінення для збільшення концентрації власних дефектів.
%Додатковим підтвердженням цього є те, що низка методів, спрямованих на оцінку загальної концентрації заліза у  p-Si спирається на процеси розбиття саме пар FeВ.
%По-друге, треба взяти до уваги рекомбінаційну активність центрів, яка залежить від розташування енергетичного рівня дефекту стосовно рівня Фермі та поперечного перерізу захоплення носіїв заряду.
%Interstitial iron and FeB pair are the one of the most detrimental recombination active impurities in р-silicon на відміну від інших iron-related defects.
%Так, iron gettering by oxygen precipitate or structural defects leads to збільшення часу життя носіїв заряду.
%В n-Si Fei не є рекомбінаційно активною домішкою внаслідок заповнення енергетичного рівня.
%Зменшення the impact of iron внаслідок gettering під час phosphorus diffusion чи створення просвітлюючого шару є добре відомими і використовуються у in solar cell industry.
%
%Таким чином, використане в роботі припущення про те, що основний вплив домішкового заліза в р-областях сонячного елементу на фотоелектричні параметри пов'язаний з FeB pair  and Fei є цілком виправданим.
%Крім того, підкреслимо, що в роботі акцентовано увагу саме на відносні зміни параметрів при перебудові Fe-FeB, що дозволяє додатково виокремити внесок саме цих дефектів.



%The aim of the research was added in the introduction part
%(page~2, paragraph~3 (last paragraph in the Introduction), line~1-2).

\vspace{1cm}
\noindent
\textcolor[rgb]{0.00,0.50,1.00}{\textbf{Comment~2.}}
\emph{ The author should give the transient nature of defect dynamics, how do you assess the long-term stability of silicon solar cells in the presence of iron contamination? What experimental approaches would you recommend to study the aging effects of these defects over time?}

%Автору слід навести перехідний характер динаміки дефектів, як ви оцінюєте довготривалу стабільність кремнієвих сонячних елементів за наявності забруднення залізом? Які експериментальні підходи ви б порекомендували для вивчення ефектів старіння цих дефектів з часом?

\noindent
\textcolor[rgb]{0.51,0.00,0.00}{\textbf{Reply:}}

Interstitial iron atoms exhibit high mobility in silicon \cite{Istratov1999}.
For example, the characteristic time required for FeB pair formation,
which depends on the diffusion coefficient of Fe$_i$, is given by the following expression \cite{FeBAssJAP2014,FeBKin2019,FeBAssSST2011}:
\begin{equation}
\label{eqTass}
\tau=5.7\times10^5\,\frac{\mathrm{s}}{\mathrm{K}\;\mathrm{cm}^3}\times\frac{T}{N_\mathrm{B}}\exp\left(\frac{0.65}{kT}\right)\,.
\end{equation}
where
$N_\mathrm{B}$ is the doping level.

That is, at room temperature, the formation of an equilibrium concentration of FeB pairs is completed within 24 hours,
while at $T=340$~K, the $\tau$ is approximately 10 minutes.
Therefore, iron contamination is not a determining factor in the long--term stability of silicon solar cells, unlike sodium atoms, which are considered one
of the primary contributors to potential-induced degradation \cite{Yamaguchi2021}.


The information on the characteristic time of FeB pair formation has been added to the revised manuscript (page 5, last paragraph in Section 2.2.).

\begin{mdframed}
The measurements were carried over the temperature range of 300-340~K.
The sample temperature was driven by a thermoelectric cooler controlled by an STS-21 sensor
and maintained constant by a PID algorithm embedded in the software that serves the experimental setup.
\textcolor[rgb]{1.00,0.07,0.00}{Notably, the characteristic time for FeB pair recovery after the cessation of illumination is approximately 13,000 seconds at 300 K
and only 600 seconds at 340 K \cite{FeBAssJAP2014,FeBKin2019}.
These limitations affect the measurement duration at high temperatures, and the experiments accounted for them.}
\end{mdframed}


%The relevant information can be found on page 3, paragraphs 1-2.


\vspace{1cm}
\noindent
\textcolor[rgb]{0.00,0.50,1.00}{\textbf{Comment~3.}}
\emph{An Author should provide a theoretical framework for understanding how the temperature range you studied (290 K to 340 K) influences the activation
energy of iron-related defects? How might this affect the performance of solar cells in varying environmental conditions?}

%Автор повинен надати теоретичну основу для розуміння того, як досліджуваний вами діапазон температур
%(від 290 K до 340 K) впливає на енергію активації дефектів, пов'язаних із залізом?
%Як це може вплинути на продуктивність сонячних елементів в різних умовах навколишнього середовища?




\noindent
\textcolor[rgb]{0.51,0.00,0.00}{\textbf{Reply:}}

Initially, we note that the selected temperature range reflects the
realistic operating conditions of most silicon solar cells.
In our calculations, we accounted for the temperature dependencies of both silicon properties ---
including bandgap, the effective density of states; effective masses, thermal velocities and mobilities of charge carriers;
Auger and band-to-band recombination coefficients, and light absorption values ---
and defect parameters, such as energy level positions within the bandgap and electron and hole capture cross-sections.
Additionally, we considered the temperature dependence of the equilibrium concentration ratio of Fe$_i$ and FeB.
Specifically, the primary effect of increasing temperature on iron-related defects is the rise in the relative equilibrium
concentration of interstitial iron atoms
and the change in the charge carrier capture cross-section of each defect.

Accounting for all the above factors in the modeling process suggests that
the results presented in this work accurately reflect the impact of both defects
and defect parameter variation on solar cell performance under varying environmental conditions,
including temperature,  illumination type, and intensity.

Details on how temperature dependencies were accounted for in the simulation can be found on page 3, last two paragraphs


\begin{mdframed}
As can be seen from Table~1, calculations spanned a broad range of temperatures and base doping levels.
Therefore, to improve the accuracy of the calculations when inputting the initial parameters into SCAPS,
\textcolor[rgb]{1.00,0.07,0.00}{temperature and concentration (where applicable) dependencies of the following silicon parameters were taken into account}:

\begin{itemize}[itemsep=2pt, parsep=0pt, topsep=0pt]
    \item bandgap according to Passler \cite{Passler2002};
    \item doping induced bandgap narrowing according to Yan \& Cuevas \cite{EgNarrow};
    \item effective density of states at conduction and valence band and intrinsic carrier concentration according to Couderc et al. \cite{Si_ni_Couderc};
    \item thermal carrier velocities according to Green \cite{Nc:Green};
    \item free carrier effective masses according to O’Mara et al. \cite{OMara};
    \item carrier mobilities according to Klaassen's theory \cite{KLAASSEN953};
\end{itemize}

The values of surface recombination coefficients were considered equal to the thermal velocities of carriers \cite{Fell2015}.
The calculations addressed recombination processes within the structural volume,
incorporating both intrinsic recombination and Shockley-Read-Hall (SRH) recombination at iron-related defects.
In the first case, processes of band-to-band radiation recombination were considered
(where the calculation of the corresponding coefficient included the fraction of radiatively emitted photons
reabsorbed via band-to-band processes according to Niewelt et al.\cite{Brad2022})
and Auger recombination (where the coefficients considered the effect of Coulomb enhancement \cite{AugerSi2022} and
\textcolor[rgb]{1.00,0.07,0.00}{temperature dependence} \cite{Si_Auger}).
\end{mdframed}


and page~4, 3rd  and 4th paragraph from the top

\begin{mdframed}
Case~2.
Iron atoms predominantly form pairs with acceptors, $\left[\mathrm{Fe}_i\mathrm{B}_s\right] \gg \left[\mathrm{Fe}_i\right]$,
\textcolor[rgb]{1.00,0.07,0.00}{but the exact concentration ratio depends on the position of the Fermi level and temperature} \cite{FeB:kinetic,MurphyJAP2011}
and varies from point to point within the solar cell.
Further details about calculation of the concentration profiles of Fe$_i$B$_s$ and Fe$_i$ are provided in \cite{Olikh2022PPV,Olikh2019SM}.
This case corresponds to prolonged storage of the structure in darkness or under conditions of low-intensity ($< 0.01$~J~cm$^{-2}$ \cite{Macdonald2004}) illumination.

During the calculations, it was assumed that Fe$_i$ forms a single donor level,
while the Fe$_i$B$_s$ pair has a trigonal configuration and acts as an amphoteric defect.
\textcolor[rgb]{1.00,0.07,0.00}{We obtained defect parameters, including energy level positions within the bandgap,
electron and hole capture cross-sections, and their temperature dependencies, based on data from relevant studies} \cite{ROUGIEUX2018,Istratov1999,Paudyal}.
\end{mdframed}





%The relevant information can be found on page 3, paragraphs 1-2.


\vspace{1cm}
\noindent
\textcolor[rgb]{0.00,0.50,1.00}{\textbf{Comment~4.}}
\emph{An Author mentioned the application of principal component analysis in evaluating impurity levels.
Could you elaborate on how this technique was implemented in your study and its effectiveness in distinguishing between different types of impurities?}

%Автор згадував про застосування аналізу головних компонент для оцінки рівнів домішок.
%Чи не могли б ви докладніше розповісти про те, як ця методика була застосована у вашому дослідженні та її ефективність у розрізненні різних типів домішок?

\noindent
\textcolor[rgb]{0.51,0.00,0.00}{\textbf{Reply:}}

One of the objectives of this study was to evaluate the feasibility of using variations 
in short-circuit current, open-circuit voltage, solar cell efficiency, or fill factor, 
which results from the decay of iron-containing pairs, to estimate the concentration of iron impurities.
Our simulations revealed that the relative changes in these parameters depend 
not only on the iron concentration but also on temperature and 
other solar cell characteristics, such as doping level and base thickness.
In other words, when determining the magnitude of $N_\mathrm{Fe}$ based on changes in short-circuit current, 
a set of parameters (descriptors) must be used: ($T$, $d_p$, $N_\mathrm{B}$, $\varepsilon I_\mathrm{SC}$).
Evidently, incorporating additional information, such as changes in efficiency ($\varepsilon \eta$), along with an expanded
set of descriptors ($T$, $d_p$, $N_\mathrm{B}$, $\varepsilon I_\mathrm{SC}$, $\varepsilon \eta$),
increases the complexity of calculations due to the growing number of input parameters.
However, this should enhance the accuracy of $N_\mathrm{Fe}$ predictions.
Meanwhile, variations in short-circuit current and efficiency are not entirely independent,
as both essentially characterize the same physical process --- the diffusion of photo-induced carriers.
Confirmation of this has resulted from analyzing the correlations between parameters --- see Fig.~12 in the manuscript.
Thus, the set ($T$, $d_p$, $N_\mathrm{B}$, $\varepsilon I_\mathrm{SC}$, $\varepsilon \eta$)
contains useful and redundant information compared to ($T$, $d_p$, $N_\mathrm{B}$, $\varepsilon I_\mathrm{SC}$).
To assess the degree of redundancy in different descriptor sets, we employed principal component analysis (PCA),
which constructs a new dataset of uncorrelated principal components (PCs).
PCA also enables the evaluation of each PC’s contribution to the total variance of the information.
Our analysis showed that, for example, when estimating $N_\mathrm{Fe}$ using a set of
seven descriptors ($T$, $d_p$, $N_\mathrm{B}$, $\varepsilon I_\mathrm{SC}$, $\varepsilon \eta$, $\varepsilon V_\mathrm{OC}$, $\varepsilon F\!F$)
obtained under 940~nm illumination, five PCs account for more than 99.5\% of the variance in the input data.
Therefore, applying PCA to transform the seven input variables and using only five is fully justified,
as it significantly simplifies calculations with minimal impact on estimation accuracy.

In the manuscript, PCA was not used to evaluate impurity levels.
The conclusions and abstract only mentioned its potential usefulness in estimating the concentration of iron impurities
based on variations in several photovoltaic parameters.
We apologize for any possible ambiguities in the wording.
In the revised manuscript, we have aimed for the most precise formulations.
Finally, we would like to thank the Reviewer for an idea about PCA use to distinguish the contributions of different defects in future research.


The corrections were made in the manuscript on page 15, 1st paragraph in Section 3.5.


\begin{mdframed}
The previous subsections demonstrated that by using the relative changes in photovoltaic energy conversion parameters following the dissociation of FeB pairs
(e.g., changes in short-circuit current),
along with considering solar cell parameters (such as base depth and doping level)
and measurement conditions (such as temperature and illumination type), it is possible to predict iron concentration.
One approach to evaluating the value of $N_\mathrm{Fe}$ is to apply a machine learning algorithm (random forest, gradient booster, or artificial neural network etc.).
If we assume that the training process is conducted separately for each illumination variant \textcolor[rgb]{1.00,0.07,0.00}{ and want to
determine the magnitude of $N_\mathrm{Fe}$ based on changes in short-circuit current},
the input set of descriptors could be
($T$, $d_p$, $N_\mathrm{B}$, $\varepsilon I_\mathrm{SC}$).
\textcolor[rgb]{1.00,0.07,0.00}{Evidently, incorporating additional information, such as changes in efficiency ($\varepsilon \eta$), along with an expanded
set of descriptors ($T$, $d_p$, $N_\mathrm{B}$, $\varepsilon I_\mathrm{SC}$, $\varepsilon \eta$),
increases the complexity of calculations due to the growing number of input parameters.
However, this should enhance the accuracy of $N_\mathrm{Fe}$ predictions.
At the same time, }when comparing Fig.~3 and Fig.~10, it is clear that  $\varepsilon I_\mathrm{SC}$ and $\varepsilon \eta$ are not entirely independent.
\textcolor[rgb]{1.00,0.07,0.00}{This outcome is understandable from a physical standpoint as
short-circuit current and efficiency characterize the same physical process --- the diffusion of photo-induced carriers.
Thus, the set ($T$, $d_p$, $N_\mathrm{B}$, $\varepsilon I_\mathrm{SC}$, $\varepsilon \eta$)
contains useful and redundant information compared to ($T$, $d_p$, $N_\mathrm{B}$, $\varepsilon I_\mathrm{SC}$).}
\end{mdframed}


and on page 15, two last paragraph


\begin{mdframed}
To assess the \textcolor[rgb]{1.00,0.07,0.00}{degree} of redundancy information 
\textcolor[rgb]{1.00,0.07,0.00}{in different descriptor sets, we employed} principal component analysis (PCA),
widely used in solving various problems, including identifying defects in solar cells \cite{Fadhel2019}.
PCA uses a linear combination of the original variables to \textcolor[rgb]{1.00,0.07,0.00}{construct} the new variables (principal component, PC) while keeping maximum variance information.
PCs are \textcolor[rgb]{1.00,0.07,0.00}{uncorrelated, and PCA allows one to evaluate each PC’s contribution to the total variance of the information.}
In the case of some particular principal component having a low information variance ratio, it can be discarded with little to no loss of useful information.

PCs were built for different combinations of photovoltaic parameters and solar cell characteristics
for the full set of simulated data, and the results are listed in Table~\ref{tbl2}.
As evident, when only changes in short-circuit current are considered along with the solar cell's base parameters and temperature,
no excessive information is present \textcolor[rgb]{1.00,0.07,0.00}{(in the case of the descriptor set ($T$, $d_p$, $N_\mathrm{B}$, $\varepsilon I_\mathrm{SC}$),
all four principal components exhibit a high ratio of information variance)}.
\textcolor[rgb]{1.00,0.07,0.00}{And vice versa, when estimating $N_\mathrm{Fe}$ using a maximal set of
seven descriptors ($T$, $d_p$, $N_\mathrm{B}$, $\varepsilon I_\mathrm{SC}$, $\varepsilon \eta$, $\varepsilon V_\mathrm{OC}$, $\varepsilon F\!F$)
obtained under monochromatic illumination five PCs account for more than 99.5\% of the variance in the input data.
Therefore, applying PCA to transform the seven input variables and using only five is fully justified,
as it significantly simplifies calculations with minimal impact on estimation accuracy.
For the same set of descriptors obtained under AM1.5 illumination,
the number of really independent variables is six.
In general, if the estimation of iron concentration is based on changes in multiple photovoltaic parameters,
it is advisable to apply PCA to transform the original data.
Thus, the simultaneous use of $\varepsilon \eta$ and $\varepsilon I_\mathrm{SC}$ practically does not alter
the number of independent variables (four) compared to the initial set of descriptors.}
With the additional use of $\varepsilon V_\mathrm{OC}$ values obtained under AM1.5 illumination, it is advisable to consider five independent variables.
For the same set of descriptors in a monochromatic illumination case, one can limit oneself to using only four independent components
(cumulative variance ratio for PC4 and PC5 does not exceed 1.3\%).
\end{mdframed}



\vspace{1cm}
\noindent
\textcolor[rgb]{0.00,0.50,1.00}{\textbf{Comment~5.}}
\emph{Your paper discusses the influence of doping levels on the response of solar cells to iron contamination.
Author should explain how varying the doping concentration affects the sensitivity of photovoltaic parameters to iron presence through the reference:
Augmented photovoltaic performance of Cu/Ce-(Sn: Cd)/n-Si Schottky barrier diode utilizing dual-doped Ce-(Sn: Cd) thin films.}

\noindent
\textcolor[rgb]{0.51,0.00,0.00}{\textbf{Reply:}}


The primary reason doping concentration $N_\mathrm{B}$ influences the sensitivity of photovoltaic
parameters to iron presence is that the $N_\mathrm{B}$ determines the Fermi level $E_\mathrm{F}$ position.
In turn, the rate of Shockley-Read-Hall recombination ---
and consequently, variations in photovoltaic parameters due to iron impurities ---
depends on the relative positioning of $E_\mathrm{F}$ concerning the Fe$_i$B$_s$ and Fe$_i$ levels.
Additionally, the equilibrium ratio of Fe$_i$B$_s$ and Fe$_i$ concentrations is also determined by the Fermi level position \cite{FeB:kinetic,MurphyJAP2011}.
Therefore, $N_\mathrm{B}$ affects the number of defects that change their state due to the complete dissociation of Fe$_i$B$_s$ pairs,
leading to corresponding relative changes in photovoltaic parameters analyzed in this study.
Besides, the $N_\mathrm{B}$ concentration also affects the SCR depth (see Fig.~4 in manuscript).
The ratio of different charge states for both interstitial iron atoms and iron-boron pairs differs between the SCR and the bulk of the base, leading to distinct SRH recombination rates and variations in photovoltaic conversion efficiency for structures with different doping levels.
Furthermore, according to Klaassen’s model \cite{KLAASSEN953},
the concentration of ionized impurities influences charge carrier mobility and, consequently, the diffusion coefficient $D_n$, diffusion length $L_n$,
and photoelectric conversion efficiency (see Eqs.~(4), (5), and (7) in the manuscript).
However, the effect of $N_\mathrm{B}$ via $D_n$ and $L_n$ is significantly weaker than its influence through $E_\mathrm{F}$ and SCR depth variations.
Minor effects are expected as well due to both bandgap narrowing and changes in light absorption by free carriers with varying doping levels.
The impact of doping level on photovoltaic parameters has been studied in heterojunction \cite{Sultana2024},
thin-film \cite{Akila2024},
and perovskite \cite{MasumMia2025} solar cells.
Moreover, it has been shown \cite{Akila2024} that
doping concentration variations affect other barrier structure parameters, such as saturation current and ideality factor.


This answer is incorporated in the text on page 4, paragraph 4, line 1-2.
Reference  is included in the revised manuscript (references 1)




%У вашій статті обговорюється вплив рівня легування на реакцію сонячних елементів на забруднення залізом.
%Автору слід пояснити, як зміна концентрації легування впливає на чутливість фотоелектричних параметрів до присутності заліза через посилання:


%We would also like to express our gratitude to the Reviewer for providing exceptionally insightful references. We
%hope the Reviewer will not object to their inclusion in the revised manuscript (references 1, 2, 3, 91, 92).

\begin{mdframed}
...while the presence of the acceptor state Fe$_i$B$_s$ causes the
carrier lifetime to be doping-dependent under low injection conditions [FeB:Schmidt].

\textcolor[rgb]{1.00,0.07,0.00}{
The relationship between $N_\mathrm{B}$ and $E_\mathrm{F}$ is the key  factor
in the influence of doping level on the sensitivity of photovoltaic parameters to iron presence.
Specifically, the rate of SRH recombination ---
and consequently, variations in photovoltaic parameters due to iron impurities ---
depends on the relative positioning of $E_\mathrm{F}$ concerning the Fe$_i$B$_s$ and Fe$_i$ levels.
Additionally, the equilibrium ratio of Fe$_i$B$_s$ and Fe$_i$ concentrations is also determined by the Fermi level position [FeB:kinetic,MurphyJAP2011].
Therefore, $N_\mathrm{B}$ affects the number of defects that change their state due to the complete dissociation of Fe$_i$B$_s$ pairs,
leading to corresponding relative changes in photovoltaic parameters analyzed in this study.
Besides, the $N_\mathrm{B}$ concentration also affects the SCR depth (see Fig.~4).
The ratio of different charge states for both interstitial iron atoms and iron-boron pairs differs between the SCR and the bulk of the base, leading to distinct SRH recombination rates and variations in photovoltaic conversion efficiency for structures with different doping levels.
Furthermore, according to Klaassen’s model [KLAASSEN953],
the concentration of ionized impurities influences charge carrier mobility and, consequently, the diffusion coefficient $D_n$, diffusion length $L_n$,
and photoelectric conversion efficiency (see Eq.~(4)).
However, the effect of $N_\mathrm{B}$ via $D_n$ and $L_n$ is significantly weaker than its influence through $E_\mathrm{F}$ and SCR depth variations.
Minor effects are expected as well due to both bandgap narrowing and changes in light absorption by free carriers with varying doping levels.
The impact of doping level on photovoltaic parameters has been studied in heterojunction [8],
thin-film [10],
and perovskite [6] solar cells.
Moreover, it has been shown [10] that
doping concentration variations affect other barrier structure parameters, such as saturation current and ideality factor.
}
\end{mdframed}


\vspace{1cm}
\noindent
\textcolor[rgb]{0.00,0.50,1.00}{\textbf{Comment~6.}}
\emph{In your findings, Author mentions that changes in short-circuit current under monochromatic illumination are the most
reliable for estimating iron concentration.
An author should provide more details on the methodology used to derive this conclusion and any potential limitations of this approach?}

%У своїх висновках автор зазначає, що зміни струму короткого замикання при монохроматичному освітленні є найбільш
%надійними для оцінки концентрації заліза.
%Автор повинен надати більш детальну інформацію про методологію, використану для отримання цього висновку, і будь-які потенційні обмеження цього підходу?


\noindent
\textcolor[rgb]{0.51,0.00,0.00}{\textbf{Reply:}}

Ми використовували дві Key metrics для визначення параметра сонячного елементи, зміни якого внаслідок перебудови FeB->Fei дають можливість найбільш надійно визначити концентрацію заліза. А саме: 1) монотонність залежності відносних змін параметру як функції концентрації домішки: це необхідна умова для можливості однозначної оцінки Nfe;
2) абсолютні величини зміни параметру: зростання цієї величини дозволяє провести більш точні вимірювання, а отже і більш коректно оцінити Nfe.
Першому критерію не задовольняють $\varepsilon F\!F$, $\varepsilon V_\mathrm{OC}$, $\varepsilon \eta$ (два останні при АМ1.5 освітленні).
При однакових концентраціях заліза, величини $\varepsilon I_\mathrm{SC}$ при монрохроматичному освітленні переважають $\varepsilon I_\mathrm{SC}$ при АМ1.5.
Також при тотожніх умовах значення змін напруги холостого ходу суттєво менші ніж для струму короткого замикання чи ефективності.

 

Величини $\varepsilon V_\mathrm{OC}$ для сонячних елементів з однакових концентраціях заліза 


In addition, when comparing monochromatic and solar illumination,
the latter generates a significantly larger number of non-equilibrium carriers in the emitter due to shorter-wavelength
photons.


Проведені дослідження показали, що 

($T$, $d_p$, $N_\mathrm{B}$, $\varepsilon I_\mathrm{SC}$, $\varepsilon \eta$, $\varepsilon V_\mathrm{OC}$, $\varepsilon F\!F$)

For example, the most evident conditions for utilizing a specific parameter
include its change due to the transformation Fe𝑖BSi → Fe𝑖 +BSi, at least by 10 %, along with a monotonic dependence
of these changes on 𝑁Fe.

Key metrics for reliability:

\begin{itemize}
    \item \textbf{Monotonicity}: $\epsilon I_\mathrm{SC}$ under monochromatic light exhibited a linear, monotonic relationship with $N_\mathrm{Fe}$ across all $N_\mathrm{B}$ and $T$, unlike $V_\mathrm{OC}$ or $FF$, which showed non-monotonic trends;
    \item \textbf{Sensitivity}: at $N_\mathrm{B}$ = $10^{17}$ cm$^{-3}$, $\epsilon I_\mathrm{SC}$ under 940~nm changed by >100~\% for $N_\mathrm{Fe}$ = $10^{14}$ cm$^{-3}$, vs. <30~\% under AM1.5;
    \item \textbf{Correlation strength}: Pearson correlation coefficient ?
\end{itemize}

Validation:

\begin{itemize}
    \item \textbf{Experimental agreement}: measured $\epsilon I_\mathrm{SC}$ (940 nm) matched simulations after applying a correction factor $C_\mathrm{cor}$ = 1.4 to account for unmodeled effects (e.g. shunt resistance, surface passivation).
\end{itemize}

Limitations of the approach:

1) Real solar cells operate under AM1.5, so monochromatic-based $N_\mathrm{Fe}$ estimates can not predict performance under standard conditions.


2) Co-existing impurities (e.g. Cu, Ni) could introduce overlapping recombination signals, though this study focused exclusively on iron.




%\begin{mdframed}
%
%\end{mdframed}

\vspace{1cm}
\noindent
\textcolor[rgb]{0.00,0.50,1.00}{\textbf{Comment~7.}}
\emph{An Author must improve the introduction section in the application part through the recent referencs
CuO-La2O3 Composite-Enabled MIS Schottky Barrier Diodes: A Novel Approach to Optoelectronic Device Diversification;
Enhancing photovoltaic applications through precipitating agents in ITO/CIS/CeO2/Al heterojunction solar cell;
Manifestation on the choice of a suitable combination of MIS for proficient Schottky diodes for optoelectronics applications: A comprehensive review.}

\cite{Paul2024,Gayathri2024,AlanSibu2024}

We sincerely thank the Reviewer for suggesting the inclusion of recent references (cited as References 2, 3, and 9 in the revised manuscript).
Below, we present an improved version of the 1st paragraph of Introduction.


\begin{mdframed}
The necessity for renewable energy sources to meet the growing global demand for sustainable and environmentally friendly energy alternatives has become evident.
Among the wide range of renewable energy sources, sunlight is the cleanest, safest,
and most abundant source for use in sustainable energy to support economic growth [1].
The utilization of solar energy heavily depends on the use of photovoltaic cells.
\textcolor[rgb]{1.00,0.07,0.00}{
The development of next-generation solar cells is primarily driven by the search for novel materials suitable for their fabrication.
Particular attention is given to exploring the potential use of composites and nanoparticles
(e.g., Cu-La-based systems [2], cerium oxide, and copper indium sulfate [3]),
MAX phases [4,5],
and chalcogenides (such as GeSe, MoSe$_2$, Sb$_2$Se$_3$, and SnSe as hole transport materials [6]
or SnS$_2$ and WS$_2$ as electron transport materials [7]),
as well as metal silicides (e.g., FeSi$_2$ as an absorber layer [8]).
These studies cover a broad range of materials [9], with the primary goal of developing highly efficient and cost-effective photovoltaic devices.
Frequently, these novel materials are integrated with silicon in solar cells [8,10].
However, it is worth noting that traditional} silicon-based devices
\textcolor[rgb]{1.00,0.07,0.00}{continue} to play
\textcolor[rgb]{1.00,0.07,0.00}{a dominant role in the photovoltaic market} [11,12].
\end{mdframed}




\noindent
\textcolor[rgb]{0.51,0.00,0.00}{\textbf{Reply:}}





\subsection*{Response to Reviewer \#2 }

\noindent
\textcolor[rgb]{0.00,0.50,1.00}{\textbf{Comment~1.}}
\emph{The abstract section should be more informative.}

\noindent
\textcolor[rgb]{0.51,0.00,0.00}{\textbf{Reply:}}




\noindent
\textcolor[rgb]{0.00,0.50,1.00}{\textbf{Comment~2.}}
\emph{The novelty of the work is missing in the introduction. Authors should explain what are the key advantages of iron's impact on silicon solar cell?}

%У вступі відсутня новизна роботи. Авторам варто було б пояснити, в чому полягають ключові переваги впливу заліза на кремнієвий сонячний елемент?

\noindent
\textcolor[rgb]{0.51,0.00,0.00}{\textbf{Reply:}}




\noindent
\textcolor[rgb]{0.00,0.50,1.00}{\textbf{Comment~3.}}
\emph{Authors should improve the image quality of all figures.}

%Автори повинні покращити якість зображення всіх рисунків.

\noindent
\textcolor[rgb]{0.51,0.00,0.00}{\textbf{Reply:}}




\noindent
\textcolor[rgb]{0.00,0.50,1.00}{\textbf{Comment~4.}}
\emph{Author should explain how does the band alignment affect the overall performance of the solar cell?}

%Автор повинен пояснити, як вирівнювання смуг впливає на загальну продуктивність сонячного елемента?

\noindent
\textcolor[rgb]{0.51,0.00,0.00}{\textbf{Reply:}}




\noindent
\textcolor[rgb]{0.00,0.50,1.00}{\textbf{Comment~5.}}
\emph{What are the primary sources of iron contamination in silicon used for solar cells?}

%Які основні джерела забруднення залізом кремнію, що використовується для сонячних елементів?

\noindent
\textcolor[rgb]{0.51,0.00,0.00}{\textbf{Reply:}}




\noindent
\textcolor[rgb]{0.00,0.50,1.00}{\textbf{Comment~6.}}
\emph{What role do recombination centers created by iron play in the modeling of solar cell performance?}

%Яку роль відіграють центри рекомбінації, створені залізом, у моделюванні роботи сонячних елементів?

\noindent
\textcolor[rgb]{0.51,0.00,0.00}{\textbf{Reply:}}




\noindent
\textcolor[rgb]{0.00,0.50,1.00}{\textbf{Comment~7.}}
\emph{Author should discuss and cited recent Si based solar cell in the revised manuscript:
$DOI:10.1016/j.mseb.2024.117360,
DOI:10.1016/j.mseb.2023.117141,
DOI:10.1016/j.mseb.2024.117817,
DOI:10.1007/s42247-024-00821-y,
DOI:10.1016/j.inoche.2024.112785$}

\noindent
\textcolor[rgb]{0.51,0.00,0.00}{\textbf{Reply:}}


We sincerely appreciate the reviewer’s valuable suggestion.
Most of the recommended works are highly relevant, and we have cited them in multiple sections of the manuscript.
Advised articles are referenced as References 8, 4, 6, 7, and 5 in the revised version.

So, we discuss and cited suggested papers in the Introduction, first paragraph

\begin{mdframed}
The necessity for renewable energy sources to meet the growing global demand for sustainable and environmentally friendly energy alternatives has become evident.
Among the wide range of renewable energy sources, sunlight is the cleanest, safest,
and most abundant source for use in sustainable energy to support economic growth [1].
The utilization of solar energy heavily depends on the use of photovoltaic cells.
\textcolor[rgb]{1.00,0.07,0.00}{
The development of next-generation solar cells is primarily driven by the search for novel materials suitable for their fabrication.
Particular attention is given to exploring the potential use of composites and nanoparticles
(e.g., Cu-La-based systems [2], cerium oxide, and copper indium sulfate [3]),
MAX phases [4,5],
and chalcogenides (such as GeSe, MoSe$_2$, Sb$_2$Se$_3$, and SnSe as hole transport materials [6]
or SnS$_2$ and WS$_2$ as electron transport materials [7]),
as well as metal silicides (e.g., FeSi$_2$ as an absorber layer [8])}
\end{mdframed}

on page 3, first paragraph from the top


\begin{mdframed}
Despite its one-dimensional modeling approach, SCAPS is extensively used for modeling various types
of solar cells [\textcolor[rgb]{1.00,0.07,0.00}{6,7,8},Joshi2024,Ravidas2024,Liu2024,You2023] in general
and for investigating the effects of defects on their performance [\textcolor[rgb]{1.00,0.07,0.00}{6,8,}AitAbdelkadir2023,Liang2024,SCAPSDefect3] in particular.
\end{mdframed}


on page 4, last paragraph

\begin{mdframed}
Impact of change of iron defects was investigated as a function of temperature from 290~K to 340~K,
base depth from 180~$\mu$m to 380~$\mu$m,
base doping level from $10^{15}$~cm$^{-3}$ to $10^{17}$~cm$^{-3}$,
and total impurity iron atom concentration from $10^{10}$~cm$^{-3}$ to $10^{14}$~cm$^{-3}$.
\textcolor[rgb]{1.00,0.07,0.00}{It is worth noting that investigating the effects of doping density, defect density, temperature, and active layer thickness on photovoltaic parameters is a well-established practice [6,7,8].}
For each illumination scenario, calculations were carried out with 11 different temperature values and 5 base depth values,
evenly distributed within the specified ranges.
\end{mdframed}

on page 7, 3rd paragraph from the bottom

\begin{mdframed}
However, the effect of $N_\mathrm{B}$ via $D_n$ and $L_n$ is significantly weaker than its influence through $E_\mathrm{F}$ and SCR depth variations.
Minor effects are expected as well due to both bandgap narrowing and changes in light absorption by free carriers with varying doping levels.
\textcolor[rgb]{1.00,0.07,0.00}{The impact of doping level on photovoltaic parameters has been studied in heterojunction [8],
thin-film [10],
and perovskite [6] solar cells.}
Moreover, it has been shown [10] that
doping concentration variations affect other barrier structure parameters, such as saturation current and ideality factor.
\end{mdframed}

on page 7, 4th paragraph from the top

\begin{mdframed}
- At low boron concentrations ($N_\mathrm{B}<10^{16}$~cm$^{-3}$),
the dependence of $\varepsilon F\!F$ on $N_\mathrm{Fe}$ is notably non-linear.
Within the used concentration range, two regions of decrease and two regions of increase in $\varepsilon F\!F$ are observed.
\textcolor[rgb]{1.00,0.07,0.00}{A similar relationship between the fill factor and defect concentration was observed in BaZrSe$_3$--based perovskite solar cells [6]};
\end{mdframed}




%\cite{Sultana2024,MasumMia2025,Rahman2024}
%
%
%\cite{Sultana2024,Behera2024,MasumMia2025,Rahman2024,AzzouzRached2024}
%         8           4             6          7            5

\noindent
\textcolor[rgb]{0.00,0.50,1.00}{\textbf{Comment~8.}}
\emph{How does the modeling in this study contribute to the design of processes for impurity control in silicon?}

%Як моделювання в цьому дослідженні сприяє розробці процесів контролю домішок у кремнії?

\noindent
\textcolor[rgb]{0.51,0.00,0.00}{\textbf{Reply:}}




\noindent
\textcolor[rgb]{0.00,0.50,1.00}{\textbf{Comment~9.}}
\emph{State the main findings in the conclusions.}

%У висновках викладіть основні результати.

\noindent
\textcolor[rgb]{0.51,0.00,0.00}{\textbf{Reply:}}


\bibliographystyle{model1-num-names}
\bibliography{olikh}


\end{document}

%Будь-ласка, покращ англійську в наступних реченнях, які я буду пропонувати. Зокрема, треба буде виправити граматичні та стилістичні помилки
%якщо можна, окрім виправленого речення, наводь інформацію про зміни, які зроблені 