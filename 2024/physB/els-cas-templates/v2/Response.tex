

%\documentclass[aip,reprint]{revtex4-1}
%\documentclass[aip,jap,preprint]{revtex4-1}
\documentclass[a4paper,fleqn]{cas-sc}

\usepackage[numbers]{natbib}
\usepackage{mdframed}
\usepackage{color}


\begin{document}
\shorttitle{}


%Dear Editor,
%
%We like to express our appreciation to the reviewers for their comments.
%We are resubmitting the revised version of the paper number MSB-S-24-02710.
%We have studied the comments of the reviewer carefully, and have changed the text according to the comments they
%have listed.
%%The location of revisions is pointed by blue color in ``MarkedManuscript.pdf''.
%Below we refer to each of the reviewer’s comments.


Dear Editor and Reviewers,

We sincerely thank you for taking the time to review our manuscript
``Iron's impact on silicon solar cell execution: comprehensive modeling across diverse scenarios''
(Ms. Ref. No.: MSB-S-24-02710).
Your insightful comments and constructive suggestions have greatly helped us improve
the quality of our work.
We particularly appreciate your careful reading and thoughtful feedback, which have
led to significant improvements in both the technical content and presentation clarity of our manuscript.
We have carefully addressed all the comments and made corresponding revisions to the manuscript.
%All changes are marked in red in the revised version.
%The location of revisions is pointed by blue color in ``MarkedManuscript.pdf''.
Below we provide our detailed point-by-point responses to
each comment.
We hope the revised manuscript better meets your expectations and standards for publication in Materials Science and Engineering: B.


\subsection*{Response to Reviewer \#1 }

\noindent
\textcolor[rgb]{0.00,0.50,1.00}{\textbf{Comment~1.}}
\emph{The author must explain the specific mechanisms you propose for the interaction between iron-related defects and other impurities in silicon solar cells? How might these interactions complicate the interpretation of photovoltaic performance metrics?}

\noindent
\textcolor[rgb]{0.51,0.00,0.00}{\textbf{Reply:}}
In our study, we primarily focus on the impact of iron-related defects, particularly $\mathrm{Fe}_i$ and $\mathrm{Fe}_i\mathrm{B}_s$ pairs, on the photovoltaic parameters of silicon solar cells. However, we acknowledge that the interaction of iron with other impurities (such as oxygen, carbon, or transition metals) can introduce additional complexities in interpreting photovoltaic performance.

Mechanisms of Interaction:
\begin{itemize}
    \item \textbf{iron-oxygen complexes}: oxygen is a common impurity in Czochralski-grown silicon and can form stable complexes with interstitial iron ($\mathrm{Fe}_i$-O). These complexes alter the recombination activity of $\mathrm{Fe}_i$ and may reduce the effectiveness of $\mathrm{Fe}_i\mathrm{B}_s$ pair dissociation as a means to estimate iron concentration;
    \item \textbf{iron-carbon interactions}: carbon-related defects, such as Fe-C pairs, may modify the charge state of $\mathrm{Fe}_i$, influencing its recombination properties and affecting the carrier lifetime. This interaction can lead to deviations in the expected trends of $I_\mathrm{SC}$ and $V_\mathrm{OC}$;
    \item \textbf{transition metal contamination}: if other transition metals (Cu, Ni, etc.) are present, they can introduce additional deep-level traps, competing with iron-related recombination centers. This may lead to non-monotonic behavior in efficiency trends, complicating the interpretation of photovoltaic responses.
\end{itemize}


Impact on Photovoltaic Metrics:
\begin{itemize}
    \item the presence of multiple recombination centers can lead to overlapping effects in $V_\mathrm{OC}$ and $I_\mathrm{SC}$, making it difficult to isolate the influence of Fe-related defects;
    \item some impurity interactions may cause non-monotonic trends in efficiency and fill factor changes, particularly under varying injection conditions;
    \item the temperature dependence of defect interactions may further introduce variability in photovoltaic responses, as different impurities have different activation energies.
\end{itemize}


$\mathrm{Fe}_i\mathrm{B}_s$ pairs introduce deep-level recombination centers that strongly impact minority carrier lifetime, leading to reduced $I_\mathrm{SC}$ and $V_\mathrm{OC}$. When $\mathrm{Fe}_i\mathrm{B}_s$ pairs dissociate, the more mobile $\mathrm{Fe}_i$ leads to different recombination dynamics, which is particularly evident under monochromatic (940~nm) illumination. Our study demonstrates that changes in $I_\mathrm{SC}$ after $\mathrm{Fe}_i\mathrm{B}_s$ dissociation can serve as a reliable method to estimate Fe concentration.
%The aim of the research was added in the introduction part
%(page~2, paragraph~3 (last paragraph in the Introduction), line~1-2).

\vspace{1cm}
\noindent
\textcolor[rgb]{0.00,0.50,1.00}{\textbf{Comment~2.}}
\emph{ The author should give the transient nature of defect dynamics, how do you assess the long-term stability of silicon solar cells in the presence of iron contamination? What experimental approaches would you recommend to study the aging effects of these defects over time?}

\noindent
\textcolor[rgb]{0.51,0.00,0.00}{\textbf{Reply:}}

The transient nature of defect dynamics, particularly the dissociation and recombination of $\mathrm{Fe}_i\mathrm{B}_s$ pairs, plays a significant role in the long-term stability of silicon solar cells.


\textbf{Long-term stability of silicon solar cells in the presence of iron contamination}


Under illumination or thermal excitation, $\mathrm{Fe}_i\mathrm{B}_s$ pairs dissociate into interstitial iron ($\mathrm{Fe}_i$) and substitutional boron ($\mathrm{B}_s$). In the absence of external excitation (e.g., during storage in the dark), $\mathrm{Fe}_i$ captures an electron and recombines with $\mathrm{B}_s$, restoring the $\mathrm{Fe}_i\mathrm{B}_s$ pair. This dynamic equilibrium between $\mathrm{Fe}_i\mathrm{B}_s$ and $\mathrm{Fe}_i$ results in time-dependent variations in carrier lifetime and photovoltaic parameters. Over extended periods, $\mathrm{Fe}_i$ can diffuse and form precipitates, particularly at dislocations and grain boundaries. These precipitates act as deep recombination centers and may partially dissolve under elevated temperatures modifying solar cell performance.


\textbf{Recommended experimental approaches to study aging effects:}


\begin{itemize}
    \item \textbf{time-Resolved Minority Carrier Lifetime Measurements:} quasi-Steady-State Photoconductance (QSSPC) and microwave photoconductance decay ($\mu$-PCD) can track lifetime variations due to $\mathrm{Fe}_i\mathrm{B}_s$ recombination-dissociation cycles. Measuring at different injection levels provides insight into the injection-dependent recombination activity of $\mathrm{Fe}_i\mathrm{B}_s$ pairs;
    \item \textbf{deep-level transient spectroscopy (DLTS):} DLTS can resolve the capture cross-sections and activation energies of Fe-related defects, distinguishing $\mathrm{Fe}_i$ from $\mathrm{Fe}_i\mathrm{B}_s$ states. This provides a quantitative approach to monitoring defect evolution over time;
    \item \textbf{illumination-induced defect metastability studies:} electroluminescence (EL) and photoluminescence (PL) imaging can spatially track Fe-related degradation over time;
\end{itemize}

\textbf{Implications for silicon solar cell reliability:}

\begin{itemize}
    \item fluctuations in $\mathrm{Fe}_i\mathrm{B}_s$ Pair Concentration impact solar cell efficiency due to time-dependent variations in recombination rates;
    \item iron precipitation vs. $\mathrm{Fe}_i\mathrm{B}_s$ recombination equilibrium  dictates whether degradation is reversible or permanent;
    \item understanding $\mathrm{Fe}_i\mathrm{B}_s$ defect dynamics is critical for optimizing gettering strategies and passivation techniques to enhance long-term device performance;
\end{itemize}


\vspace{1cm}
\noindent
\textcolor[rgb]{0.00,0.50,1.00}{\textbf{Comment~3.}}
\emph{An Author should provide a theoretical framework for understanding how the temperature range you studied (290 K to 340 K) influences the activation energy of iron-related defects? How might this affect the performance of solar cells in varying environmental conditions?}

\noindent
\textcolor[rgb]{0.51,0.00,0.00}{\textbf{Reply:}}

The temperature range studied in our work (290 K to 340 K) was chosen to reflect realistic operating conditions of silicon solar cells, particularly those exposed to moderate climate variations. The behavior of iron-boron ($\mathrm{Fe}_i\mathrm{B}_s$) pairs and interstitial iron ($\mathrm{Fe}_i$) is strongly temperature-dependent due to thermal excitation and defect migration effects. The key theoretical considerations include:

\begin{itemize}
    \item Thermal dissociation of $\mathrm{Fe}_i\mathrm{B}_s$ pairs:
    \begin{itemize}
        \item the equilibrium between $\mathrm{Fe}_i\mathrm{B}_s$ pairs and free $\mathrm{Fe}_i$ follows an Arrhenius-type relation, governed by the dissociation rate:
        \begin{equation}
        \label{eq1}
            k_\mathrm{d}  = k_\mathrm{0} \exp(-\frac{E_\mathrm{a}}{k_\mathrm{b}T}),
        \end{equation}
        where $k_\mathrm{d}$ is the dissociation rate, $E_\mathrm{a}$ is the activation energy of $\mathrm{Fe}_i\mathrm{B}_s$ dissociation, $k_\mathrm{b}$ is the Boltzmann constant, and $T$ is the absolute temperature;
        \item prior studies estimate Ea $\approx$ 0.7 $-$ 0.8 eV for $\mathrm{Fe}_i\mathrm{B}_s$ dissociation in silicon;
        \item as temperature increases, $\mathrm{Fe}_i\mathrm{B}_s$ pairs dissociate more readily, increasing $\mathrm{Fe}_i$ concentration, which enhances recombination activity and reduces carrier lifetimes.
    \end{itemize}
    \item Shockley-Read-Hall recombination dependence:
    \begin{itemize}
        \item the capture cross-section ($\sigma$) of $\mathrm{Fe}_i$ defects is temperature-dependent, influencing the recombination rate $U$ given by:
        \begin{equation}
        \label{eq2}
            U  = \frac{\sigma v_\mathrm{th} N_\mathrm{d}}{1 + \exp(\frac{E_\mathrm{t} - E_\mathrm{F}}{k_\mathrm{b}T})},
        \end{equation}
        where $v_\mathrm{th}$ is the thermal velocity of carriers, $N_\mathrm{d}$ is defect concentration, and $E_\mathrm{t}$ is the defect energy level;
        \item as temperature rises, carrier thermal velocity increases, altering the recombination dynamics and further impacting photovoltaic parameters.
    \end{itemize}
\end{itemize}

The temperature-dependent behavior of iron-related defects has direct consequences on solar cell operation under different environmental conditions:
\begin{itemize}
    \item Reduction in open-circuit voltage ($V_\mathrm{OC}$):
    \begin{itemize}
        \item increased FeB dissociation at high temperatures raises $\mathrm{Fe}_i$ concentration, enhancing Shockley-Read-Hall recombination. Since $V_\mathrm{OC}$ is governed by the equation:
        \begin{equation}
        \label{eq3}
             V_\mathrm{OC} = \frac{k_\mathrm{b}T}{q} log\left(\frac{I_\mathrm{SC}}{I_\mathrm{0}} + 1\right),
        \end{equation}
        where $I_\mathrm{0}$ is the saturation current, higher recombination rates increase $I_\mathrm{0}$, leading to a reduction in $V_\mathrm{OC}$;
        \item the observed temperature coefficient of ($V_\mathrm{OC} \approx -2.3~\frac{mV}{K}$) aligns with known iron-related degradation effects.
    \end{itemize}
    \item Short-circuit current $I_\mathrm{SC}$ variability:
    \begin{itemize}
        \item as $\mathrm{Fe}_i$ concentration rises, the minority carrier lifetime decreases, reducing carrier collection efficiency and thus $I_\mathrm{SC}$. However, in certain temperature ranges, increased carrier mobility may partially compensate for these losses.
     \end{itemize}
     \item Thermal cycling and long-term degradation risks:
     \begin{itemize}
        \item field-deployed solar cells experience daily temperature fluctuations, which can lead to repeated FeB dissociation and recombination cycles, contributing to metastable defect states. Over time,$\mathrm{Fe}_i$ precipitation may lead to irreversible efficiency losses, especially in high-temperature climates.
     \end{itemize}
\end{itemize}

\vspace{1cm}
\noindent
\textcolor[rgb]{0.00,0.50,1.00}{\textbf{Comment~4.}}
\emph{An Author mentioned the application of principal component analysis in evaluating impurity levels. Could you elaborate on how this technique was implemented in your study and its effectiveness in distinguishing between different types of impurities?}

\noindent
\textcolor[rgb]{0.51,0.00,0.00}{\textbf{Reply:}}




\vspace{1cm}
\noindent
\textcolor[rgb]{0.00,0.50,1.00}{\textbf{Comment~5.}}
\emph{Your paper discusses the influence of doping levels on the response of solar cells to iron contamination.
Author should explain how varying the doping concentration affects the sensitivity of photovoltaic parameters to iron presence through the reference:
Augmented photovoltaic performance of Cu/Ce-(Sn: Cd)/n-Si Schottky barrier diode utilizing dual-doped Ce-(Sn: Cd) thin films.}

\noindent
\textcolor[rgb]{0.51,0.00,0.00}{\textbf{Reply:}}


The primary reason doping concentration $N_\mathrm{B}$ influences the sensitivity of photovoltaic
parameters to iron presence is that the $N_\mathrm{B}$ determines the Fermi level $E_\mathrm{F}$ position.
In turn, the rate of Shockley-Read-Hall recombination ---
and consequently, variations in photovoltaic parameters due to iron impurities ---
depends on the relative positioning of $E_\mathrm{F}$ concerning the Fe$_i$B$_s$ and Fe$_i$ levels.
Additionally, the equilibrium ratio of Fe$_i$B$_s$ and Fe$_i$ concentrations is also determined by the Fermi level position \cite{FeB:kinetic,MurphyJAP2011}.
Therefore, $N_\mathrm{B}$ affects the number of defects that change their state due to the complete dissociation of Fe$_i$B$_s$ pairs,
leading to corresponding relative changes in photovoltaic parameters analyzed in this study.
Besides, the $N_\mathrm{B}$ concentration also affects the SCR depth (see Fig.~4 in manuscript).
The ratio of different charge states for both interstitial iron atoms and iron-boron pairs differs between the SCR and the bulk of the base, leading to distinct SRH recombination rates and variations in photovoltaic conversion efficiency for structures with different doping levels.
Furthermore, according to Klaassen’s model \cite{KLAASSEN953},
the concentration of ionized impurities influences charge carrier mobility and, consequently, the diffusion coefficient $D_n$, diffusion length $L_n$,
and photoelectric conversion efficiency (see Eqs.~(4), (5), and (7) in the manuscript).
However, the effect of $N_\mathrm{B}$ via $D_n$ and $L_n$ is significantly weaker than its influence through $E_\mathrm{F}$ and SCR depth variations.
Minor effects are expected as well due to both bandgap narrowing and changes in light absorption by free carriers with varying doping levels.
The impact of doping level on photovoltaic parameters has been studied in heterojunction \cite{Sultana2024},
thin-film \cite{Akila2024},
and perovskite \cite{MasumMia2025} solar cells.
Moreover, it has been shown \cite{Akila2024} that
doping concentration variations affect other barrier structure parameters, such as saturation current and ideality factor.


This answer is incorporated in the text on page 4, paragraph 4, line 1-2.
Reference  is included in the revised manuscript (references 1)




%У вашій статті обговорюється вплив рівня легування на реакцію сонячних елементів на забруднення залізом.
%Автору слід пояснити, як зміна концентрації легування впливає на чутливість фотоелектричних параметрів до присутності заліза через посилання:


%We would also like to express our gratitude to the Reviewer for providing exceptionally insightful references. We
%hope the Reviewer will not object to their inclusion in the revised manuscript (references 1, 2, 3, 91, 92).

\begin{mdframed}
...while the presence of the acceptor state Fe$_i$B$_s$ causes the
carrier lifetime to be doping-dependent under low injection conditions [FeB:Schmidt].

\textcolor[rgb]{1.00,0.07,0.00}{
The relationship between $N_\mathrm{B}$ and $E_\mathrm{F}$ is the key  factor
in the influence of doping level on the sensitivity of photovoltaic parameters to iron presence.
Specifically, the rate of SRH recombination ---
and consequently, variations in photovoltaic parameters due to iron impurities ---
depends on the relative positioning of $E_\mathrm{F}$ concerning the Fe$_i$B$_s$ and Fe$_i$ levels.
Additionally, the equilibrium ratio of Fe$_i$B$_s$ and Fe$_i$ concentrations is also determined by the Fermi level position [FeB:kinetic,MurphyJAP2011].
Therefore, $N_\mathrm{B}$ affects the number of defects that change their state due to the complete dissociation of Fe$_i$B$_s$ pairs,
leading to corresponding relative changes in photovoltaic parameters analyzed in this study.
Besides, the $N_\mathrm{B}$ concentration also affects the SCR depth (see Fig.~4).
The ratio of different charge states for both interstitial iron atoms and iron-boron pairs differs between the SCR and the bulk of the base, leading to distinct SRH recombination rates and variations in photovoltaic conversion efficiency for structures with different doping levels.
Furthermore, according to Klaassen’s model [KLAASSEN953],
the concentration of ionized impurities influences charge carrier mobility and, consequently, the diffusion coefficient $D_n$, diffusion length $L_n$,
and photoelectric conversion efficiency (see Eq.~(4)).
However, the effect of $N_\mathrm{B}$ via $D_n$ and $L_n$ is significantly weaker than its influence through $E_\mathrm{F}$ and SCR depth variations.
Minor effects are expected as well due to both bandgap narrowing and changes in light absorption by free carriers with varying doping levels.
The impact of doping level on photovoltaic parameters has been studied in heterojunction [8],
thin-film [10],
and perovskite [6] solar cells.
Moreover, it has been shown [10] that
doping concentration variations affect other barrier structure parameters, such as saturation current and ideality factor.
}
\end{mdframed}


\vspace{1cm}
\noindent
\textcolor[rgb]{0.00,0.50,1.00}{\textbf{Comment~6.}}
\emph{In your findings, Author mentions that changes in short-circuit current under monochromatic illumination are the most
reliable for estimating iron concentration.
An author should provide more details on the methodology used to derive this conclusion and any potential limitations of this approach?}

%У своїх висновках автор зазначає, що зміни струму короткого замикання при монохроматичному освітленні є найбільш
%надійними для оцінки концентрації заліза.
%Автор повинен надати більш детальну інформацію про методологію, використану для отримання цього висновку, і будь-які потенційні обмеження цього підходу?


\noindent
\textcolor[rgb]{0.51,0.00,0.00}{\textbf{Reply:}}




\vspace{1cm}
\noindent
\textcolor[rgb]{0.00,0.50,1.00}{\textbf{Comment~7.}}
\emph{An Author must improve the introduction section in the application part through the recent referencs
CuO-La2O3 Composite-Enabled MIS Schottky Barrier Diodes: A Novel Approach to Optoelectronic Device Diversification;
Enhancing photovoltaic applications through precipitating agents in ITO/CIS/CeO2/Al heterojunction solar cell;
Manifestation on the choice of a suitable combination of MIS for proficient Schottky diodes for optoelectronics applications: A comprehensive review.}

\cite{Paul2024,Gayathri2024,AlanSibu2024}

We sincerely thank the Reviewer for suggesting the inclusion of recent references (cited as References 2, 3, and 9 in the revised manuscript).
Below, we present an improved version of the 1st paragraph of Introduction.


\begin{mdframed}
The necessity for renewable energy sources to meet the growing global demand for sustainable and environmentally friendly energy alternatives has become evident.
Among the wide range of renewable energy sources, sunlight is the cleanest, safest,
and most abundant source for use in sustainable energy to support economic growth [1].
The utilization of solar energy heavily depends on the use of photovoltaic cells.
\textcolor[rgb]{1.00,0.07,0.00}{
The development of next-generation solar cells is primarily driven by the search for novel materials suitable for their fabrication.
Particular attention is given to exploring the potential use of composites and nanoparticles
(e.g., Cu-La-based systems [2], cerium oxide, and copper indium sulfate [3]),
MAX phases [4,5],
and chalcogenides (such as GeSe, MoSe$_2$, Sb$_2$Se$_3$, and SnSe as hole transport materials [6]
or SnS$_2$ and WS$_2$ as electron transport materials [7]),
as well as metal silicides (e.g., FeSi$_2$ as an absorber layer [8]).
These studies cover a broad range of materials [9], with the primary goal of developing highly efficient and cost-effective photovoltaic devices.
Frequently, these novel materials are integrated with silicon in solar cells [8,10].
However, it is worth noting that traditional} silicon-based devices
\textcolor[rgb]{1.00,0.07,0.00}{continue} to play
\textcolor[rgb]{1.00,0.07,0.00}{a dominant role in the photovoltaic market} [11,12].
\end{mdframed}




\noindent
\textcolor[rgb]{0.51,0.00,0.00}{\textbf{Reply:}}





\subsection*{Response to Reviewer \#2 }

\noindent
\textcolor[rgb]{0.00,0.50,1.00}{\textbf{Comment~1.}}
\emph{The abstract section should be more informative.}

\noindent
\textcolor[rgb]{0.51,0.00,0.00}{\textbf{Reply:}}




\noindent
\textcolor[rgb]{0.00,0.50,1.00}{\textbf{Comment~2.}}
\emph{The novelty of the work is missing in the introduction. Authors should explain what are the key advantages of iron's impact on silicon solar cell?}

\noindent
\textcolor[rgb]{0.51,0.00,0.00}{\textbf{Reply:}}




\noindent
\textcolor[rgb]{0.00,0.50,1.00}{\textbf{Comment~3.}}
\emph{Authors should improve the image quality of all figures.}

\noindent
\textcolor[rgb]{0.51,0.00,0.00}{\textbf{Reply:}}




\noindent
\textcolor[rgb]{0.00,0.50,1.00}{\textbf{Comment~4.}}
\emph{Author should explain how does the band alignment affect the overall performance of the solar cell?}

\noindent
\textcolor[rgb]{0.51,0.00,0.00}{\textbf{Reply:}}




\noindent
\textcolor[rgb]{0.00,0.50,1.00}{\textbf{Comment~5.}}
\emph{What are the primary sources of iron contamination in silicon used for solar cells?}

\noindent
\textcolor[rgb]{0.51,0.00,0.00}{\textbf{Reply:}}




\noindent
\textcolor[rgb]{0.00,0.50,1.00}{\textbf{Comment~6.}}
\emph{What role do recombination centers created by iron play in the modeling of solar cell performance?}

\noindent
\textcolor[rgb]{0.51,0.00,0.00}{\textbf{Reply:}}




\noindent
\textcolor[rgb]{0.00,0.50,1.00}{\textbf{Comment~7.}}
\emph{Author should discuss and cited recent Si based solar cell in the revised manuscript:
$DOI:10.1016/j.mseb.2024.117360,
DOI:10.1016/j.mseb.2023.117141,
DOI:10.1016/j.mseb.2024.117817,
DOI:10.1007/s42247-024-00821-y,
DOI:10.1016/j.inoche.2024.112785$}

\noindent
\textcolor[rgb]{0.51,0.00,0.00}{\textbf{Reply:}}


We sincerely appreciate the reviewer’s valuable suggestion.
Most of the recommended works are highly relevant, and we have cited them in multiple sections of the manuscript.
Advised articles are referenced as References 8, 4, 6, 7, and 5 in the revised version.

So, we discuss and cited suggested papers in the Introduction, first paragraph

\begin{mdframed}
The necessity for renewable energy sources to meet the growing global demand for sustainable and environmentally friendly energy alternatives has become evident.
Among the wide range of renewable energy sources, sunlight is the cleanest, safest,
and most abundant source for use in sustainable energy to support economic growth [1].
The utilization of solar energy heavily depends on the use of photovoltaic cells.
\textcolor[rgb]{1.00,0.07,0.00}{
The development of next-generation solar cells is primarily driven by the search for novel materials suitable for their fabrication.
Particular attention is given to exploring the potential use of composites and nanoparticles
(e.g., Cu-La-based systems [2], cerium oxide, and copper indium sulfate [3]),
MAX phases [4,5],
and chalcogenides (such as GeSe, MoSe$_2$, Sb$_2$Se$_3$, and SnSe as hole transport materials [6]
or SnS$_2$ and WS$_2$ as electron transport materials [7]),
as well as metal silicides (e.g., FeSi$_2$ as an absorber layer [8])}
\end{mdframed}

on page 3, first paragraph from the top


\begin{mdframed}
Despite its one-dimensional modeling approach, SCAPS is extensively used for modeling various types
of solar cells [\textcolor[rgb]{1.00,0.07,0.00}{6,7,8},Joshi2024,Ravidas2024,Liu2024,You2023] in general
and for investigating the effects of defects on their performance [\textcolor[rgb]{1.00,0.07,0.00}{6,8,}AitAbdelkadir2023,Liang2024,SCAPSDefect3] in particular.
\end{mdframed}


on page 4, last paragraph

\begin{mdframed}
Impact of change of iron defects was investigated as a function of temperature from 290~K to 340~K,
base depth from 180~$\mu$m to 380~$\mu$m,
base doping level from $10^{15}$~cm$^{-3}$ to $10^{17}$~cm$^{-3}$,
and total impurity iron atom concentration from $10^{10}$~cm$^{-3}$ to $10^{14}$~cm$^{-3}$.
\textcolor[rgb]{1.00,0.07,0.00}{It is worth noting that investigating the effects of doping density, defect density, temperature, and active layer thickness on photovoltaic parameters is a well-established practice [6,7,8].}
For each illumination scenario, calculations were carried out with 11 different temperature values and 5 base depth values,
evenly distributed within the specified ranges.
\end{mdframed}

on page 7, 3rd paragraph from the bottom

\begin{mdframed}
However, the effect of $N_\mathrm{B}$ via $D_n$ and $L_n$ is significantly weaker than its influence through $E_\mathrm{F}$ and SCR depth variations.
Minor effects are expected as well due to both bandgap narrowing and changes in light absorption by free carriers with varying doping levels.
\textcolor[rgb]{1.00,0.07,0.00}{The impact of doping level on photovoltaic parameters has been studied in heterojunction [8],
thin-film [10],
and perovskite [6] solar cells.}
Moreover, it has been shown [10] that
doping concentration variations affect other barrier structure parameters, such as saturation current and ideality factor.
\end{mdframed}

on page 7, 4th paragraph from the top

\begin{mdframed}
- At low boron concentrations ($N_\mathrm{B}<10^{16}$~cm$^{-3}$),
the dependence of $\varepsilon F\!F$ on $N_\mathrm{Fe}$ is notably non-linear.
Within the used concentration range, two regions of decrease and two regions of increase in $\varepsilon F\!F$ are observed.
\textcolor[rgb]{1.00,0.07,0.00}{A similar relationship between the fill factor and defect concentration was observed in BaZrSe$_3$--based perovskite solar cells [6]};
\end{mdframed}




%\cite{Sultana2024,MasumMia2025,Rahman2024}
%
%
%\cite{Sultana2024,Behera2024,MasumMia2025,Rahman2024,AzzouzRached2024}
%         8           4             6          7            5

\noindent
\textcolor[rgb]{0.00,0.50,1.00}{\textbf{Comment~8.}}
\emph{How does the modeling in this study contribute to the design of processes for impurity control in silicon?}

\noindent
\textcolor[rgb]{0.51,0.00,0.00}{\textbf{Reply:}}




\noindent
\textcolor[rgb]{0.00,0.50,1.00}{\textbf{Comment~9.}}
\emph{State the main findings in the conclusions.}

\noindent
\textcolor[rgb]{0.51,0.00,0.00}{\textbf{Reply:}}


\bibliographystyle{model1-num-names}
\bibliography{olikh}


\end{document}

%Будь-ласка, покращ англійську в наступних реченнях, які я буду пропонувати. Зокрема, треба буде виправити граматичні та стилістичні помилки
%якщо можна, окрім виправленого речення, наводь інформацію про зміни, які зроблені 