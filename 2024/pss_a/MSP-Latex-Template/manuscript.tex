\documentclass{WileyMSP-template}

\begin{document}


\pagestyle{fancy}
\rhead{\includegraphics[width=2.5cm]{vch-logo.png}}


\title{Influence of illumination spectrum on dissociation kinetic of iron-boron pairs in silicon}

\maketitle



% Author: Please give full first and last names for authors and include * after the name of all corresponding authors

\author{Oleg Olikh*}
\author{Oleksandr Datsenko}
\author{Serhiy Kondratenko}

% Dedication

\dedication{}






% Affiliations: Please provide adacemic titles (Prof. or Dr.) for all authors where applicable, and include an institutional email address for all corresponding authors
\begin{affiliations}
Prof. O. Olikh, Dr. O. Datsenko, Prof. S. Kondratenko\\
Taras Shevchenko National University of Kyiv, 64/13, Volodymyrska Street, 01601, Kyiv, Ukraine\\
Email Address: olegolikh@knu.ua

%A. N. O. Author\\
%Address

\end{affiliations}


% Keywords: Please provide a minimum of three and a maximum of seven keywords, separated by commas

\keywords{silicon, iron-boron pairs, light-induced dissociation}



% Abstract should be written in the present tense and impersonal style (i.e., avoid we), and be at most 200 words long
\begin{abstract}

Please insert your abstract here

\end{abstract}

% Text: Please use section headings and subheadings as specified below. For communications, all section headings apart from Experimental Section should be removed
% Please make the first reference to a display item bold: \textbf{Figure 1}
% Do not abbreviate Figure, Equation, etc.; display items are always singular, i.e., Figure 1 and 2.
% Equations are always singular, i.e., Equation 1 and 2, and should be inserted using the {equation} environment, not as graphics
% Please do not use footnotes in the text, additional information can be added to the Reference list.


\section{Introduction}

Defects significantly impact semiconductor properties.
Although minimizing device dimensions to nanometers shifts some focus from extensive to point defects, 
physical properties still rely heavily on the presence and distribution of these irregularities. 
Hence, many strategies for enhancing semiconductor structures, including radiation and temperature treatments or certain fabrication conditions, strive to decrease the defect concentration or neutralize its effects \cite{Cai2023,Vobecky2021,Frascaroli2021}.
For instance, in the case of photovoltaic devices, we must understand and optimize the carrier properties tied to defects and impurities  \cite{Cai2023}.
Such controlled alteration methods of the defective subsystem have been generalized under the term ``defect engineering'' and are extremely important from a practical standpoint.

Successful defect engineering hinges on an in-depth understanding of defect properties.
Key factors are defect formation energy, transition energy levels, self-compensating effects, nonradiative recombination caused by defects, and the methods of reconstruction and diffusion  \cite{Cai2023}.
Considering the extraordinary diversity of possible intrinsic and impurity defects, complete information on all of them is lacking even for silicon, which is the most studied semiconductor.
Nevertheless, it must be noted that considerable data have been amassed on silicon, and have a solid understanding of some defects \cite{Juhl2018}.


For instance, a similar situation is observed for iron contamination, which is a common and often unavoidable contaminant in photovoltaic silicon materials \cite{Frascaroli2021,Sun2021}.
Зокрема відомо, що атоми заліза у кремнії знаходяться у міжвузіллях і у додатньо зарядженому стані є надзвичайно ефективними рекомбінаційними центрами.
У p-type Si at room temperature, iron atoms in Si : B are almost completely bound in complexes з легуючими атомами (B, Ga, Al, In).
Ця пара є bistable дефектом: основному стану відповідає конфігурація при якій атом заліза знаходиться on the first nearest Td interstitial site 
по відношенню до заміщуючого атому, тоді як в metastable конфігурації Fe  знаходиться в second Td site. 
Також добре відомі положення енергетичних рівнів, які відповідають залізу та його комплексам та відповідні перерізи захоплення носіїв заряду.
Серед згаданих пар акцептор-залізо найбільш дослідженим є комплекс FeB насамперед через те, що саме Si:B переважно використовується для створення різноманітних пристроїв,
наприклад сонячних елементів. 
Хоча мусимо зауважити, що все частіше у ролі акцепторної домішки використовується галій, що дозволяє, наприклад, зменшити ефекти light and elevated temperature-induced degradation \cite{Ning2022}.
It is known that
FeB pairs can be dissociated by illumination, minority carrier injection and thermal treatment at 200С FeBAssJAP2014.



FeB:PhysRevB49




%Загальновідомо, що дефекти є визначальними для властивостей напівпровідникових приладів.
%Навіть зменшення розмірів пристроїв до нанометрового діапазону лише частково зміщує акценти від протяжних дефектів до точкових дефектів,
%проте суттєва залежність фізичних властивостей від наявності та розташування порушень періодичності залишається.
%Як наслідок, більшість шляхів покращення властивостей різноманітних напівпровідникових структур, пов'язаних, наприклад,
%з опроміненням, температурними обробками чи вибором умов створення мають на меті зменшити концентрацію дефектів чи перевести їх у неактивний стан.
%Наприклад, для photovoltaic пристроїв, it is of paramount importance to comprehend and
%optimize the carrier properties associated with defects and impurities.
%Подібні методи контрольованої зміни дефектної підсистеми дістали узагальнену назву інженерія дефектів і вони є надзвичайно важливими
%з точки зору практичного використання.
%
%Цілком очевидно, що для успішного вирішення задач інженерії дефектів необхідно знати властивості самих дефектів.
%Ключовими з них є the defect formation energy, transition energy level, self-compensating effect,
%defect-induced nonradiative recombination, механізми перебудови та дифузії.
%Враховуючи надзвичайну різноманітність можливих власних та домішкових дефектів, повна інформація про всі з них відсутня навіть для кремнію,
%який is the most studied semiconductor.
%Хоча мусимо зауважити, що саме для Si накопичено чи не найбільше подібної інформації і деякі дефекти є достатньо вивченими.
%
%Наприклад, це стосується домішки заліза, яка is a common and often unavoidable contaminant  in photovoltaic silicon materials.







%\subsection{First Subsection}
%
%
%\subsubsection{First Sub Subsection}
%
%
%\threesubsection{First lowest-level subsection}


\section{Results and Discussion}
\subsection{First Subsection}


\section{Conclusion}

% Experimental section

\section{Experimental Section}
\threesubsection{First part of experimental section}\\
\threesubsection{Second part of experimental section}\\



\medskip
\textbf{Supporting Information} \par %Please delete the Suppporting Information statement if it is not applicable. Please supply Supporting Information in another file. Supporting information should not be provided in .tex format
Supporting Information is available from the Wiley Online Library or from the author.



% Acknowledgements
\medskip
\textbf{Acknowledgements} \par %delete if not applicable))
Please insert your acknowledgements here

% References
\medskip

% Use the following code if you wish to generate your bibliography with BibTeX;
% replace the string "MSP-template" below with the name(s) of
% the BibTeX data base(s) you want to use.
% The resulting bibliography-output (the content of the .bbl file)
% must be pasted back into this file before submission.
% Please also include your BibTeX data base file(s) in your submission
% so that we can re-run BibTeX if necessary.
%

\bibliographystyle{MSP}
\bibliography{olikh}

\textbf{References}\\

%1	((Journal articles)) a) A. B. Author 1, C. D. Author 2, Adv. Mater. 2006, 18, 1; b) A. Author 1, B. Author 2, Adv. Funct. Mater. 2006, 16, 1.\\
%2	((Work accepted)) A. B. Author 1, C. D. Author 2, Macromol. Rapid Commun., DOI: 10.1002/marc.DOI.\\
%3	((Books)) H. R. Allcock, Introduction to Materials Chemistry, Wiley, Hoboken, NJ, USA 2008.\\
%4	((Edited books or proceedings volumes)) J. W. Grate, G. C. Frye, in Sensors Update, Vol. 2 (Eds: H. Baltes, W. Göpel, J. Hesse), Wiley-VCH, Weinheim, Germany 1996, Ch. 2.\\
%5	((Presentation at a conference, proceeding not published)) Author, presented at Abbrev. Conf. Title, Location of Conference, Date of Conference ((Month, Year)).\\
%6	((Thesis)) Author, Degree Thesis, University (location if not obvious), Month, Year.\\
%7	((Patents)) a) A. B. Author 1, C. D. Author 2 (Company), Country Patent Number, Year; b) W. Lehmann, H. Rinke (Bayer AG) Ger. 838217, 1952.\\
%8	((Website)) Author, Short description or title, URL, accessed: Month, Year.\\
%9	…((Please include all authors, and do not use “et al.”))\\




% Figures/tables and captions
% Permission statements are required for all figures reproduced or adapted from previously published articles/sources. Please also ensure that all necessary permissions to reproduce images have been received
% Please remove these statements for original figures


\begin{figure}
  \includegraphics[width=\linewidth]{placeholder-image.png}
  \caption{Figure 1 caption goes here. Reproduced with permission.\textsuperscript{[Ref.]} Copyright Year, Publisher. }
  \label{fig:boat1}
\end{figure}

\begin{figure}
  \includegraphics[width=\linewidth]{placeholder-image.png}
  \caption{Figure 2 caption goes here. Reproduced with permission.\textsuperscript{[Ref.]} Copyright Year, Publisher.}
  \label{fig:boat1}
\end{figure}

\begin{figure}
  \includegraphics[width=\linewidth]{placeholder-image.png}
  \caption{Figure 3 caption goes here. Reproduced with permission.\textsuperscript{[Ref.]} Copyright Year, Publisher.}
  \label{fig:boat1}
\end{figure}

\begin{table}
 \caption{Table 1 caption}
  \begin{tabular}[htbp]{@{}lll@{}}
    \hline
    Description 1 & Description 2 & Description 3 \\
    \hline
    Row 1, Col 1  & Row 1, Col 2  & Row 1, Col 3  \\
    Row 2, Col 1  & Row 2, Col 2  & Row 2, Col 3  \\
    \hline
  \end{tabular}
\end{table}


% Please provide Biographies and photos for Essays, Feature Articles, Progress Reports, Reviews, and Perspectives for those authors who should be highlighted
% These should be at most 100 words long
% For other article types this section can be removed
% Photographs should be 40mm broad and 50 mm high

\begin{figure}
  \includegraphics{bio-placeholder.jpg}
  \caption*{Biography}
\end{figure}

\begin{figure}
  \includegraphics{bio-placeholder.jpg}
  \caption*{Biography}
\end{figure}

\begin{figure}
  \includegraphics{bio-placeholder.jpg}
  \caption*{Biography}
\end{figure}

\begin{figure}
  \includegraphics{bio-placeholder.jpg}
  \caption*{Biography}
\end{figure}


% Table of contents entry should be 50 - 60 words long
% Image should be 55 mm broad and 50 mm high or 110 mm broad and 20 mm high


\begin{figure}
\textbf{Table of Contents}\\
\medskip
  \includegraphics{toc-image.png}
  \medskip
  \caption*{ToC Entry}
\end{figure}


\end{document}
