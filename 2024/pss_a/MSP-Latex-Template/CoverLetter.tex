


\documentclass[preprint]{elsarticle}

\begin{document}


To:
physica status solidi (a) Editorial Board


Subject:
Article Submit

\vspace{5mm}
Dear Editors,

\vspace{3mm}
Enclosed with this letter you will find the electronic submission of manuscript entitled
``Influence of illumination spectrum on dissociation kinetic of iron-boron pairs in silicon''
by Oleg Olikh, Oleksandr Datsenko, and Serhiy Kondratenko.


It is widely recognized, that defects significantly impact semiconductor properties.
Therefore it's crucial to understand the parameters of defects and the mechanisms behind their alteration,
which holds significant practical importance.
Over several decades, extensive knowledge has been amassed regarding specific defects,
with the iron-boron pair in silicon being a well-studied example.
Therefore, the discovery of new findings or the resolution of contentious issues is particularly intriguing.

It has been established that the rate of light-induced dissociation of FeB pairs is influenced by integrated illumination intensity,
temperature, and the defect composition of the material.
Our investigation found that the spectral composition of illumination is an additional important factor
affecting dissociation efficiency.
Specifically, we demonstrated that increased photon energy leads to higher photo-dissociation efficiency.
The results indicate that the recombination-enhanced defect reaction is the more probable mechanism
at the second stage of light-induced dissociation, as opposed to iron ion recharge.
We are confident that this study, shedding new light on the long-studied FeB pair in silicon, will be of significant interest to your readers.



The corresponding abstract has been accepted for poster presentation at the 20th Conference on
Gettering and Defect Engineering in Semiconductor Technology (GADEST 2024, Reference ID 105868).

This is an original paper which has not been simultaneously submitted as a whole or in part anywhere else.
No elements of the work have been published in any form.
No conflict of interest exits in the submission of this manuscript.


We would  very much appreciate if you would consider the manuscript for publication in the GADEST 2024 special issue in \emph{physica status solidi (a)}.
%We appreciate your consideration of our manuscript, and we look forward to receiving comments from the reviewers.

%Possible reviewers are the following.
%
%\begin{itemize}
%  \item Yimin Zhang,
%Shenyang University of Chemical Technology,
%Equipment Reliability Institute, Shenyang 110142, China,
%ymzhang@mail.neu.edu.cn
%  \item Kang Li,
%University of Leeds,
%School of Electronic and Electrical Engineering,  Leeds, LS2 9JT, UK,
%k.li1@leeds.ac.uk
%  \item Rebecca Saive,
%University of Twente Institute for Nanotechnology, Enschede 7522 NB, The Netherlands,
%r.saive@utwente.nl
%  \item Ripon Chakrabortty,
%University of New South Wales,
%School of Engineering and IT, Canberra, Australia,
%r.chakrabortty@adfa.edu.au
%%  \item Belen Arredondo,
%%Universidad Rey Juan Carlos
%%Área de Tecnología Electrónica, C/ Tulipán s/n, 28933 Móstoles, Spain,
%%belen.arredondo@urjc.es
%\end{itemize}


\vspace{3mm}

Sincerely yours,

Oleg Olikh and co-authors.


Taras Shevchenko National University of Kyiv


Kyiv 01601, Ukraine

E-mail: olegolikh@knu.ua


%Dear Editors
%It is more than 12 weeks since I submitted our manuscript (No: ) for possible publication in your journal. I have not yet received a reply and am wondering whether you have reached a decision. I should appreciated your letting me know what you have decided as soon as possible.








\end{document}

