

%\documentclass[aip,reprint]{revtex4-1}
%\documentclass[aip,jap,preprint]{revtex4-1}
%\documentclass[a4paper,fleqn]{cas-sc}
\documentclass{WileyMSP-template}

%\usepackage[numbers]{natbib}

\usepackage{color}

\begin{document}


Dear Editor,

We like to express our appreciation to the reviewers for their comments.
We are resubmitting the revised version of the paper number MSB--D--24--00484.
We have studied the comments of the reviewer carefully, and have changed the text according to the comments they
have listed.
%The location of revisions is pointed by blue color in ``MarkedManuscript.pdf''.
Below we refer to each of the reviewer’s comments.


\subsection*{Response to Reviewer \#1 }

\noindent
\textcolor[rgb]{0.00,0.50,1.00}{\textbf{Comment~1.}}
\emph{The author should add the aim of the research in the introduction part.}

\noindent
\textcolor[rgb]{0.51,0.00,0.00}{\textbf{Reply:}}
This study aimed to compare the effectiveness of meta-heuristic algorithms in extracting
the parameters of new-generation solar cells from the S-shaped \emph{IV} curves
and to determine the most suitable ones for addressing this optimization problem.

The aim of the research was added in the introduction part
(page~2, paragraph~3 (last paragraph in the Introduction), line~1-2).

%
%\vspace{1cm}
%\noindent
%\textcolor[rgb]{0.00,0.50,1.00}{\textbf{Comment~2.}}
%\emph{Figure captions are poorly written. Pay attention to them, besides i couldn't find Fig.~3 in the texts.}
%
%\noindent
%\textcolor[rgb]{0.51,0.00,0.00}{\textbf{Reply:}}
%Figure captions were revised.
%
%The Fig.~3 is referenced on page~5 (last line in Subsection~2.2.2) and on page~18 (first line in Subsection~3.3.2)
%in the revised manuscript.
%Fig.~3 located at page~5.
%
%
%\vspace{1cm}
%\noindent
%\textcolor[rgb]{0.00,0.50,1.00}{\textbf{Comment~3.}}
%\emph{Figure~4, 5 and 6 are hard to understand for authors, try to make it clear.}
%
%\noindent
%\textcolor[rgb]{0.51,0.00,0.00}{\textbf{Reply:}}
%Parts of Figure~4 were separated to clarify in the revised manuscript.
%Overall, it is a standard box plot with added mean values.
%
%Figure~5 has been divided into several parts, and a more detailed legend has been added.
%Additionally, the figure caption has been revised.
%
%
%Figure~6 remains mostly unchanged; however, a more detailed caption has been added.
%
%
%We sincerely hope that these modifications have made the figures more understandable.
%
%\vspace{1cm}
%\noindent
%\textcolor[rgb]{0.00,0.50,1.00}{\textbf{Comment~4.}}
%\emph{Details about equivalent circuit model of a solar cell is insufficient.
%How it is useful and novel compared with one diode model.}
%
%
%
%\noindent
%\textcolor[rgb]{0.51,0.00,0.00}{\textbf{Reply:}}
%Our study focused on the opposed two--diode model, proposed by De Castro \cite{Castro2010}.
%This model represents a significant advancement
%as it successfully reproduces the S-shaped kink in the power-producing fourth quadrant
%of the illuminated \emph{IV} characteristics.
%However, it encounters difficulties in accurately describing the \emph{IV} curve beyond the open-circuit point in the first quadrant.
%Despite this drawback, the model is gaining attention for deriving analytical solutions
%of equivalent circuits \cite{Yu2019a}
%and is widely used to describe experimental IV curves of SCs with different structures
%\cite{CastroUseBook,Pillai2017,Arredondo2018,delPozo2012,BrenesBadilla2018,Tada2015Organic,Makha2018,CastroUsePerovskitIonikLiquid,CastroUsePerovskitFullerene}.
%In particular, these include polymer \cite{Tada2015Organic}
%and polymer/fullerene \cite{delPozo2012} bulk heterojunction photocells,
%ternary organic solar cells \cite{Makha2018},
%and other types of organic structures \cite{Pillai2017,Arredondo2018},
%perovskite solar cells with fullerene transport layer and carbon nanotube electrode \cite{CastroUsePerovskitFullerene},
%and perovskite solar cells with ionic liquid gating \cite{CastroUsePerovskitIonikLiquid}.
%The popularity of the De Castro model is also because, in experiments,
%\emph{IV} curves are typically measured only in the fourth quadrant from short-circuit current to open-circuit voltage.
%Therefore, our selection of the two-diode model is based on its universality and widespread applicability.
%
%It can be seen from the model structure, as shown in Fig.~1 (page 3), that some elements are identical to SDM.
%It is a current source accompanied by a diode D1, a shunt resistor $R_\mathrm{p1}$ to represent the leakage current,
%and a series resistor $R_\mathrm{s}$ to account for the losses associated with the load current.
%However, SDM fails to describe the S-shaped kink, which requires additional elements.
%As a result, a second diode  (D2) and a second parallel resistor ($R_\mathrm{p2}$) are used.
%D2 is placed opposite D1 and represents the effect of traps at the active layer/cathode interface \cite{Castro2010}.
%
%
%The relevant information can be found on page 3, paragraphs 1-2.
%
%
%\vspace{1cm}
%\noindent
%\textcolor[rgb]{0.00,0.50,1.00}{\textbf{Comment~5.}}
%\emph{The artwork quality in this manuscript could be improved thoroughly.
%Figures should be placed in a uniform location with a uniform font, size, figures are unclear.}
%
%\noindent
%\textcolor[rgb]{0.51,0.00,0.00}{\textbf{Reply:}}
%The same font is consistently utilized across all figures in the revised manuscript.
%We endeavored to ensure that all figures are positioned uniformly,
%albeit our capacity is somewhat constrained
%by the LaTeX template utilized for manuscript preparation
%(the latest Elsevier article class).
%Likewise, we lack control over the font used in figure captions.
%The dimensions of most figures are configured to occupy half the width of a text column.
%
%
%\vspace{1cm}
%\noindent
%\textcolor[rgb]{0.00,0.50,1.00}{\textbf{Comment~6.}}
%\emph{Table~3 need to modify, text size and structure should be similar in whole manuscript.}
%
%\noindent
%\textcolor[rgb]{0.51,0.00,0.00}{\textbf{Reply:}}
%The manuscript was prepared in LaTeX using the latest Elsevier article class.
%No additional commands, especially for resizing, were used in preparing Table~3.
%In our opinion, the difference in text between the tables and the main text is due to the article class's characteristics.
%
%%
%%You are recommended to use the latest Elsevier article class to prepare your manuscript and BibTeX to generate your bibliography.
%%Our Guidelines has full details.
%
%
%\vspace{1cm}
%\noindent
%\textcolor[rgb]{0.00,0.50,1.00}{\textbf{Comment~7.}}
%\emph{Please rectify all the English language mistakes.}
%
%\noindent
%\textcolor[rgb]{0.51,0.00,0.00}{\textbf{Reply:}}
%We are sorry for our English.
%The text has been revised by a bilingual speaker, and we hope for the language improvement.
%
%
%\vspace{1cm}
%\noindent
%\textcolor[rgb]{0.00,0.50,1.00}{\textbf{Comment~8.}}
%\emph{Please check the grammar.}
%
%\noindent
%\textcolor[rgb]{0.51,0.00,0.00}{\textbf{Reply:}}
%The text was revised.
%We are really hoping to see a significant decrease in the number of grammatical errors.
%
%\vspace{1cm}
%\subsection*{Response to Reviewer \#4 }
%
%
%\noindent
%\textcolor[rgb]{0.00,0.50,1.00}{\textbf{Comment~1.}}
%\emph{The paper would benefit from more detailed descriptions of the individual meta-heuristic algorithms used,
%especially the lesser-known STLBO and ADELI.
%Explaining their mechanisms, strengths, and why they are particularly suited to solving the parameter estimation problem
%for S-shaped IV curves would help readers understand their effective application in this
%context added recent LR}
%
%\emph{
%``1. Zhu, J., Chaturvedi, R., Fouad, Y., Albaijan, I., Juraev, N., Alzubaidi, L. H.,... Garalleh, H. A. L. (2024).
%A numerical modeling of battery thermal management system using nano-enhanced phase
%change material in hot climate conditions. Case Studies in Thermal Engineering, 58, 104372.
%doi: https://doi.org/10.1016/j.csite.2024.104372
%}
%
%\emph{
%2. Zhang, X., Tang, Y., Zhang, F., \& Lee, C. (2016).
%A Novel Aluminum-Graphite Dual-Ion Battery. Advanced energy materials, 6(11), 1502588.
%doi: 10.1002/aenm.201502588
%}
%
%\emph{
%3. Wang, M., Jiang, C., Zhang, S., Song, X., Tang, Y.,... Cheng, H. (2018).
%Reversible calcium alloying enables a practical room-temperature rechargeable
%calcium-ion battery with a high discharge voltage. Nature chemistry, 10(6), 667-672.
%doi: 10.1038/s41557-018-0045-4
%}
%
%\emph{
%4. Su, Y., Shang, J., Liu, X., Li, J., Pan, Q.,... Tang, Y. (2024).
%Constructing p-p Superposition Effect of Tetralithium Naphthalenetetracarboxylate
%with Electron Delocalization for Robust Dual-ion Batteries.
%Angewandte Chemie International Edition, e202403775.
%doi: https://doi.org/10.1002/anie.202403775
%}
%
%\emph{
%5. Zhang, M., Zhang, W., Zhang, F., Lee, C., \& Tang, Y. (2024).
%Anion-hosting cathodes for current and late-stage dual-ion batteries.
%Science China Chemistry.
%doi: 10.1007/s11426-023-1957-3
%}
%
%\emph{
%6. Liu, Q., Liu, L., Zheng, Y., Li, M., Ding, B., Diao, X.,... Tang, Y. (2024).
%On-demand engineerable visible spectrum by fine control of electrochemical reactions.
%National Science Review, 11(3), nwad323.
%doi: https://doi.org/10.1093/nsr/nwad323
%}
%
%\emph{
%7. Zhu, C. (2023).
%Optimizing and using AI to study of the cross-section of finned
%tubes for nanofluid-conveying in solar panel cooling with phase change materials.
%Engineering Analysis with Boundary Elements, 157, 71-81.
%doi: https://doi.org/10.1016/j.enganabound.2023.08.018''.
%}
%
%\noindent
%\textcolor[rgb]{0.51,0.00,0.00}{\textbf{Reply:}}
%
%ADELI improved version of DE incorporates three main elements:
%local search using Lagrange interpolation, self-adaptive DE control parameter settings,
%and an adaptive mutational strategy \cite{ADELI}.
%The first element involves interpolating three potential solutions using a
%polynomial function to calculate the local minimum value.
%The self-adaptive control parameter settings include randomly altering the
%scaling factor and crossover rate values in each iteration.
%The adaptive mutational strategy determines the probability of
%employing Lagrange interpolation in each generation based on the best fitness function value.
%These incorporations aim to enhance exploitation capability and speed up the convergence.
%
%This answer is incorporated in the text on page~6, paragraph~4.
%
%In STLBO, the teacher phase is redefined and simplified, while the learner phase remains unchanged \cite{STLBO}.
%In the redefined teacher phase, the mutation of potential solutions is possible,
%with the mutation probability decreasing as the iteration number increases.
%During the early stages, a higher mutation probability helps explore a larger solution space
%and approach the optimum quickly.
%However, in the latter stages of optimization, when the teacher (best solution) is near the global optimum,
%a lower mutation probability serves as a fine-tuning mechanism
%to enhance local search capability, proving to be an effective strategy.
%To enrich the mutation behavior, a chaotic sequence is introduced to generate values for the mutation parameters.
%A chaotic sequence is a deterministic, random-like process found in nonlinear dynamic systems,
%which is non-periodic, non-converging, and bounded \cite{May1976}.
%Additionally, the elite strategy replaces the worst solutions in the current population with new solutions based on objective function values.
%
%This answer is incorporated in the text on page~7, paragraph~2.
%
%Unfortunately, providing a detailed description of each algorithm requires too much space.
%However, the paper includes references that allow obtaining additional information.
%
%Furthermore, some speculation about possible modification of meta--heuristic algorithms
%to successful solving the parameter estimation problem for S-shaped \emph{IV} curves was added
%(page 22, last paragraph)
%
%We would also like to express our gratitude to the Reviewer for providing exceptionally insightful references.
%We hope the Reviewer will not object to their inclusion in the revised manuscript (references 1, 2, 3, 91, 92).
%
%Additionally, the idea of employing meta-heuristic algorithms during the optimization of dual-ion batteries appears to be particularly intriguing.
%
%
%
%\vspace{1cm}
%\noindent
%\textcolor[rgb]{0.00,0.50,1.00}{\textbf{Comment~2.}}
%\emph{While the use of synthetic IV curves is mentioned, the process for generating
%these curves could be elaborated upon.
%Details such as the assumptions made, the variability in parameters used to generate the curves,
%and how these synthetic models compare to actual solar cell data would greatly enhance the transparency and reproducibility of the results.}
%
%\noindent
%\textcolor[rgb]{0.51,0.00,0.00}{\textbf{Reply:}}
%In the first part of the study, the performance of meta--heuristic algorithms for parameter estimation was evaluated using a single \emph{IV} curve.
%This curve represents the experimental data of bulk heterojunction photocells prepared
%using a composite of $p$--DTS(FBTTh$_2$)$_2$ and neat C$_{70}$ \cite{Tada2015Organic}.
%
%This answer is incorporated in the text on page~3, last paragraph~1, line~1,
%and page~4, first paragraph~1, line~1.
%
%In the second part, we simulated a set of \emph{IV} characteristics.
%These curves correspond to the temperature range from 260~K to 350~K.
%In the simulation process, we considered the temperature dependencies of the parameters.
%We based our approach on known physical mechanisms of current flow in NG-SCs and
%used the reported temperature dependencies of saturation current, ideality factor, shunt resistance, and series resistance.
%However, the main focus was on achieving a diversity of parameter ratios rather
%than attempting to precisely replicate real--life PV converters of a specific type.
%Furthermore, an S--shaped \emph{IV} curve is observed in various types of solar cells,
%and diverse charge transport mechanisms significantly complicate the selection of a single possible temperature dependence for each of the eight model parameters.
%The specific temperature dependencies and parameter values utilized are detailed in Subsection~2.2.2.
%
%This answer is incorporated in the text on page~4, paragraph~1, line~3-5;
%page~4, last paragraph, line~2-5;
%and page~5, paragraph~1, line~1-3.
%
%
%The \emph{IV} curves ware simulated using Eqs.~(1)--(2) over a voltage range of 0-0.8~V (in ‘‘Single--\emph{IV} case’’) or
%0-1.0~V (in ‘‘\emph{IV}--set case’’) with a 10~mV step.
%
%This answer is incorporated in the text on page~4, paragraph~4, line~1-2,
% and page~5, paragraph~6, line~3-4
%(last paragraphs of Subsection~2.2.1 and Subsection~2.2.2).
%
%
%\vspace{1cm}
%\noindent
%\textcolor[rgb]{0.00,0.50,1.00}{\textbf{Comment~3.}}
%\emph{The conclusion hints at future research into the impact of noisy data on algorithm performance.
%It would strengthen the current study to include preliminary investigations or simulations on how noise influences the effectiveness of the meta-heuristic algorithms.
%Even basic insights could provide significant value in understanding the robustness of these algorithms under practical conditions.}
%
%\noindent
%\textcolor[rgb]{0.51,0.00,0.00}{\textbf{Reply:}}
%The results of some preliminary simulations on how noise influences the effectiveness of
%the meta-heuristic algorithm are presented in the Appendix~A (page 24) of revised manuscript.
%
%
%\vspace{1cm}
%\noindent
%\textcolor[rgb]{0.00,0.50,1.00}{\textbf{Comment~4.}}
%\emph{The paper mentions the use of nonparametric statistical methods but could include more comprehensive details about these analyses.
%For instance, discussing why particular tests were chosen, the statistical significance levels,
% and the interpretation of these tests would help in understanding the comparative analysis more deeply.}
%
%\noindent
%\textcolor[rgb]{0.51,0.00,0.00}{\textbf{Reply:}}
%
%Wilcoxon test is used to assess whether there are statistically significant differences between pairs of algorithms.
%Meanwhile, the Friedman, Friedman Aligned, and Quade tests are employed when it's necessary to compare three or more related groups of results (algorithms).
%Friedman test evaluates whether there are statistically significant differences between the medians of the ranks of these algorithms.
%Friedman Aligned Ranks test addresses the issue of rank correlation in the original Friedman test, providing more precise results.
%Finally, the Quade test helps account for the effects of observed factors, such as random variations,
%to more accurately determine the statistical differences between groups.
%
%The main drawback of the Friedman, Friedman Aligned, and Quade tests is that
%they can only detect significant differences over the whole set of multiple comparisons,
%making it difficult to establish proper comparisons between specific algorithms \cite{Derrac2011}.
%To address these issues, it is necessary to employ post-hoc procedures.
%Post-hoc methods are applied after the initial analysis and allow for controlling the overall error rate
%when comparing multiple algorithms, thereby reducing the likelihood of randomly identifying statistically significant differences.
%
%$1\times N$ designs help determine if there are statistically significant differences between one algorithm
%(the control algorithm) and each of the other algorithms.
%Multidimensional comparisons $N\times N$ designs involve analyzing statistical differences between all possible pairs of algorithms.
%Typically, different post-hoc procedures are used for $1\times N$ and $N\times N$ comparisons.
%Description of all the used post-hoc procedures can be found in Derrac \emph{et al.} \cite{Derrac2011}.
%
%All mentioned methods are standard for comparing metaheuristic algorithms \cite{Derrac2011}.
%Their comprehensive application enables making the most well-founded conclusions.
%In our case, only applying the entire set of tests and post--hoc methods
%has enabled us to detect slight differences in the accuracy of model parameter evaluation by different algorithms.
%
%In most cases, the study used the widely accepted significance level of 0.05.
%
%This answer is incorporated in the text on page~8, paragraph~8-11 (last 4 paragraphs on the page).
%
%
%
%\vspace{1cm}
%\noindent
%\textcolor[rgb]{0.00,0.50,1.00}{\textbf{Comment~5.}}
%\emph{ Adding visual representations such as scatter plots, error distributions,
%or convergence graphs of the algorithms' performance over iterations could provide intuitive insights into their behavior.
%This would also help in visually comparing the efficiency and accuracy of the algorithms beyond the tabular or textual data presented.}
%
%\noindent
%\textcolor[rgb]{0.51,0.00,0.00}{\textbf{Reply:}}
%
%The convergence graphs for  algorithms were added (see Fig.S1 in the supplementary materials).
%Information about the existence of convergence graphs has been added to the revised manuscript
%(page~9, paragraph~1, line~7-8).
%
%Besides the scatter plots for parameter determination results are represented in Figs.~S15-S22 (supplementary materials).
%Information about the existence of scatter plots has been added to the revised manuscript
%(page~18, paragraph~2, line~8-9).
%
%Simultaneously, the interquartile range (IQR) for estimated parameter values deals with error distributions.
%IQR data is represented in Figs.~4, 9, and S6--S14.
%Therefore, we believe that the separate error distribution graphs are somewhat redundant.
%
%
%\vspace{1cm}
%\noindent
%\textcolor[rgb]{0.00,0.50,1.00}{\textbf{Comment~6.}}
%\emph{ the paper could discuss the practical implications of implementing these algorithms
%in real-world solar cell testing and production environments.
%Suggestions on how to integrate these algorithms into existing systems, the expected improvements
%in parameter estimation accuracy, and potential challenges in deployment would be beneficial for practitioners in the field.}
%
%\noindent
%\textcolor[rgb]{0.51,0.00,0.00}{\textbf{Reply:}}
%When analyzing the significance of the results, it is essential to consider the following points.
%The \emph{IV} characteristics are one of the key measurements  in the analysis
%of solar cells in both research and industrial mass production.
%Typically, \emph{IV} curves are analyzed using a specific model,
%and the model parameters are closely related to the internal physical mechanisms acting within the solar cell.
%Therefore, their efficient and sufficiently accurate extraction is important
%as an analysis tool for understanding the processes involved.
%Thus, the algorithms selected in our study can significantly aid in modeling NG-SCs,
%aiming to enhance the understanding of their internal processes.
%Specifically, this includes the underlying causes of the S--shaped kink
%and the peculiarities of PV conversion in advanced solar cells, such as those employing phase--change materials \cite{Zhu2024,Zhu2023}.
%
%Precise knowledge of the model parameter values is crucial for many practical applications.
%For example, they are used in the simulation and emulation of PV systems,
%for quality control of PV cells during manufacturing,
%and in the study of SC degradation \cite{Chin2019}.
%So, for a successful implementation of a PV system, the availability of an accurate,
%fast, and reliable computer simulation tool is indispensable.
%The algorithm for model parameter extraction can be a valuable tool for developing a
%reliable computational engine for a PV simulator \cite{Chin2017}.
%Diagnosing PV degradation can be achieved by monitoring and comparing the SC parameters with those in their initial states.
%In this case, it is impossible to overstate the importance of the ability to parameter extraction with high precision.
%Finally, optimization efforts and quantitative studies to assess the capabilities of a particular technology greatly benefit from the correct extraction of the model parameters under various test conditions \cite{OrtizConde2006}.
%Implementing selected algorithms in real-world solar cell testing and production environments allows for
%enhanced characterization precision (due to precise determination of model parameters),
%production process optimization (more accurate SC characterization leads to minimizing defects and increasing efficiency),
%quality improvement (the algorithm helps identify shortcomings at early production stages,
%thereby enhancing overall product quality),
%testing acceleration (the automated model parameter determination can shorten the time
%required for testing each element, enabling increased production volumes),
%and cost reduction (optimizing the testing process and reducing the number of defective elements decreases production costs).
%Moreover, potential challenges in deployment are minimal, given the algorithms'
%low resource requirements and compatibility with modern computers.
%There is no need for personnel training, as understanding the intricacies of the algorithms is unnecessary for their use.
%Additionally, constant maintenance, updates, and bug fixes are not required.
%The only requirements for implementation involve developing software interfaces to integrate the new algorithms with existing control and monitoring systems and ensuring the market's readiness to adapt established processes to these changes.
%
%The information was added to the revised manuscript (page~23, paragraphs 1 and 2).
%
%Unfortunately, we cannot offer insights into expected improvements in parameter estimation accuracy
%because, to our knowledge,
%the studies dedicated to examining the accuracy of parameter determination from S-shaped current-voltage curves are currently absent.


\bibliographystyle{MSP}
\bibliography{olikh}


\end{document}

%Будь-ласка, покращ англійську в наступних реченнях, які я буду пропонувати. Зокрема, треба буде виправити граматичні та стилістичні помилки
%якщо можна, окрім виправленого речення, наводь інформацію про зміни, які зроблені 