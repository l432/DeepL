

%\documentclass[aip,reprint]{revtex4-1}
%\documentclass[aip,jap,preprint]{revtex4-1}
%\documentclass[a4paper,fleqn]{cas-sc}
\documentclass{WileyMSP-template}


\usepackage{hyperref}

\usepackage{color}

\begin{document}


Dear Editor,

We like to express our appreciation to the reviewers for their comments.
We are resubmitting the revised version of the paper number pssa.202400351.
We have studied the comments of the reviewer carefully, and have changed the text according to the comments they
have listed.
The location of revisions is pointed by blue in ``RevisedMarkedManuscript.pdf''.
Below we refer to each of the reviewer’s comments.


\subsection*{Response to Reviewer \#1 }

\noindent
\textcolor[rgb]{0.00,0.50,1.00}{\textbf{Comment~1.}}
\emph{The authors could consider mentioning where does the iron come to the samples?
Is it in the wafers to begin with after the crystal growth or is some intentional iron contamination done prior to device processing?
In my opinion this information would be worth mentioning in the manuscript.}

\noindent
\textcolor[rgb]{0.51,0.00,0.00}{\textbf{Reply:}}
The technological process of solar cells (SCs) manufacturing from Cz-p-Si wafers included the formation of separating and isotype barriers
($n^+$-$p$ and $p$-$p^+$ junctions) by diffusion of phosphorus (POCl$_3$) and boron (BCl$_3$) from the gas phase, respectively;
thermal oxidation;
thermal annealing;
photolithography;
etching the dividing groove;
chemical treatments;
magnetron sputtering of aluminum contacts to the front and back sides.

It has been found that some SC lots have significantly worse parameters compared to typical solar cells for this technological process.
In particular, the photoconversion efficiency was almost halved.
The analysis showed that the reason for such a deterioration in the SC parameters is a sharp drop in the diffusion length of minority charge carriers (electrons) $L_n$ in the SC base.
Additional experiments with thermal annealing at temperatures of 200$^\circ$C and 90$^\circ$C
(the procedure is described by Tayyib~\emph{et al.}\cite{TAYYIB201221})
showed that the decrease in $L_n$ value is caused by iron impurities available in the SC base at concentrations up to $4\cdot10^{13}$~cm$^{-3}$.
It has also been found that the source of iron impurity is insufficiently pure chemicals that were used for chemical treatments in the technological process, obtained from another supplier.
These reagents were the source of contamination in the process of manufacturing experimental SC samples.



At the same time, such detailed information about the samples has already been provided in a previous paper \cite{Olikh2021JAP}.
Therefore, in revised manuscript, we briefly describe the causes of contamination and provide links
to a more comprehensive description --- see page~9, paragraphs~5.


\vspace{1cm}
\noindent
\textcolor[rgb]{0.00,0.50,1.00}{\textbf{Comment~2.}}
\emph{What was the size of the samples and solar cells used in the experiments?
Was the whole sample/cell surface illuminated during the experiments or just part of it locally?}

\noindent
\textcolor[rgb]{0.51,0.00,0.00}{\textbf{Reply:}}
The area of the samples used in the study was $1\times1$~cm$^2$.
The entire surface of the solar cell was illuminated during the experiments.

This answer is incorporated in the text on page~9, paragraph~6.


\vspace{1cm}
\noindent
\textcolor[rgb]{0.00,0.50,1.00}{\textbf{Comment~3.}}
\emph{Based on the reported sheet resistance values,
the emitter and back surface field diffusions are quite heavy and could thus act as strong gettering sinks for iron during the device processing.
Could the authors comment on this?
This could lead to rather uneven distribution of iron in the samples and also the iron being in different forms in different locations.
Does this affect the results and if yes, how?
(Perhaps here the fact that the light used in the IV-measurements was monochromatic with a wavelength of 940 nm plays a role.)}

\noindent
\textcolor[rgb]{0.51,0.00,0.00}{\textbf{Reply:}}
Really, phosphorus diffusion \cite{FeB:Vahanissi} and implantation \cite{LaineIEEEPV2016,FeBKin2019} gettering
are well--known methods to relocate iron from the bulk to the emitter, where it is less harmful.
Therefore, due to creating barriers (see Reply to Comment~1), the concentration of iron near the surfaces of the solar cell could have increased.
However, gettered iron is not recombination-active like
Fe$_i$ or FeB, and even high amounts of gettered iron do not increase emitter recombination \cite{FeB:Vahanissi}.

Thus, only the fraction of iron remaining in the bulk could respond to illumination in our experiments.
Considering that other high--temperature process operations (thermal oxidation, thermal annealing) were performed after the formation of barriers,
we believe that the bulk iron is evenly distributed.
Previous studies have not reported uneven distribution of iron due to phosphorus-induced gettering,
and the possibility of uneven distribution was not considered in the analysis of the light-induced kinetics of iron-boron pairs \cite{LaineIEEEPV2016,FeBKin2019}.

Additionally, for light with a wavelength of 940~nm, the absorption depth is approximately 50~$\mu$m at 340~K \cite{Green2022}.
Therefore, using monochromatic light as a probe tool primarily allowed us to focus on processes deep within the solar cell base.


\vspace{1cm}
\noindent
\textcolor[rgb]{0.00,0.50,1.00}{\textbf{Comment~4.}}
\emph{Keeping in mind the cell structure including e.g. strong emitter and BSF diffusions,
the reported value of tau\_other (2.2 ms) seems quite high.
Could the authors comment on this?
Were there any uncontaminated reference cells included in the experiments?
(These would perhaps be good references also elsewhere in the manuscript.)
Was their lifetime characterized to back up this claim and the resulting validity of tau\_other being much much higher than tau\_feb?
}


\noindent
\textcolor[rgb]{0.51,0.00,0.00}{\textbf{Reply:}}
First and foremost, we would like to apologize for the inaccuracy. 
The cited value of 2.2~$\mu$s corresponds to the lifetime associated with Shockley-Read–Hall recombination at Fe-related defects. 
The information in the revised manuscript has been rephrased for clarity --- see page~4, paragraph~2.

During fitting the time dependencies of short-circuit current,
the following expression was used to estimate the minority carrier lifetime in the base $\tau$:
\begin{equation}
\label{eqTau}
\frac{1}{\tau(t)}=\frac{1}{\tau_i}+\frac{1}{\tau_{SRH}^{\mathrm{Fe_i}}(t)}
+\frac{1}{\tau_{SRH}^\mathrm{FeB}(t)}+\frac{1}{\tau_{other}}\,,
\end{equation}
where
$\tau_i$ is the lifetime associated with intrinsic recombination,
$\tau_{SRH}^{\mathrm{Fe_i}}$ and $\tau_{SRH}^\mathrm{FeB}$ are related to the recombinations at interstitial iron atoms Fe$_i$ and at FeB pairs, accordingly;
$\tau_{other}$ describes further recombination channels
(other impurities, lattice defects, surface recombination).

It turned out that the final term in Eq.~(\ref{eqTau}) is significantly smaller than the preceding ones.
This fact gave grounds to assert that $\tau_\mathrm{other}$ significantly exceeds $\tau_\mathrm{FeB}$.

For other samples of the same series with a lower iron concentration (and a lower $\tau_\mathrm{other}/\tau_\mathrm{FeB}$ ratio),
the $\tau_\mathrm{other}$  value was estimated more accurately.
Thus, the measurements showed that for solar cells with
$N_\mathrm{Fe,tot}=(1.8\times10^{11}-1.4\times10^{12})$~cm$^{-3}$,
the bulk lifetime ranged from 20 to 300~$\mu$s,
and no correlation was observed between the values of $N_\mathrm{Fe,tot}$ and $\tau_\mathrm{other}$.

The last part of answer is incorporated in the text on page~4, paragraph~2.




\vspace{1cm}
\noindent
\textcolor[rgb]{0.00,0.50,1.00}{\textbf{Comment~5.}}
\emph{
Was the iron concentration in the samples/cells verified with some of the well-established methods
for iron concentration determination such as surface photovoltage method (SPV)?
That would provide also simultaneously a good reference value for the diffusion length value determined in the manuscript.
}

\noindent
\textcolor[rgb]{0.51,0.00,0.00}{\textbf{Reply:}}
We must acknowledge that additional verification of the iron concentration was not carried out for the solar cell used in this work.
However, the approach to determining the iron concentration from the short-circuit current kinetics
was initially validated by assessing the diffusion length \cite{FeB_Zong}.
The results obtained by different methods were well-agreed, as described in previous papers \cite{Olikh2021JAP,Olikh2022:JMatSci}.

It should be noted that even some inaccuracy in iron concentration evaluation could have affected only
the absolute values of the coefficient $K$, which are reported \cite{FeBLight2,FeBAssJAP2014,FeBKin2019} with a big enough scatter.
The primary study's conclusion (the dependence of the dissociation rate and $K$ on the illumination spectrum) remains unchanged.




\vspace{1cm}
\noindent
\textcolor[rgb]{0.00,0.50,1.00}{\textbf{Smaller Comment~1.}}
\emph{The authors could consider moving the experimental from the end to the second section in the paper.
There is a lot of crucial information in the experimental part that is needed to understand the results properly.
Therefore, it could be very useful for the readers to read the experimental part first before going into the results section.}

\noindent
\textcolor[rgb]{0.51,0.00,0.00}{\textbf{Reply:}}
Overall, we also support the opinion that the ability to read the experimental part first
before going into the results section is useful, customary, and fully justified.
However, in this case, we must adhere to the
%‘‘Guide for Authors’’
%(\url{https://onlinelibrary.wiley.com/page/journal/18626319/homepage/author-guidelines})
\href{https://onlinelibrary.wiley.com/page/journal/18626319/homepage/author-guidelines}{‘‘Guide for Authors’’}
and the
%\LaTeX template
%(\url{https://onlinelibrary.wiley.com/pb-assets/assets/vch/msp/LaTeX-template-1593698612150.zip}),
\href{https://onlinelibrary.wiley.com/pb-assets/assets/vch/msp/LaTeX-template-1593698612150.zip}{\LaTeX~template}
which state that the order of the sections must be as follows:
‘‘Title -- Author(s) -- (Dedication) -- Affiliation(s), -- Keywords --
Abstract -- Main text -- (Experimental/Methods Section) -- Acknowledgements -- References -- (Biographies) -- Table of Contents text
 [Sections in brackets are only present in certain article types]’’.


\vspace{1cm}
\noindent
\textcolor[rgb]{0.00,0.50,1.00}{\textbf{Smaller Comment~2.}}
\emph{The paper would benefit from one round of language and typo checks.
There were quite a lot of such problems all around the paper.
Here are just some examples:}

\emph{Title: Should it be dissociation kinetics and not kinetic?}

\emph{Page 1, line 42:
"and have a solid understanding of some defects".
Something wrong in this sentence,
should it be e.g. "and that there is a solid understanding of some defects."}

\emph{
Page 4, lin49: "the lifetime associated with…(about 2.2 µm)" Unit wrong, should be ms?}

\emph{
Page 6, line 29: "These behaviour…" Should be This behaviour?}

\emph{
Figure 5 caption: "carrier generate rate"}

\emph{
etc.}

\noindent
\textcolor[rgb]{0.51,0.00,0.00}{\textbf{Reply:}}
We apologize for any language errors.
A bilingual speaker has revised the text, and we hope it shows improvement.

The reviewer is completely correct and
\begin{itemize}
  \item title revised to ‘‘Influence of Illumination Spectrum on Dissociation Kinetics of Iron-boron Pairs in Silicon
  \item sentence revised to ‘‘Nevertheless, it must be noted that considerable data have been amassed on silicon, resulting in a solid understanding of certain defects.’’
  (page~1, paragraphs~2)
  \item correct unit is $\mu$s (page~4, paragraphs~2)
  \item the correct beginning of sentence is ‘‘This behavior...’’ (page~6, paragraphs~2)
  \item ‘‘carrier generate rate’’ in Figure~5 caption is replaced by ‘‘carrier generation rate’’ (page~7).
\end{itemize}

Other numerous corrections are highlighted in blue in ``RevisedManuscript.pdf''.



\vspace{1cm}
\subsection*{Response to Reviewer \#2 }


\noindent
\textcolor[rgb]{0.00,0.50,1.00}{\textbf{Comment~1.}}
\emph{Grammatical error in the last sentence of the abstract.}


\noindent
\textcolor[rgb]{0.51,0.00,0.00}{\textbf{Reply:}}
The revised last sentence in the abstract reads:

‘‘The results indicate that recombination-enhanced defect reaction is the primary factor in the second stage of pair dissociation.’’

The manuscript has been revised.
We sincerely hope to see a significant decrease in grammatical errors.



\vspace{1cm}
\noindent
\textcolor[rgb]{0.00,0.50,1.00}{\textbf{Comment~2.}}
\emph{Figure1 and TOC, it looks nice, but it is difficult to understand.
Authors need to add inputs and outputs.
One of outputs will be K values.
For example, add an arrow coming out from the box named "Discussion \& Conclusion",
and write K.
Otherwise, readers will not understand what is this figure for.
Page2, Line40, there is an explanation of "First step".
I do not find "Second step".}

\noindent
\textcolor[rgb]{0.51,0.00,0.00}{\textbf{Reply:}}
The Figure~1 has been modified according to the Reviewer’s suggestions.
Additionally, the boxes in the figure have been numbered,
and these numbers are referenced in the updated description
(page~2, paragraph~4).


\vspace{1cm}
\noindent
\textcolor[rgb]{0.00,0.50,1.00}{\textbf{Comment~3.}}
\emph{A main result is that K values depends on the light source, namely spectral shape.
These values were obtained by fitting equation(1) to the plot in Fig.~6(a).
I have a doubt that K should not change by experimental condition.
It is an intrinsic constant.
For the fitting, N\_FeB was assumed to be constant in the manuscript, but it may not.}

\emph{
\textbf{Comment3 is critical. Explanations are required.}
}

\noindent
\textcolor[rgb]{0.51,0.00,0.00}{\textbf{Reply:}}
For us, it was also somewhat surprising that $K$,
which was considered a material constant, depended on illumination parameters.
However, from a physical point of view, such dependence is not entirely incomprehensible.
Indeed, according to the recombination-enhanced defect reaction,
electron--phonon interactions during the dissociation of FeB pairs lead to the athermal diffusion of iron ions \cite{FeBKin2019,KIMERLINGFeB}.
The increase in the number of phonons due to the thermalization of photoelectrons,
which occurs when higher energy photons are absorbed,
should intensify the electron-phonon interaction process.
Such a phenomenon is observed in the experiment and causes changes in the value of $K$,
which integrally characterizes the dissociation process of the pairs.

Regarding the fitting where the concentration of pairs was assumed to be constant,
the study presents results obtained under the illumination of a single solar cell at a constant temperature (340 K).
Before strong illumination, the sample was kept in the dark for an extended period.

Therefore, there is no reason to expect that the total concentration of defects or the ratio between the amounts
of FeB and Fe$_i$ would change.
Moreover, this was verified in the experiment (see Table~1, last column).


\vspace{1cm}
\noindent
\textcolor[rgb]{0.00,0.50,1.00}{\textbf{Comment~4.}}
\emph{A unit "cm\^3" is in equation(5). It should not be.}

\noindent
\textcolor[rgb]{0.51,0.00,0.00}{\textbf{Reply:}}
We respectfully disagree with the Reviewer.
On one hand, the factor of $5.7\times10^5$
corresponds precisely to the case
when the unit of acceptor--doping concentration is cm$^{-3}$ \cite{FeBAssJAP2014,FeBKin2019,FeBAssSST2011}.
On the other hand,
since the temperature unit is Kelvin
and the unit of $p$ is cm$^{-3}$,
the factor $\left(\frac{1}{\mathrm{K}\;\mathrm{cm}^3}\cdot\frac{T}{p}\right)$ is dimensionless.
Thus, the expression
\begin{equation}
\label{eqTass}
R_a^{-1}=5.7\times10^5\,\frac{\mathrm{s}}{\mathrm{K}\;\mathrm{cm}^3}\times\frac{T}{p}\exp\left(\frac{E_m}{kT}\right)\,
\end{equation}
has the dimension of time.


\vspace{1cm}
\noindent
\textcolor[rgb]{0.00,0.50,1.00}{\textbf{Comment~5.}}
\emph{ I did not find where Eq.(3) and (4) are used.
Exactly the same equations are in the cited articles.
These equations and related sentences can be deleted so that readers can easily understand the content.}

\noindent
\textcolor[rgb]{0.51,0.00,0.00}{\textbf{Reply:}}
We fully understand that it is not good practice to disagree with the reviewer;
however, we kindly request permission to retain the equations.
Eq.~(3) is needed to demonstrate a relationship between the fitting parameters of the experimental curve
(Figure~3) and defect characteristics --- see page~4, paragraph~6.
Eq.~(4) allows for an easy and understandable entering of $t_\mathrm{dark}$
(time after stopping strong illumination),
which is extensively used throughout the text and figures.
Additionally, the equation directly relates to the time dependence of the recovery of the short--circuit current (Fig. 2a)
and is mentioned on page 2, second paragraph from the bottom.

While references to cited articles can be sufficient,
we believe that including Eq.(3) and (4) enables readers to clearly understand
the connection between the experiment and the FeB characteristics within the study framework.





\vspace{1.5cm}

Sincerely yours,

Oleg Olikh, Oleksandr Datsenko, and Serhiy Kondratenko.


Taras Shevchenko National University of Kyiv


Kyiv 01601, Ukraine

E-mail: olegolikh@knu.ua

\bibliographystyle{MSP}
\bibliography{olikh}


\end{document}

%Будь-ласка, покращ англійську в наступних реченнях, які я буду пропонувати. Зокрема, треба буде виправити граматичні та стилістичні помилки
%якщо можна, окрім виправленого речення, наводь інформацію про зміни, які зроблені 