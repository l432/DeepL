\documentclass[fleqn,twoside,twocolumn,nofootinbib,showkeys]{revtex4} % Specifies the document class %,unsortedaddress
\usepackage[sec,doi,cyr]{ujp_UTF8} 
%\usepackage[cyr]{ujp_UTF8} %for cyrillic
\usepackage{multirow}
\usepackage{makecell}

\begin{document}
\title[Застосування моделей машинного навчання для оцінки рухливості носіїв у Si]%колонтитул
{ЗАСТОСУВАННЯ МОДЕЛЕЙ МАШИННОГО НАВЧАННЯ ДЛЯ ЕФЕКТИВНОЇ ОЦІНКИ РУХЛИВОСТІ НОСІЇВ У КРЕМНІЇ}
\author{O.Я.~Оліх}%1 автор
\affiliation{Київський національний університет імені Тараса Шевченка}%институт
\address{Україна, 01601, місто Київ, вул. Володимирська, 64/13}%адрес
\email{olegolikh@knu.ua}%e-mail
%\affiliation{National University ``Lvivska Politechnika''}%
%\address{12, Bandera Str., Lviv 79013, Ukraine}%
%\author{D.~Fruchart}
%\affiliation{Laboratoire de N\'{e}el, CNRS}%
%\address{BP 166, 38042 Grenoble Cedex 9, France}%
%\author{L.P.~Romaka}%
%\affiliation{Ivan Franko Lviv National University}%
%\address{6, Kyrylo and Mefodii Str., Lviv 79005, Ukraine}%
%\author{A.M.~Horyn}%
%\affiliation{Ivan Franko Lviv National University}%
%\address{6, Kyrylo and Mefodii Str., Lviv 79005, Ukraine}%
%\author{O.V.~Bovgyra}%
%\affiliation{Ivan Franko Lviv National University}%
%\address{6, Kyrylo and Mefodii Str., Lviv 79005, Ukraine}%
%\author{R.V.~Krayovskyy}%
%\affiliation{National University ``Lvivska Politechnika''}%
%\address{12, Bandera Str., Lviv 79013, Ukraine}%

\udk{004.932:537.311:621.382.21} \pacs{72.20.Fr, 72.80.Cw, 07.05.Tp} \razd{\secvii}

\autorcol{O.Я.~Оліх}

%\cit{Romaka~V.A., Fruchart~D., Romaka~L.P., Horyn~A.M., Bovgyra~O.V., Krayovskyy~R.V. Mechanism of defect formation in heavily Y-doped $n$-ZrNiSn. II.~Electro-transport studies.}{\selectlanguage{ukrainian} Ромака~В.А., Фрушарт~Д., Бовгира~О.В., Ромака~Л.П., Горинь~А.М., Крайовський~Р.В. Механізм дефектоутворення у сильнолегованому атомами Y $n$-ZrNiSn. II.~Електрокінетичні дослідження.}

\setcounter{page}{1}
%\setcounter{jnumber}{5}

\begin{abstract}
The temperature and concentration dependences of the resistivity $\rho$
and the thermopower coefficient $\alpha$ of a
Zr$_{1-x}$Y$_{x}$NiSn intermetallic semiconductor heavily doped with
an acceptor Y impurity have been studied in
the ranges $T=80\div380~\mathrm{K}$ and $N_{\rm A}^{Y}=3.8\times10^{10}%
$~\textrm{cm}$^{-3}$
($x=0.02)\div4.8\times10^{21}$~\textrm{cm}$^{-3}$ ($x=0.25$). A
conclusion on the mechanisms of conductivity in this compound is
made. The dependences between the impurity concentration and the
parameters of the modulation amplitude of the continuous energy
bands have been established. The results obtained are
discussed in the framework of the Shklovskii--Efros model for a
heavily doped and strongly compensated semiconductor.
\end{abstract}

\keywords{mechanism of defect formation, Shklovskii--Efros model,
strongly compensated semiconductor, heavily doped semiconductor.}

\maketitle

\section{Introduction}

In the previous work \cite{1}, the structure characteristics of the intermetallic
semiconductor $n$-ZrNiSn heavily doped with an Y impurity have been studied,
and the corresponding calculations of the electron density distribution and
the band structure have been carried out. The doping of $n$-ZrNiSn was
demonstrated to be accompanied by the ordering of its crystal structure, when
the impurity atoms occupy only the positions of Zr ones and generate
acceptor-type defects. By detecting the conductivity transition
insulator--metal, the existence range of the solid solution Zr$_{1-x}$Y$_{x}%
$NiSn and the dependences between the impurity concentration, on the one hand,
and the direction and the rate of Fermi level drift, on the other hand, have
been established. In particular, the introduction of an acceptor Y impurity into
the $n$-ZrNiSn crystal structure is accompanied by a redistribution of
the electron concentration, a monotonous motion of the Fermi level from the
conduction band edge to the valence band, and its intersection at $x=0.137$. A
conclusion was drawn that the mechanism of hopping conductivity in Zr$_{1-x}%
$Y$_{x}$NiSn takes place at concentrations of the Y impurity, when
the Fermi level is located below the percolation levels in the
conduction band or the valence band ($x<0.137$).

This work is a sequel of work \cite{1}. Here, we report the results
of our researches concerning the electro-transport and energy
characteristics of $n$-ZrNiSn heavily doped with an acceptor Y
impurity. The temperature dependences of the electroresistivity
$\rho$ and the thermopower coefficient $\alpha$ were measured. The
impurity concentration was changed in the range from $N_{\rm
A}^{Y}=3.8\times10^{20}$~\textrm{cm}$^{-3}$ (at $x=0.02$) to
$4.8\times10^{21}~\mathrm{cm}^{-2}$ (at $x=0.25$). The techniques
applied for the measurements of the electroresistivity and the
thermopower coefficient relative to copper in the temperature
interval $T=80\div380~\mathrm{K}$ are discussed in work~\cite{2}.

%Fig.~1
\begin{figure*}% figure* for wide figure, [h] [!] to change the placement
%\includegraphics[width=15cm]{1_e}
\caption{Temperature dependences of the resistivity $\rho$ and the
thermopower coefficient $\alpha$ for Zr$_{1-x}$Y$_{x}$NiSn at
$x=0.02$ ({\it 1}), 0 ({\it 2}), 0.08 ({\it 3}), and 0.2
({\it 4})  }
\end{figure*}

\section{Electro-Transport\\ Researches of Zr\boldmath$_{1-x}$Y\boldmath$_{x}$NiSn}

The temperature dependences of the electrical resistivity $\ln\rho(1/T)$ and
the thermopower coefficient $\alpha(1/T)$ of Zr$_{1-x}$Y$_{x}$NiSn are
depicted in Fig.~1. They demonstrate that Zr$_{1-x}$Y$_{x}$NiSn specimens with
$x=0\div0.08$ reveal semiconducting properties: the resistivity decreases with
the temperature growth, and the dependences $\ln\rho(1/T)$ and $\alpha(1/T)$
manifest high- and low-temperature activation sections. The specimen with
$x=0.08$ is an exception: its dependence $\ln\rho(1/T)$ contains no
low-temperature activation section. For specimens with the content of Y
impurity $x\geq0.2$, the conductivity has a metallic character, and their
electroresistance grows, as the temperature increases. The dependences
$\ln\rho(1/T)$ can be approximated with a high accuracy by the known
relation
%1
\begin{equation}
\rho^{-1}(T)=\rho_{1}^{-1}\exp\left(  -\frac{\varepsilon_{1}^{\rho}}{k_{\rm B}%
T}\right)  +\rho_{3}^{-1}\left(  -\frac{\varepsilon_{3}^{\rho}}{k_{\rm B}%
T}\right)  ,\label{GrindEQ__1_}%
\end{equation}
where the first high-temperature term describes the charge carrier activation
from the Fermi level onto the percolation level in the continuous energy
bands, and the second low-temperature term describes hopping conduction
\cite{3}.

In turn, the temperature dependences $\alpha(1/T)$ of the
Zr$_{1-x}$Y$_{x} $NiSn thermopower coefficient can be approximated
by the following formula:
%2
\begin{equation}
\alpha=\frac{k_{\rm B}}{e}\left(  \frac{\varepsilon_{V}^{\alpha}}{k_{\rm B}T}%
-\gamma+1\right)  ,\label{GrindEQ__2_}%
\end{equation}
where $\gamma$ is a parameter dependent on the scattering origin, and, in the
case of $n$-ZrNiSn, it equals 1.04.



The introduction of the lowest concentrations of Y impurity is accompanied by
a reduction of the electroconductivity $\sigma(x)$ (see Fig.~2). For instance, at
$T=80~\mathrm{K}$, the conductivity diminishes from $\sigma=7.6\times
10^{-3}~(\mu\Omega\times\mathrm{m)}^{-1}$ at $x=0$ to $1.2\times10^{-4}%
~(\mu\Omega\times\mathrm{m)}^{-1}$ at $x=0.02$. Such a behavior of
$\sigma(x)$ is connected with a reduction of the density of states
at the Fermi level, when the compensation degree in a semiconductor
of the electron conductivity type changes owing to the introduction
of an acceptor impurity. The minimum in the dependence $\sigma(x)$
corresponds to a state close to the complete compensation of
the semiconductor: the concentrations of donor- and acceptor-type
defects are almost in equilibrium, and the electroconductivity is caused
by free electrons and holes simultaneously, as well as by the hops
of charge carriers between localized states. We associate the
increase of $\sigma(x)$ at $x\geq0.02$ with the intersection of the
energy gap midpoint by the Fermi level $\varepsilon_{\rm F}$ and
with increase in the concentration of free holes by their
thermal throwing from the Fermi level onto the percolation level in
the valence band.


\begin{table}[b]
\noindent\caption{Concentration and energy\\ characteristics of
Zr\boldmath$_{1-x}$Y$_{x}$NiSn alloys }\vskip3mm\tabcolsep4.5pt

\noindent{\footnotesize\begin{tabular}{|c|c|c|c|c|c|}
 \hline \multicolumn{1}{|c}
{\rule{0pt}{5mm}$x$} & \multicolumn{1}{|c}{$N_{\rm A}$, cm$^{-3}$}&
\multicolumn{1}{|c}{$\varepsilon _{1}^{\rho }$, meV}&
\multicolumn{1}{|c}{$\varepsilon _{1}^{\alpha }$, meV}&
\multicolumn{1}{|c}{$\varepsilon _{3}^{\rho }$, meV}&
\multicolumn{1}{|c|}{$\varepsilon _{3}^{\alpha }$, meV}\\[2mm]%
\hline%
\rule{0pt}{5mm}0& --& 28.9& 44.6& 1.6& 11.5 \\%
0.02& 3.8\,$\times $\,10$^{20}$& 61.8& 133.4& 2.3& 2.4 \\%
0.05& 9.5\,$\times $\,10$^{20}$& 16.3& 21.1& 1.7& 2.2 \\%
0.08& 1.5\,$\times $\,10$^{21}$& 12.9& 17.3& 1.1& 0.8 \\%
0.12& 2.3\,$\times $\,10$^{21}$& 8.3& 7.9& 0& 0.7 \\%
0.2& 3.8\,$\times $\,10$^{21}$& 0& 8.6& 0& 0.5 \\%
0.25& 4.8\,$\times $\,10$^{21}$& 0& 8.4& 0& 0.3 \\[2mm]%
\hline
\end{tabular}}
\end{table}

The character of variations of the thermopower coefficient $\alpha(x)$ in Zr$_{1-x}%
$Y$_{x}$NiSn agrees with the arguments given above on the
simultaneous participation of charge carriers of several types in
the electroconductivity. The charge carrier concentration changes as
either the number of introduced Y atoms or the temperature varies;
this occurs through the variation of the number of ionized
acceptors. We connected the decrease of $\sigma(x)$ in the interval
$x=0\div0.02$ with a reduction of the contribution made by free
electrons of the semiconductor, due to their capture by acceptors.
In the same concentration range, the values of $\alpha(x)$ are
negative, which testifies to the electron type of conductivity of
the semiconductor, whereas a reduction of $\alpha(x)$-values
indicates a reduction of the electron contribution to the
Zr$_{1-x}$Y$_{x}$NiSn conductivity. The minimum in the dependences
$\sigma(x),$ when the concentrations of electrons and holes become
close, is associated with the conductivity type change, and the
values of $\alpha(x)$ are close to zero there. At Y contents
$x>0.02$ in Zr$_{1-x}$Y$_{x}$NiSn, free holes determine the
conductivity of the semiconductor, and the positive sign of the
thermopower coefficient evidences for that. A practical invariance
of $\alpha(x)$-values in Zr$_{1-x}$Y$_{x}$NiSn at $x\geq0.12$
testifies to the intersection between the Fermi level
$\varepsilon_{\rm F}$ and the percolation level in the valence band,
i.e. the electroconductivity transition insulator--metal is
realized.


%Fig.~2
\begin{figure}% figure* for wide figure, [h] [!] to change the placement
%\includegraphics[width=\column]{2_e}
\vskip-3mm\caption{Y-content dependences of the specific
conductivity $\sigma$ ($a$) and the thermopower coefficient $\alpha$
($b$) of Zr$_{1-x}$Y$_{x}$NiSn at various temperatures: $T=80$ ({\it
1}), 250 ({\it 2}), and 370$~\mathrm{K}$ ({\it 3})  }
\end{figure}

Analyzing the high- and low-temperature activation sections in the
dependences $\ln\rho(1/T)$ and $\alpha(1/T)$ for Zr$_{1-x}$Y$_{x}$NiSn and using the
relations (\ref{GrindEQ__1_}) and (\ref{GrindEQ__2_}), we
calculated the activation energies (see Table). In particular, we
determined the energies of activation $\varepsilon_{1}^{\rho}$ from the Fermi
level onto the percolation level in the conduction (valence) band from the
high-temperature activation sections in the dependences $\ln\rho(1/T)$ and
the activation energies $\varepsilon_{3}^{\rho}$
of hopping conduction from the low-temperature ones. Similarly, from the analogous activation sections in
the dependences $\alpha(1/T)$, we calculated the activation energies
$\varepsilon_{1}^{\alpha}$ and $\varepsilon_{3}^{\alpha}$. In work \cite{4},
it was shown that the values for $\varepsilon_{1}^{\rho}$ and $\varepsilon
_{1}^{\alpha}$ determined from the activation sections of the dependences
$\ln\rho(1/T)$ and $\alpha(1/T)$, respectively, are substantially different:
$\varepsilon_{1}^{\rho}$ gives the difference between the Fermi and
percolation levels, whereas $\varepsilon_{1}^{\alpha}$ evaluates the amplitude
of continuous energy band fluctuations. The activation energies $\varepsilon
_{3}^{\rho}$ and $\varepsilon_{3}^{\alpha}$ are related to the occupation
degree and the amplitude of a small-scale fluctuation, respectively (Fig.~3). On
the one hand, the results obtained confirm the conclusions made above on the
role of carriers of different types and the mechanisms of conductivity in
Zr$_{1-x}$Y$_{x}$NiSn. On the other hand, they evaluate the drift rate of
$\varepsilon_{\mathrm{F}}$ and determine the modulation amplitudes and
the \textquotedblleft fine structure\textquotedblright\ parameters of the
energy bands \cite{5,6}.



%Fig.~3
\begin{figure}% figure* for wide figure, [h] [!] to change the placement
%\includegraphics[width=\column]{3_e}
\vskip-3mm\caption{Y-content dependences of the activation energies in Zr$_{1-x}$Y$_{x}%
$NiSn: ($a$)~$\varepsilon_{1}^{\alpha}$ ({\it 1}) and $\varepsilon
_{1}^{\rho}$ ({\it 2}), ($b$)~$\varepsilon_{3}^{\alpha}$ ({\it 1})
and $\varepsilon_{3}^{\rho}$ ({\it 2})  }
\end{figure}

The value $\varepsilon_{1}^{\rho}(x=0)=28.9$~meV estimates a
remoteness of the Fermi level from the mobility edge of the
conduction band in $n$-ZrNiSn, which is evidenced by the
negative values of $\alpha(x)$. The introduction of the lowest
experimental concentration of the Y impurity moves the Fermi level away
from the percolation one in the conduction band. The value
$\varepsilon _{1}^{\rho}(x=0.02)$ reflects the position of the
Fermi level in a practically completely compensated semiconductor --
close to the energy gap mid-point. However, in our specimen, there
is an insignificant overcompensation by acceptors, which is evidenced by the
positive values of $\alpha(x)$.
In this case, the value $\varepsilon_{1}^{\rho}(x=0.02)=61.8$%
~meV corresponds to the remoteness of the Fermi level from the percolation
level in the valence band. A drastic recession in the dependence
$\varepsilon_{1}^{\rho}(x)$ in the interval $x=0.02\div0.05$, together with a
practically linear reduction in the interval $x=0.05\div0.2$, describes the
dynamics of the Fermi level drift toward the percolation level in the valence
band of Zr$_{1-x}$Y$_{x}$NiSn. Basing on the linear character of the
dependence $\varepsilon_{1}^{\rho}(x)$ in the interval $x=0.05\div0.2$, we
determined the drift rate of $\varepsilon_{\mathrm{F}}$ toward the conduction
band to be $\Delta\varepsilon_{\mathrm{F}}/\Delta x=1.1\mathrm{~meV/at.\%}$.
The fact that $\varepsilon_{1}^{\rho}(x)=0$ at $x\geq0.2$ testifies to the
intersection between the Fermi level and the percolation level in the valence
band. For such specimens, no high-temperature activation sections would be
observed in the dependences $\ln\rho(1/T)$ (Fig.~1).

The behavior of $\varepsilon_{1}^{\alpha}(x)$ agrees completely with the
Shklovskii--Efros model describing the energy state of a heavily doped and
compensated semiconductor \cite{5,6}. Really, in a completely compensated
semiconductor, the amplitude of a large-scale fluctuation is maximum, being
equal to the activation energy from the Fermi level onto the percolation one,
and the Fermi level $\varepsilon_{\mathrm{F}}$ is located in the middle of the
energy gap. As is seen from Fig.~3, the dependence $\varepsilon_{1}^{\alpha
}(x)$ is maximum at the Zr$_{1-x}$Y$_{x}$NiSn content, $x\approx0.02$, close to
the complete compensation state. In the latter case, in accordance with the
conclusions of work \cite{6}, the activation energies $\varepsilon_{1}^{\rho}$
and $\varepsilon_{1}^{\alpha}$ coincide. In our case, the values
$\varepsilon_{1}^{\rho}(x=0.02)=61.8$~meV and $\varepsilon_{1}^{\alpha
}(x=0.02)=133.4$~meV are maximum but substantially different. This fact is
related to the slight overcompensation of a Zr$_{1-x}$Y$_{x}$NiSn specimen
with $x=0.02$. Above the overcompensation of Zr$_{1-x}$Y$_{x}$NiSn (at $x>0.02$), the
reduction of $\varepsilon_{1}^{\alpha}(x)$-values testifies to a decrease of
the continuous energy band modulation amplitude, which is accompanied by the
ordering of the semiconductor crystal structure.

From Fig.~3, it also follows that the behaviors of the
Zr$_{1-x}$Y$_{x}$NiSn dependences $\varepsilon_{1}^{\rho}(x)$ and
$\varepsilon_{1}^{\alpha}(x)$ correlate with each other. It is also
true for the potential well depth of a small-scale fluctuation
$\varepsilon_{3}^{\alpha}(x)$ and the occupation degree of this
potential well, the latter being proportional to the energy of
hopping conduction $\varepsilon_{3}^{\rho}(x)$. As soon as the
potential well depth becomes less than 0.8~meV ($x=0.12$), the
activation energy of hopping conduction is nullified---the
small-scale relief of the valence band becomes \textquotedblleft
flooded\textquotedblright\ with holes, and electrons are activated
from the Fermi level onto the percolation one in the valence band
only.

\section{Conclusions}

Thus, the results of electro-transport researches dealing with the doping of
$n$-ZrNiSn with an acceptor Y impurity agree with the results of structural
researches and the calculations of the electron density distribution and the band
structure of the semiconductor. It is shown that the doping of $n$-ZrNiSn
with an Y impurity allows a predictable control over the fabrication of
thermoelectric and thermometric substances with preassigned properties. On the
basis of the results given above, we may assert that the obtained
semiconductor solid solution Zr$_{1-x}$Y$_{x}$NiSn is a promising
thermoelectric substance.

\vskip3mm \textit{The work was sponsored by the National Academy of
Sciences of Ukraine (grant No.~0106U000594) and the Ministry of
Education and Science of Ukraine (grants Nos.~0109U002069 and
0109U001151).}

\section*{ДОДАТОК~A. Оцінка рухливості відповідно до моделі Klaassen (Philips)}

\setcounter{equation}{0} % скидаємо лічильник формул
\renewcommand{\theequation}{Д.1.\arabic{equation}}
\setcounter{table}{0} % скидаємо лічильник таблиць
\renewcommand{\thetable}{Д.1.\arabic{table}}

В цьому випадку для обчислення рухливості як електронів, так і дірок, незалежно від того, чи вони є основними чи неосновними носіями, 
застосовують схожі формули з різними коефіцієнтами.
А саме
\begin{equation}\label{eqK1}
  \mu^\mathrm{K}=\frac{\mu_\mathtt{L}\cdot\mu_\mathtt{DA}}{\mu_\mathtt{L}+\mu_\mathtt{DA}}\,,
\end{equation}
де
\begin{eqnarray}
  \label{eqK2}
  \mu_\mathtt{L} &=&  \mu_\mathrm{max} \left(\frac{300}{T}\right)^{2,25},\, \\ 
  \mu_\mathtt{DA}  &=& \frac{\mu_\mathrm{max}^2}{\mu_\mathrm{max} - \mu_\mathrm{min}} \cdot
    \frac{N_\mathrm{sc}}{N_\mathrm{eff}} \cdot \left(\frac{N_\mathrm{ref}}{N_\mathrm{sc}}\right)^{\alpha} \cdot \left(\frac{T}{300}\right)^{3\alpha - \frac{3}{2}}+\nonumber\\
    \label{eqK3}
    &&+ \frac{\mu_\mathrm{max} \, \mu_\mathrm{min}}{\mu_\mathrm{max} - \mu_\mathrm{min}} \cdot \frac{n + p}{N_\mathrm{eff}} \cdot \left(\frac{300}{T}\right)^\frac{1}{2},\,
\end{eqnarray}
а константи, що входять до рівнянь (\ref{eqK2})-(\ref{eqK3}) наведені у Табл.~\ref{tblK1}.

\begin{table}
\caption{Коефіцієнти для розрахунку рухливості відповідно до моделі Klaassen, формули (\ref{eqK2})-(\ref{eqK3})}
\label{tblK1}
\vspace{2mm}
\centering
\begin{tabular}{|l|c|c|c|c|}
\hline
\multirow{2}{*}{Тип носіїв} & \multicolumn{4}{c|}{Параметр} \\
\cline{2-5}
&\makecell{$\mu_\mathrm{max}$, $\frac{\text{см}^2}{\text{B}\cdot\text{с}}$}&\makecell{$\mu_\mathrm{min}$, $\frac{\text{см}^2}{\text{B}\cdot\text{с}}$}&$\alpha$&$N_\mathrm{ref}$, см$^{-3}$ \\
\hline
Електрони&1414&68,5&0,711&$9,2\cdot10^{16}$\\
\hline
Дірки&495&44,9&0,719&$2,23\cdot10^{17}$\\
\hline
\end{tabular}
\end{table}

Величини $N_\mathrm{sc}$ та $N_\mathrm{eff}$ є функціями, вигляд яких залежить від типу носіїв, а саме
при визначенні рухливості електронів
\renewcommand{\theequation}{Д.1.\arabic{equation}a}
\begin{eqnarray}
  \label{eqK4a}
  N_\mathrm{sc} &=&N_d^+ + N_a^-+ p\,, \\
  \label{eqK5a}
  N_\mathrm{eff} &=&N_d^+ + N_a^- \cdot G_n(P_n,T) +\frac{p}{F_n(P_n,T)}\,,
\end{eqnarray}
а для дірок
\addtocounter{equation}{-2}
\renewcommand{\theequation}{Д.1.\arabic{equation}б}
\begin{eqnarray}
  \label{eqK4b}
  N_\mathrm{sc}&=& N_a^-+N_d^++n\,, \\
  \label{eqK5b}
  N_\mathrm{eff}&=& N_a^- + N_d^+ \cdot G_p(P_p,T) +\frac{n}{F_p(P_p,T)}\,.
\end{eqnarray}


%\begin{subequations} \label{eqK4}
%    \begin{align}
%       \text{електрони:}& \nonumber\\
%       N_\mathrm{sc} =N_d^+ + N_a^-+ p\,, &\label{eqK4a} \\
%       N_\mathrm{eff} =N_d^+ + N_a^- \cdot G_n(P_n,T) +\frac{p}{F_n(P_n,T)}\,,& \label{eqK5a} \\
%        \text{дірки:} \quad& N_{sc}= N_a^-+N_d^++n\,, \label{eqK4b}
%    \end{align}
%\end{subequations}

%\begin{subequations} \label{eqK5}
%    \begin{align}
%       \text{електрони:} \quad& N_\mathrm{eff} =N_d^+ + N_a^- \cdot G_n(P_n,T) +\frac{p}{F_n(P_n,T)}\,, \label{eqK5a} \\
%        \text{дірки:} \quad& N_\mathrm{eff}= N_a^- + N_d^+ \cdot G_p(P_p,T) +\frac{n}{F_p(P_p,T)}\,. \label{eqK5b}
%    \end{align}
%\end{subequations}

\begin{table}[b]
\noindent\caption{Concentration and energy\\ characteristics of
Zr\boldmath$_{1-x}$Y$_{x}$NiSn alloys }\vskip3mm\tabcolsep4.5pt

\noindent{\footnotesize\begin{tabular}{|c|c|c|c|c|c|}
 \hline \multicolumn{1}{|c}
{\rule{0pt}{5mm}$x$} & \multicolumn{1}{|c}{$N_{\rm A}$, cm$^{-3}$}&
\multicolumn{1}{|c}{$\varepsilon _{1}^{\rho }$, meV}&
\multicolumn{1}{|c}{$\varepsilon _{1}^{\alpha }$, meV}&
\multicolumn{1}{|c}{$\varepsilon _{3}^{\rho }$, meV}&
\multicolumn{1}{|c|}{$\varepsilon _{3}^{\alpha }$, meV}\\[2mm]%
\hline%
\rule{0pt}{5mm}0& --& 28.9& 44.6& 1.6& 11.5 \\%
0.02& 3.8\,$\times $\,10$^{20}$& 61.8& 133.4& 2.3& 2.4 \\%
0.05& 9.5\,$\times $\,10$^{20}$& 16.3& 21.1& 1.7& 2.2 \\%
0.08& 1.5\,$\times $\,10$^{21}$& 12.9& 17.3& 1.1& 0.8 \\%
0.12& 2.3\,$\times $\,10$^{21}$& 8.3& 7.9& 0& 0.7 \\%
0.2& 3.8\,$\times $\,10$^{21}$& 0& 8.6& 0& 0.5 \\%
0.25& 4.8\,$\times $\,10$^{21}$& 0& 8.4& 0& 0.3 \\[2mm]%
\hline
\end{tabular}}
\end{table}


\begin{thebibliography}{9}                                                                                                %


\bibitem {1}V.V.~Romaka, E.K.~Hlil, O.V.~Bovgyra, L.P.~Romaka, V.M.~Davydov,
R.V.~Krayovskyy. Mechanism of defect formation in heavily Y-doped
$n$-ZrNiSn. I. Crystal and electronic structures. \textit{Ukr. J.
Phys.} \textbf{54}, 1119 (2009).

\bibitem {2}L.~Romaka, Yu.~Stadnyk, M.G.~Shelyapina {\it et~al.}
Electronic structure of the Ti$_{1-x}$Sc$_x$NiSn and
Zr$_{1-x}$Sc$_x$NiSn solid solutions. \textit{J.~Alloys Comp.}
\textbf{396}, 64 (2005).

\bibitem {3}B.I.~Shklovskii, A.L.~Efros. \textit{Electronic Properties of
Doped Semiconductors} (Springer,  1984) [ISBN: 978-3-662-02403-4].

\bibitem {4}V.A.~Romaka, Yu.V.~Stadnyk, D.~Fruchart {\it et~al.}
Features of electrical conductivity in the n-ZrNiSn intermetallic
semiconductor heavily doped with the In acceptor impurity.
\textit{Semiconductors} \textbf{41}, 1041 (2007).

\bibitem {5}B.I.~Shklovski\u{i}, A.L.~\'{E}fros.
Transition from metallic to activation conductivity in compensated
semiconductors.  \textit{Sov. Phys. JETP} \textbf{34}, 435 (1972).

\bibitem {6}B.I.~Shklovski\u{i}, A.L.~\'{E}fros.
Completely compensated crystalline semiconductors as a model of an
amorphous semiconductor. \textit{Zh. \`{E}ksp. Teor. Fiz.}
\textbf{62}, 1156 (1972).

\begin{flushright}
{\footnotesize Received 22.01.22}
\end{flushright}
\end{thebibliography}

\vspace*{-5mm} \rezume{%
O.Ya. Olikh} {MACHINE LEARNING MODELS FOR EFFICIENT MOBILITY ESTIMATION IN SILICON} {Досліджено вплив
акцепторної домішки Y на зміну питомого електроопору ($\rho $),
коефіцієнта термо-ерс ($\alpha $), енергетичних характеристик
інтерметалічного напівпровідника $n$-ZrNiSn у діапазонах: $T = 80
-380$ К, $N_{\rm A}^{Y}\approx 3,8\cdot 10^{20}$ см$^{-3}$ ($x =
0{,}02)$--$4{,}8\cdot 10^{21}$~см$^{-3 }$ ($x = 0{,}25$). Зроблено
висновки про механізми електропровідності Zr$_{1-x}$Y$_{x}$NiSn.
Встановлено залежності між концентрацією домішки та характеристиками
амплітуди модуляції зон неперервних енергій. Обговорення результатів
проводиться у межах моделі сильнолегованого і сильнокомпенсованого
напівпровідника Шкловського--Ефроса.}{\textit{K\,e\,y\,w\,o\,r\,d\,s:} механізм дефектоутворення, модель
Шкловського--Ефроса, сильнокомпенсований напівпровідник,
сильнолегований напівпровідник.}
\end{document}
