

%\documentclass[aip,reprint]{revtex4-1}
%\documentclass[sn-mathphys]{sn-jnl}
\documentclass[10pt]{iopart}
%\documentclass[aip,jap,preprint]{revtex4-1}
\usepackage{graphicx}% Include figure files
\usepackage{dcolumn}
\usepackage{color}
\usepackage{color,soul}
%\draft % marks overfull lines with a black rule on the right

\begin{document}

Dear Editor and Reviewers,

We sincerely thank you for taking the time to review our manuscript
``Computer vision-based method for quantifying iron-related defects in silicon solar cells''
(Ref. No.: SST--111269).
Your insightful comments and constructive suggestions have greatly helped us improve
the quality of our work.
We particularly appreciate your careful reading and thoughtful feedback,
which have led to significant improvements in both the technical content and presentation clarity of our manuscript.
We have carefully addressed all the comments and made corresponding revisions to the manuscript.
The location of revisions is pointed by red color and highlighted in yellow in ``CompleteDocumentForReview.pdf''.
Below we provide our detailed point-by-point responses to each comment.
We hope the revised manuscript better meets your expectations and standards for publication in Solar Energy.



\subsection*{Response to Reviewer \#1 }

\noindent
\textcolor[rgb]{0.00,0.50,1.00}{\textbf{Major Comment~1.}}
\emph{Novelty and Contribution:
The idea of applying pre-trained CV models to wavelet-transformed kinetic data is interesting and potentially generalizable.
However, similar signal-to-image ML transformations exist in other domains.
The manuscript should emphasize what new physical or methodological insight this work provides beyond prior
Fourier/wavelet-based ML approaches..}

\noindent
\textcolor[rgb]{0.51,0.00,0.00}{\textbf{Reply:}}

The additional information was added to the revised manuscript
(from page~7, left column, paragraphs 2 to page~8, left column, paragraphs 2;
page 8, right column, paragraph~2).


\vspace{1cm}
\noindent
\textcolor[rgb]{0.00,0.50,1.00}{\textbf{Major Comment~2.}}
\emph{Dataset Size and Overfitting Risk:
The study relies on extremely small datasets (25 simulated and 28 experimental samples).
Although data augmentation is performed, flipping or rotating spectrograms likely introduces redundant samples rather
than independent data points.
The near-perfect $R^2$ values (0.996–0.999) strongly suggest overfitting.
A robust cross-validation (e.g., k-fold or leave-one-out) with uncertainty quantification is needed.}

\noindent
\textcolor[rgb]{0.51,0.00,0.00}{\textbf{Reply:}}

The minimal level shift, which was used as evidence of defect transformation, was 40~meV.





\vspace{1cm}
\noindent
\textcolor[rgb]{0.00,0.50,1.00}{\textbf{Major Comment~3.}}
\emph{Simulation–Experiment Gap:
A major discrepancy is observed between simulated
and experimental predictions, requiring a post-hoc quadratic correction.
This implies that the CNN–regressor models primarily learn the synthetic data distribution rather
than physical correlations.
The authors should explore physics-based domain adaptation or partial fine-tuning using experimental data instead of empirical correction.}

\noindent
\textcolor[rgb]{0.51,0.00,0.00}{\textbf{Reply:}}


The description of observation technique was modified
(page~3, left column, paragraph~3)



\vspace{1cm}
\noindent
\textcolor[rgb]{0.00,0.50,1.00}{\textbf{Major Comment~4.}}
\emph{Physical Model Validation:
The SCAPS-1D simulations use fixed FeB parameters
(binding energy, migration energy, and pre-exponential factors).
Yet, these parameters vary widely in literature (0.55–0.69 eV for migration energy).
Without sensitivity analysis or error quantification,
the generated synthetic dataset may not reflect realistic kinetics.
Validation against first-principles or experimental benchmarks would strengthen the study.}


\noindent
\textcolor[rgb]{0.51,0.00,0.00}{\textbf{Reply:}}


Allow us to say a few words  in favor of GaAs and 6H-SiC.


\vspace{1cm}
\noindent
\textcolor[rgb]{0.00,0.50,1.00}{\textbf{Major Comment~5.}}
\emph{Regression and Feature Interpretation:
Although multiple regressors (SVR, XGB, DNN, RF, GB) are compared,
no insight is given into the learned features or their physical correlation with iron concentration.
Incorporating explainable AI techniques (e.g., SHAP, PCA loading analysis)
would add interpretability to what the CNN features represent.}

\noindent
\textcolor[rgb]{0.51,0.00,0.00}{\textbf{Reply:}}


Allow us to say a few words  in favor of GaAs and 6H-SiC.


\vspace{1cm}
\noindent
\textcolor[rgb]{0.00,0.50,1.00}{\textbf{Major Comment~6.}}
\emph{Post-Hoc Correction:
The quadratic correction (Eq. 10) is an empirical adjustment
that artificially improves metrics but lacks theoretical justification.
The authors should either
(i) replace it with a physics-informed calibration (e.g., temperature or diffusion based scaling)
or (ii) clearly acknowledge its heuristic nature and limitations.}


\noindent
\textcolor[rgb]{0.51,0.00,0.00}{\textbf{Reply:}}


Allow us to say a few words  in favor of GaAs and 6H-SiC.


\vspace{1cm}
\noindent
\textcolor[rgb]{0.00,0.50,1.00}{\textbf{Major Comment~7.}}
\emph{Statistical Reporting:
The reported MSE, MAPE, and R$^2$ values are given without variance or confidence intervals.
Given the small datasets, reporting mean ± standard deviation across multiple random splits
would be essential to establish statistical robustness.}

\noindent
\textcolor[rgb]{0.51,0.00,0.00}{\textbf{Reply:}}


Allow us to say a few words  in favor of GaAs and 6H-SiC.

\vspace{1cm}
\noindent
\textcolor[rgb]{0.00,0.50,1.00}{\textbf{Minor Comment~1.}}
\emph{The introduction is overly broad;
it should focus more on ML for microscopic defects
rather than general PV or macro-defect analysis.}

\noindent
\textcolor[rgb]{0.51,0.00,0.00}{\textbf{Reply:}}


Allow us to say a few words  in favor of GaAs and 6H-SiC.


\vspace{1cm}
\noindent
\textcolor[rgb]{0.00,0.50,1.00}{\textbf{Minor Comment~2.}}
\emph{Some typographical errors exist (e.g., ``where where'' in Eq. 1).
Please proofread carefully.}


\noindent
\textcolor[rgb]{0.51,0.00,0.00}{\textbf{Reply:}}
We carefully reviewed the manuscript and corrected multiple typographical errors. 
Namely:

\noindent
- 
changed phrase ``Standard approaches to solving such problems involve the use of Fourier or wavelet transforms, and last were applied in this study''
to  ``Standard approaches to solving such problems involve the use of Fourier or wavelet transforms, and \textcolor[rgb]{1.00,0.07,0.00}{the latter}
were applied in this study'' on page~2;

\noindent
- changed ``where where'' to ``where'' after Eq.~(1);

\noindent
- changed ``A is the constant'' to ``A is the pre-exponential constant'' after Eq.~(3);

\noindent
- changed phase ``Panels (b) and (c) show the wavelet spectrograms corresponding to the curves with filled squares and open circles, respectively'' 
to ``Panels (b) and (c) show the wavelet spectrograms corresponding to the curves with \textcolor[rgb]{1.00,0.07,0.00}{open}  squares 
and \textcolor[rgb]{1.00,0.07,0.00}{filled} circles, respectively'' 
in the caption of Fig~3;

\noindent
- a sentence was added to clarify the applicability of the formula for the median absolute percentage error
``Eq.~(8) implies that  $\mathtt{MAPE}_i$ must be arranged in order of magnitude; '' after Eq.~(8)


\noindent
etc. 

\noindent
We hope that these revisions have resolved the issue as thoroughly as possible.




\vspace{1cm}
\noindent
\textcolor[rgb]{0.00,0.50,1.00}{\textbf{Minor Comment~3.}}
\emph{Figures should include axis units, consistent color scales, and indicate whether values
are in linear or logarithmic scale.}

\noindent
\textcolor[rgb]{0.51,0.00,0.00}{\textbf{Reply:}}


Allow us to say a few words  in favor of GaAs and 6H-SiC.


\vspace{1cm}
\noindent
\textcolor[rgb]{0.00,0.50,1.00}{\textbf{Minor Comment~4.}}
\emph{The Supplementary Figures (S1–S10) are repeatedly referenced
but insufficiently summarized in the main text.
A concise overview table would be helpful.}

\noindent
\textcolor[rgb]{0.51,0.00,0.00}{\textbf{Reply:}}


Allow us to say a few words  in favor of GaAs and 6H-SiC.


\vspace{1cm}
\noindent
\textcolor[rgb]{0.00,0.50,1.00}{\textbf{Minor Comment~5.}}
\emph{The data availability statement (``upon reasonable request'')
should be replaced with
a public repository link for transparency.}

\noindent
\textcolor[rgb]{0.51,0.00,0.00}{\textbf{Reply:}}


Allow us to say a few words  in favor of GaAs and 6H-SiC.


\vspace{1cm}
\noindent
\textcolor[rgb]{0.00,0.50,1.00}{\textbf{Minor Comment~6.}}
\emph{A comparison with simpler ML baselines
(e.g., direct regression on ISC(t) data without wavelet transformation) would
contextualize the improvement due to CV-based transfer learning.}

\noindent
\textcolor[rgb]{0.51,0.00,0.00}{\textbf{Reply:}}


Allow us to say a few words  in favor of GaAs and 6H-SiC.


\vspace{1cm}
\noindent
\textcolor[rgb]{0.00,0.50,1.00}{\textbf{Minor Comment~7.}}
\emph{References [6], [35], [38] should be verified for year and page accuracy.
Some reference formatting inconsistencies (journal abbreviations, italics) should be corrected.}

\noindent
\textcolor[rgb]{0.51,0.00,0.00}{\textbf{Reply:}}


Allow us to say a few words  in favor of GaAs and 6H-SiC.


\subsection*{Response to Reviewer \#2 }
\noindent
\textcolor[rgb]{0.00,0.50,1.00}{\textbf{Comment~1.}}
\emph{It is additionally essential to examine temperatures between 270 to 350 kelvin.}

\noindent
\textcolor[rgb]{0.51,0.00,0.00}{\textbf{Reply:}}


To our shame, the reviewer is correct about some fog in Results and discussion.
We hopefully rephrased the section to add sunlight.



\vspace{1cm}
\noindent
\textcolor[rgb]{0.00,0.50,1.00}{\textbf{Comment~2.}}
\emph{You should compare and review your manuscript with
other new articles such as ``Novel Design of Multi-Layer Cubic Nanoparticles for Achieving Efficient Thin-Film Perovskite Solar Cells''}

\noindent
\textcolor[rgb]{0.51,0.00,0.00}{\textbf{Reply:}}

The investigation did not show
the essential dependence of the influence of MW treatment on doping levels as well.


\vspace{1cm}
\noindent
\textcolor[rgb]{0.00,0.50,1.00}{\textbf{Comment~3.}}
\emph{Put the solar cell parameters in a table with references.}

\noindent
\textcolor[rgb]{0.51,0.00,0.00}{\textbf{Reply:}}
We added Table~1 to the revised manuscript (page~5). 
This table summarizes the parameters of the solar cell, silicon, and defect states at the temperature used in the main calculations.


\vspace{1cm}
\noindent
\textcolor[rgb]{0.00,0.50,1.00}{\textbf{Comment~4.}}
\emph{Actually, all solar cells have Rs and Rsh values.
By investigating parasitic losses on cell performance, the article could be made more interesting.}

\noindent
\textcolor[rgb]{0.51,0.00,0.00}{\textbf{Reply:}}
The text was revised.


\subsection*{Response to EDITOR REPORT}
\noindent
\textcolor[rgb]{0.00,0.50,1.00}{\textbf{REPORT.}}
\emph{We have found that your manuscript contains text which appears to have been replicated from the following published articles:}

\emph{www.sciencedirect.com/science/article/abs/pii/S0038092X25005171?via\%3Dihub}

\emph{Please reduce the level of overlap in your revised manuscript by rewriting the appropriate sections.}

\noindent
\textcolor[rgb]{0.51,0.00,0.00}{\textbf{Reply:}}


First, we apologize for the observed similarities. 
The cited article is our own and also addresses the determination of iron concentration in silicon solar cells. 
However, the approaches used in the cited work and in the present study are fundamentally different. 
In the former, regression models are employed that utilize changes in photoelectric parameters during the decay of FeB pairs, 
whereas in the present manuscript, the primary approach involves converting the kinetic dependencies of the short-circuit current 
into images and extracting features using computer vision models. 
Nevertheless, both studies use similar solar cell models, apply standard regression algorithms and metrics, 
and perform testing on experimental samples from the same batch. This explains certain similarities in wording.

We have revised the relevant sections and reduced the extent of overlap.


\cite{Wijaranakula}


\section*{References}

\bibliographystyle{iopart-num}
\bibliography{olikh}

\end{document}

