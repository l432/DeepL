


\documentclass[preprint]{elsarticle}

\begin{document}


To:
Solar Energy Editorial Board


Subject:
Manuscript Submission

\vspace{5mm}
Dear Editors,

\vspace{3mm}
%Enclosed is a manuscript entitled “XXX XXXX XXXXX” by sb, which we are submitting for publication in the journal of …. We have chosen this journal because it deals with …

Enclosed with this letter you will find the electronic submission of manuscript entitled
``Extracting the iron concentration in silicon solar cells using photovoltaic parameters and machine learning'' by Oleg Olikh and Oleksii Zavhorodnii.

The integration of machine learning (ML) technology is becoming increasingly widespread across various domains of human activity, 
and photovoltaics is no exception. 
Thus, researchers have extensively applied ML techniques to analyze electroluminescence and photoluminescence images of solar cells, 
facilitating the detection and classification of defects such as cracks, finger failures, hot spots, scratches, and dislocations. 
However, impurity quantification has received relatively little attention.

In this study, we propose a ML-based approach for extracting information about iron contamination in silicon solar cells from current-voltage measurements. 
This manuscript presents the results of a comparative analysis of 80 different models, 
which vary in terms of the ML algorithm used, feature extraction conditions, and their combinations, as well as the application of data preprocessing techniques. 
Testing was conducted on both simulated data and experimentally obtained data.
Unlike traditional non-ML techniques, the proposed method does not rely on complex equipment, 
enabling faster, non-destructive, and inline-compatible analysis of silicon solar cells. 
This approach has the potential to offer a more efficient and cost-effective solution for impurity quantification in photovoltaic devices.
We believe this work will be highly relevant to researchers focused on photovoltaic device characterization methods and the optimization of solar cell manufacturing processes.
Given the growing interest in renewable energy technologies and machine learning,
we are confident that our findings will resonate with \emph{Solar Energy} readership and stimulate further exploration in this area.


This manuscript is original and has not been submitted elsewhere, in whole or in part.
No aspects of this work have been published in any form.
There are no conflicts of interest related to this submission.


We would  very much appreciate if you would consider the manuscript for publication in the \emph{Solar Energy}.


\vspace{3mm}

Sincerely yours,

Oleg~Olikh and Oleksii Zavhorodnii


Taras Shevchenko National University of Kyiv


Kyiv 01601, Ukraine

E-mail: olegolikh@knu.ua


%Dear Editors
%It is more than 12 weeks since I submitted our manuscript (No: ) for possible publication in your journal. I have not yet received a reply and am wondering whether you have reached a decision. I should appreciated your letting me know what you have decided as soon as possible.








\end{document}

