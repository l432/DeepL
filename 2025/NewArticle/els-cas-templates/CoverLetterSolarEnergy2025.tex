


\documentclass[preprint]{elsarticle}

\begin{document}


To:
Solar Energy Editorial Board


Subject:
Manuscript Submission

\vspace{5mm}
Dear Editors,

\vspace{3mm}
%Enclosed is a manuscript entitled “XXX XXXX XXXXX” by sb, which we are submitting for publication in the journal of …. We have chosen this journal because it deals with …
Enclosed with this letter you will find the electronic submission of manuscript entitled
``Iron quantification in silicon solar cell by machine learning'' by Oleg Olikh and Oleksii Zavhorodnii

Although researchers have widely applied machine learning (ML) to image-based defect detection in photovoltaics, its application to impurity quantification from $I$-$V$ measurements has received little attention.

 
Our method differs from traditional techniques that rely on complex setups like deep-level transient spectroscopy or photoluminescence imaging. Instead, we extract iron contamination concentration using only standard $I$-$V$ measurements, making our approach highly accessible and applicable to industrial settings. Compared to conventional defect characterization techniques, our method is non-destructive and enables a faster, inline-compatible analysis of silicon solar cells. 


By benchmarking several ML models—Random Forest, Gradient Boosting, Support Vector Regression, eXtreme Gradient Boosting, and Deep Neural Networks—we demonstrate that our framework can outperform classical regression techniques in predictive accuracy. 


We believe this work will be highly relevant to researchers focused on advanced photovoltaic characterization methods and the optimization of solar cell manufacturing processes. 


Given the growing demand for scalable defect characterization in photovoltaic manufacturing, we believe this work will be highly relevant to \emph{Solar Energy} readers.


This manuscript is original and has not been submitted elsewhere, in whole or in part. 
No aspects of this work have been published in any form. 
There are no conflicts of interest related to this submission.


We would  very much appreciate if you would consider the manuscript for publication in the \emph{Solar Energy}.
%We appreciate your consideration of our manuscript, and we look forward to receiving comments from the reviewers.

%Possible reviewers are the following.
%
%\begin{itemize}
%  \item Yimin Zhang,
%Shenyang University of Chemical Technology,
%Equipment Reliability Institute, Shenyang 110142, China,
%ymzhang@mail.neu.edu.cn
%  \item Kang Li,
%University of Leeds,
%School of Electronic and Electrical Engineering,  Leeds, LS2 9JT, UK,
%k.li1@leeds.ac.uk
%  \item Rebecca Saive,
%University of Twente Institute for Nanotechnology, Enschede 7522 NB, The Netherlands,
%r.saive@utwente.nl
%  \item Ripon Chakrabortty,
%University of New South Wales,
%School of Engineering and IT, Canberra, Australia,
%r.chakrabortty@adfa.edu.au
%%  \item Belen Arredondo,
%%Universidad Rey Juan Carlos
%%Área de Tecnología Electrónica, C/ Tulipán s/n, 28933 Móstoles, Spain,
%%belen.arredondo@urjc.es
%\end{itemize}


\vspace{3mm}

Sincerely yours,

Oleg~Olikh


Taras Shevchenko National University of Kyiv


Kyiv 01601, Ukraine

E-mail: olegolikh@knu.ua


%Dear Editors
%It is more than 12 weeks since I submitted our manuscript (No: ) for possible publication in your journal. I have not yet received a reply and am wondering whether you have reached a decision. I should appreciated your letting me know what you have decided as soon as possible.








\end{document}

