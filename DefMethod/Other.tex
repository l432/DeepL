\chapter{Короткий огляд інших методів}\label{chapOther}

Звичайно, п'ятьма методами, більш--менш детально описаними
у попередніх частинах, перелік експериментальних методик,
спрямованих на визначенння параметрів дефектів,
 не обмежується.
Кількість різних підходів до вирішення такого важливого завдання, як
характеризація дефектів, достатньо велика і постійно зростає.
Описати, та й навіть згадати, їх всі в одному
виданні розумного обсягу практично не можливо.
Тому надалі буде наведений короткий перелік найпоширеніших з них.

Психологічно чи не найцікавішими методами є ті, 
які дозволяють побачити чи отримати зображення
дефектів.
Проте аж до початку 80--х років минулого століття
переконатися на власні очі можна було, фактично, лише в існуванні
дислокацій.
Наприклад, серед оптичних методів спостереження одномірних порушень періодичності
найширше використовувалися 
\begin{itemize}
  \item \emph{вибіркове травлення}, пов'язане з утворенням під дією хімічних реагентів 
  на поверхні кристалу у місці виходу дислокаційних ліній
  характерних фігур 
  та \emph{проекційне травлення}, що дозволяло отримувати інформацію про особливості
  мікрорельєфу шару товщиною аж до декількох сотень мікрометрів;
  \item декорування, яке передбачало створення навколо дислокаційних ліній
   значних за розмірами скупчень домішкових атомів (Cu, Au, Li тощо);
  \item \emph{поляризаційно--оптичний метод} (або \emph{метод фотопружності}),
  який ґрунтується на ефекті виникнення подвійного променезаломлення
  під дією пружних напруг дислокацій з краєвою компонентою. 
\end{itemize}
Для спостереження дислокацій також використовується просвічуюча \emph{електронна
мікроскопія} та ряд рентгенотопографічних методів.
Зокрема виділяють 
\emph{метод Берга--Баррета}, де аналізується відбитий пучок;
\emph{метод Ленга}, в якому монохроматичний промінь має проходити через кристал
та є можливість отримувати стереопару використавши відбивання $(h,k,l)$ та $(\bar{h},\bar{k},\bar{l})$;
і \emph{метод Бормана}, пов'язаний зі спостереженням аномального проходження рентгенівських хвиль.
Цікавим підходом також є \emph{метод каналювання} (іонів чи нейтронів), в якому
аналізується рух частинок у кристалі, який супроводжується лише ковзними зіткненнями.

Протягом довгого періоду часу практично єдиним способом безпосереднього 
спостереження точкових дефектів був \emph{метод автоіонної мікроскопії}.
При цьому зображення формувалося за допомогою іонів газу (найчастіше гелію),
утворених у сильному електричному полі поблизу поверхні зразка;
особливості розподілу напруженості поля віддзеркалювали рельєф кристалу.
Все змінилося з появою зондових методів досліджень:
\emph{скануюча тунельна мікроскопія}, \emph{атомно--силова мікроскопія},
\emph{електросилома мікроскопія}, \emph{магніто-силова мікроскопія},
\emph{оптична мікроскопія ближнього поля} характеризуються 
мають роздільну здатність, що дозволяє вирізняти окремі вакансії чи міжвузольні атоми.

Ще одним прямим способом дослідження точкових дефектів є 
\emph{метод прецизійного вимірювання густини}, існування якого пов'язане
з впливом на вказаний параметр наявності значної кількості власних дефектів.

Проте найчастіше дефекти вивчаються за їхнім впливом на різноманітні властивості кристалів.
Наприклад, у \emph{методі релаксації фотопровідності} передбачається відбувається
аналіз спадної ділянки залежності провідності напівпровідника від часу після
генерації надлишкових носіїв заряду внаслідок лазерного опромінення.
Сама величина провідності нерідко визначається за допомогою ефекту Холла.
До речі, окремим методом вважається
дослідження \emph{температурних залежностей сталої Холла}, яка безпосередньо пов'язана з
концентрацією носіїв заряду.
При цьому можна визначити як концентрацію дефектів, так і 
енергетичне положення рівнів дефектів у забороненій зоні: для характеризації глибоких рівнів
необхідно проводити дослідження починаючи з азотних температур, 
а для вивчення мілких --- з гелієвих.
Зокрема, запропоновано графічний метод аналізу температурної залежності концентрації основних носіїв,
який називається \emph{спектроскопія концентрації вільних носіїв} (free carrier 
concentration Spectroscopy, FCCS)).


Ряд методик використовують для характеризації нульвимірних порушень періодичності акустичні хвилі.
Наприклад, у \emph{методі акустоелектричної нестаціонарної спектроскопії} проводиться аналіз
релаксації поперечної акустоелектричної напруги (ПАН), що виникла внаслідок захоплення носіїв
дефектами у приповерхневому шарі напівпровідникового кристалу.  
Інший варіант цього методу полягає у вивченні залежності величина ПАН від довжини хвилі додаткового оптичного освітлення.
\emph{Метод акустофотопровідності} ґрунтується на порівнянні залежностей провідності 
напівпровідникового зразка від частоти падаючого світла при відсутності та за наявності 
акустичних хвиль.

До речі, збурення системи можна проводити не лише за допомогою пружних полів,
і це лежить в основі цілого сімейства \emph{методик диференційної спектроскопії},
в яких проводиться дослідження різницевих спектрів відбивання, пропускання чи поглинання світла.
Не відходячи далеко від процесів поглинання фотонів,
згадаємо \emph{оптичну абсорбційну спектроскопію}  (порівнюються спектральні залежності
коефіцієнтів поглинання досліджуваного та опорного  зразків)
та \emph{інфрачервону спектроскопію} (базується на тому, що власні частоти коливань 
дефектів у ґратці відповідають саме такому спектральному діапазону, 
а отже можливі фотонно--стимульовані переходи,
частоти яких однозначно пов'язані з типом порушення періодичності).

\emph{Люмінесцентні методи} дозволяють досліджувати структурні дефекти, 
з якими пов'язані випромінювальні переходи.
Методи відрізняються, насамперед, шляхом збудження світіння.
Наприклад, воно може виникати внаслідок дії електричного поля 
(завдяки інжекції носіїв через омічні контакти, їхнього лавинного розмноження чи тунелювання),
зовнішнього світла (фотолюмінесценція),
електронного променя (катодолюмінесценція),
тепла (термолюмінесценція),
механічного навантаження (триболюмінесценція) тощо.
На практиці найчастіше використовується фото--, катодо-- та інжекційна люмінесценції.
Щонайперше, об'єктом дослідження є спектральна залежність інтенсивності
випромінювання.
Проте додаткову інформацію можна отримати шляхом вивчення
 кінетики згасання люмінесценції та особливостей її гасіння.
Симетрійні властивості центрів можна отримати за допомогою так званих
поляризаційних діаграм --- залежностей ступеню
поляризації випромінювання від кута між вектором напруженості 
електричного поля у збуджуючому промені та віссю кристалу.
До люмінесцентних теж відноситься \emph{метод оптичної реєстрації магнітного резонансу},
який ґрунтується на явищі впливу переходів між зееманівськими підрівнями,
які відповідають збудженому стану дефекту, на інтенсивність та поляризацію випромінювання.


Наявність надзвичайно великої кількості методів дослідження дефектів свідчить 
як про невгамовність пошуку дослідників, так і про важливість задачі характеризації дефектів.