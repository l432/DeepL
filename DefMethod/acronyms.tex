\chapter*{Перелік умовних скорочень та позначень}             % Заголовок
\addcontentsline{toc}{chapter}{Перелік умовних позначень та скорочень}  % Добавляем его в оглавление
\noindent
%\begin{longtabu} to \dimexpr \textwidth-5\tabcolsep {r X}
\begin{longtabu} to \textwidth {r X}
%  CDLR& coupled defect level recombination,  рекомбінація у системі спарених рівнів дефектів\\
  DLTS & deep--level transient spectroscopy, перехідна спектроскопія локальних рівнів\\
%  DAT & defect--assisted tunneling, тунелювання за участю рівнів дефектів \\
 % DE & differential evolution, метод диференційної еволюції \\%
EPR& electron paramagnetic resonance, електронний парамагнітний резонанс\\
ENDOR & electron nuclear double resonance, подвійний електронно--ядерний резонанс\\
%  FRC & fast--formed recombination center, швидко сформовані ВО дефекти \\
 % NIEL & non--ionizing energy losses, втрати, не пов'язані з іонізацією \\
%  MABC & modified artificial bee colony, метод  штучної бджолиної сім'ї\\
%  OSFR & oxidization induced stacking--faults ring, кільцеві дефекти пакування, що виникли при окисненні \\
%  PAT & phonon-assisted tunneling, стимулюване фононами тунелювання \\
%  PSO & particle swarm optimization, метод оптимізації зграї частинок\\
PAS & positron annihilation spectroscopy, позитронно--анігіля\-цій\-на спектроскопія \\
%  RT & running time, час, необхідний для визначення параметрів\\
%  SCLC & space-charge limited current, струм, обмеженим просторовим зарядом \\
 % SRC & slow--formed recombination center, повільно сформовані ВО дефекти\\
TSC & thermally stimulated current, термостимульований струм\\
%  TLBO & teaching learning based optimization, метод  оптимізованого викладання та навчання\\
%  VRHC &thermally-assisted variable-range-hopping conduction, термічно--активована стрибкова провідність зі змінною довжиною стрибка \\
%  AAД & акустоактивний дефект\\
%  АДВ & акусто--дефектна взаємодія \\
%  АІ & акусто--індукований\\
%  АХ & акустична хвиля\\
%  АЧХ & амплітудно--частотна характеристика\\
%  ВАХ & вольт--амперна характеристика\\
%  ВБШ & висота бар'єру Шотткі\\
%  ВТКС & високотемпературна компонента струму\\
%  ВФХ & вольт--фарадна характеристика\\
%  ГР &глибокий рівень \\
%  ДШ & діод Шотткі\\
%  ЕА & еволюційний алгоритм\\
%  КНО &  квазі--нейтральна область \\
%  КП & кисневмісні преципітати\\
%  КСЕ & кремнієвий сонячний елемент\\
%  MH & метал---напівпровідник \\
%  MОH & метал---окис---напівпровідник \\
%  МХО & мікро--хвильова обробка\\
%  НВЧ & надвисокочастотне \\
%  НТКС & низькотемпературна компонента струму\\
  ОПЗ & область просторового заряду \\
%  ПАН & поперечна акустоелектрична напруга\\
%  ПЕ & польова емісія\\
%  ППЗ & поперечний переріз захоплення \\
%  РД & радіаційний дефект \\
%  ТД &точковий дефект \\
%  ТЕ & термоелектронна емісія \\
%  ТПЕ & термопольова емісія \\
%  УЗ & ультразвук \\
%  УЗН & ультразвукове навантаження \\
%  УЗО & ультразвукова обробка \\
%  ШРХ & теорія Шоклі--Ріда--Хола  \\
$\alpha_+$ & коефіцієнт поглинання позитронів\\
$\alpha_{\sigma}$ & показник ступеня температурної залежності поперечного перерізу захоплення  \\
%$\alpha_R$ & температурний коефіцієнт опору\\
%$\alpha_\mathrm{\,FB}$ & температурний коефіцієнт ВБШ в наближені плоских зон\\
$\beta^+$ & позитрон  \\
%$\beta_1$, $\beta_2$  & коефіцієнти Варшні  \\
%$\Delta P$ & абсолютна АІ зміна параметра $P$\\
$\Delta C$ & надлишкова ємність безпосередньо після зміни прикладеної напруги\\
$\delta C$ & сигнал DLTS\\
$\delta_{ij}$ & символ Кронекера\\
$\gamma$ & гамма--квант \\
$\gamma_g$ & відношення кратностей виродження станів дефекту до та після захоплення електрону  \\
$\gamma_p$ & відношення кратностей виродження станів дефекту до та після захоплення дірки  \\
$\varepsilon$ & діелектрична проникність матеріалу  \\
$\varepsilon_0$ & діелектрична стала \\
%$\varepsilon_P$ & відносна AI зміна параметра $P$\\
$\eta_{+,b}$ & парціальна частка позитронів, які анігілювали у ґратці\\
$\eta_{+,\mathrm{V}}$ & парціальна частка позитронів, які анігілювали  в околі вакансійних
дефектів \\
$\xi$ & напруженість електричного поля\\
%$\xi_\mathtt{cur}$ & відносна деформація приповерхневих кристалічних площин\\
%$\xi_\mathtt{US}$& амплітуда деформації ґратки при поширенні УЗ\\
%$\vartheta$ & темп генерації РД\\
%$\lambda$ &довжина хвилі падаючого світла\\
$\lambda_+$& темп анігіляції позитронів\\
$\lambda_{+,b}$& темп анігіляції позитронів у бездефектній області кристалу\\
$\lambda_{+,\mathrm{V}}$& темп анігіляції позитронів в околі вакансії\\
$\lambda_d$ & ширина області заряджених глибоких рівнів у ОПЗ\\
$\hat{\vec{\mu}}_L$ & оператор повного магнітного орбітального моменту\\
$\hat{\vec{\mu}}_S$ & оператор повного магнітного спінового моменту\\
$\mu_B$ & магнетон Бора\\
$\nu$ & частота падаючого світла \\
$\nu_e$& нейтрино \\
%%$\rho_\mathtt{LNO}$ & густина ніобату літію\\
%%$\rho_\mathtt{Si}$ & густина кремнію\\
$\rho_e$ & електронна густина\\
$\rho_N$ & концентрація ядер \\
$\varrho$&густина матеріалу\\
$\mu_t$ & коефіцієнт захоплення позитронів дефектом\\
%$\sigma_{\Phi0}$ & стандартне відхилення висоти бар'єру при нульовому зміщенні\\
$\sigma_{0}$& незалежний від температури множник у поперечному перерізі захоплення носіїв\\
$\sigma_{n(p)}$& поперечний переріз захоплення електронів (дірок) дефектом\\
%%$\sigma_p$& поперечний переріз захоплення дірок дефектом\\
$\tau_+$ & час життя позитронів\\
$\tau_{+,b}$& час життя позитронів у бездефектній області кристалу\\
$\tau_{+,\mathrm{V}}$& час життя позитронів у околі вакансії\\
$\left\langle\tau_+\right\rangle$ &середній час життя позитронів\\
%%$\tau$ & час релаксації заряду на пастках\\
%$\tau_{g}$ &ефективний час життя носіїв заряду в ОПЗ\\
%$\tau_{n}$ &ефективний час життя електронів\\
%$\tau_{n,\mathtt{RD}}$ & час життя електронів при рекомбінації на РД\\
%%$\upsilon_\mathtt{LNO}$ & швидкість звуку в ніобаті літію\\
$\upsilon_{th,n(p)}$& теплова швидкість електронів (дірок)\\
%%$\upsilon_{\mathrm{th},n}$ & теплова швидкість електронів\\
%%$\upsilon_{\mathrm{th},p}$ & теплова швидкість дірок\\
%%$\upsilon_\mathtt{Si}$ & швидкість звуку в кремнії\\
$\Phi_b$ & світловий потік\\
%$\Phi_b$ & ВБШ при нульовому зміщенні\\
%$\Phi_{b}^0$ & середнє значення ВБШ при нульовому зміщенні (ВБШ в однорідній області) \\
%$\Phi_{b}^\mathrm{FB}$ & ВБШ в наближені плоских зон \\
$\chi$& магнітна сприйнятливість \\
%$\Psi$ & флюєнс опромінення\\
$\psi_+$ & хвильова функція позитрону\\
%$\phi_0$ & рівень нейтральності інтерфейсних станів у структурі МН\\
%%$\zeta$ & диференційний показник нахилу ВАХ \\
$\omega$ & циклічна частота електромагнітної хвилі \\
$\omega_0$ & резонансна частота в EPR \\
%%$\omega_{ph}$ & частота фонону\\
%%$\omega_\mathtt{US}$ & циклічна частота АХ\\
%$A$ & площа зразка \\
%%$A_\mathtt{LNO}$ & площа п'єзоперетворювача\\
%$A^*$ & ефективна стала Річардсона \\
$\tilde{A}$ & тензор надтонкої взаємодії\\
$A_{LS}$ & стала спін--орбітальної взаємодії\\
$a_T$ & швидкість нагріву \\
$\vec{B}$ & вектор індукції магнітного поля\\
%%$a$ & стала ґратки \\
%%$a_B$ & радіус Бора\\
%%$B$ & коефіцієнт динамічної в'язкості \\
%%$b$ & модуль вектора Бюргерса \\
$C$ & ємність бар'єрної структури\\
$c$ & швидкість світла\\
$c_{n(p)}$ & швидкість захоплення вільних електронів (дірок) дефектом\\
$c_+$&швидкість захоплення позитронів дефектом \\
$c_{+,\mathrm{V}}$&швидкість захоплення позитронів дефектом вакансійного типу\\
%$D$ & доза опромінення\\
%%$D_d$ & displacement damage dose, ефективна доза, пов'язана з дефектоутворенням\\
%$D_{ss}$ & густина інтерфейсних станів у структурі МН\\
$d$ & ширина області спустошення дефектів в ОПЗ\\
$E$ & енергія електрону \\
$E_{+}$ & енергія позитрону\\
$E_{+,m}$ & максимальна енергія спектру позитронів\\
$E_{\gamma}$ & енергія гамма--кванту\\
$E_{\sigma}$ & активаційна енергія поперечного перерізу захоплення \\
$E_C$ & енергія дна зони провідності \\
$E_D$ & положення енергетичного рівня донорної домішки\\
$E_G$ & ширина забороненої зони\\
$E_F$ & енергія Фермі\\
%$E_i$ & положення рівня Фермі у власному напівпровіднику\\
$E_t$ & положення енергетичного рівня, зв'язаного з дефектом\\
$E_{t+}$ & енергія позитронної іонізації дефекту\\
%$F\!F$ & фактор форми КСЕ\\
%$F_m$ & напруженість електричного поля на межі розділу МН \\
$E_V$ & енергія стелі валентної зони \\
$e^{-}$ & електрон \\
$e_+$ & швидкість емісії позитронів\\
$e_{n(p)}$ &швидкість термічної емісії електронів (дірок) дефектом\\
$e_n^o$& темп оптичної емісії електрону з глибокого рівня \\
%%$f_r$& резонансна частота п'єзоперетворювача\\
%$f_\mathtt{US}$& частота УЗ\\
$f_t$ & ймовірність заселеності електронного рівня\\
%%$G$ & модуль зсуву \\
$g$ & кратність квантовомеханічного виродження стану\\
$\tilde{g}$&тензор Ланде\\
$g_0$ & гіромагнітний фактор\\
%$\hat{H}$ & оператор Гамільтона\\
$h$, $\hbar$ & стала Планка\\
$h^+$ & дірка\\
$\hat{\vec{I}}$ & оператор спіну  ядра\\
%$I$ & струм\\
%$I_s$ & струм насичення\\
%$I_R$ & зворотний струм\\
%%$J$ & густина струму\\
%$J_{ph}$ & густина фотогенерованого струму\\
%$J_{sc}$ & густина струму короткого замикання\\
$k$ & стала Больцмана\\
$\hat{\vec{L}}$ & оператор повного орбітального моменту\\
%$L_n$ & довжина дифузії електронів\\
$m_0$&маса спокою електрону\\
$m_+^*$& ефективна маса позитрону\\
$m_{n(p)}^*$ &  ефективна маса електрону (дірки)\\
$m_S$ & спінове число\\
$N_t$ & концентрація дефектів \\
$N_C$ & ефективна густина станів біля дна зони провідності\\
$N_D$ & концентрація донорів\\
%$N_d$ & концентрація електронів поблизу контакту МН\\
%$N_{t,\mathtt{RD}}$ & концентрація радіаційних дефектів\\
$N_V$ & ефективна густина станів біля вершини валентної зони\\
%$n_i$ & концентрація власних носіїв заряду\\
$n$ & концентрація електронів\\
$n_1$ &концентрація електронів у зоні провідності, коли рівень Фермі
співпадає з рівнем дефекту\\
$n_e$& кількість електронів в околі дефекту\\
%$n_\mathrm{id}$ & фактор неідеальності\\
%$n_{n(p)}$ & концентрація електронів у електронному (дірковому) напівпровіднику \\
%%$n_n$ & концентрація основних носіїв у електронному напівпровіднику \\
%%$n_p$ & концентрація неосновних носіїв у дірковому напівпровіднику \\
%%$q$ & елементарний заряд\\
$P_{abs}$ & поглинута потужність електромагнітної хвилі\\
$P_L$&ймовірність поглинання падаючої частинки при проходженні нею одиничного шляху\\
$p$ & концентрація дірок \\
$p^+(t)$ & частка  позитронів, які ще не проанігілювали в момент часу $t$\\
$p_{b}^+$ & частка вільних позитронів\\
$p_{t}^+$ & частка позитронів, захоплених дефектами\\
$p_{\mathrm{V}}^+$ & частка позитронів, захоплених дефектом вакансійного типу\\
$p_1$ &концентрація електронів у зоні провідності, коли рівень Фермі
співпадає з рівнем дефекту\\
%$p_{n(p)}$ & концентрація дірок у електронному (дірковому) напівпровіднику \\
%%$p_n$ & концентрація неосновних носіїв у електронному напівпровіднику \\
%%$p_p$ & концентрація основних носіїв у дірковому напівпровіднику \\
$Q^g$ & узагальнена координати\\
$Q$ & об'ємний заряд\\
$q$ & елементарний заряд\\
%%$R$ & темп рекомбінації \\
%%$R_\mathtt{cur}$ & радіус кривизни зразка \\
%$R_{\mathtt{DA}}$ & параметр зв'язку у моделі CDLR\\
%$R_{ph}$ & коефіцієнт відбивання світла\\
%$R_s$ & послідовний опір\\
%$R_{sh}$ & опір шунтування\\
$S_+$ & valence annihilation parameters\\
$\hat{\vec{S}}$ & оператор повного спінового моменту\\
$T$ & абсолютна температура\\
$T_{0}$ &нижня межа температурного діапазону досліджень \\
$T_{1/2}$ &період напіврозпаду \\
$T_{m}$ & абсциса максимуму температурної залежності  \\
%$T_0$ & константа температурної залежності фактора неідеальності\\
%%$T_\mathtt{US}$ & період АХ\\
$t$ & час\\
$t_p$ & тривалість імпульсу заповнення\\
%$t_\mathtt{MWT}$ & час експозиції при МХО\\
%$t_\mathtt{UST}$ & час експозиції при УЗО\\
%$u_\mathtt{US}$&амплітуда зміщень атомів при поширенні УЗ\\
$V$ & напруга\\
$V_{bi}$ & контактна різниця потенціалів\\
%$V_{bb}$ & вигин зон напівпровідника поблизу контакту\\
%%$V_d$ & падіння напруги в околі бар'єру\\
%%$V_n$ & різниця потенціалів між дном зони провідності та положенням рівня Фермі в об'ємі напівпровідника\\
%$V_{oc}$ & напруга холостого ходу\\
%$V_R$ & зворотна напруга\\
%%$V_\mathtt{TAV}$ & величина ПАН\\
%$V_\mathtt{RF}$ & амплітуда напруги, прикладеної до п'єзоперетворювача\\
%%$V_v$ & об'єм кристалу\\
$W$ & ширина області просторового заряду \\
$W_+$ & core annihilation parameters\\
%$W_{ph}$ & інтенсивність освітлення \\
%$W_\mathtt{US}$ & інтенсивність акустичної хвилі\\

\end{longtabu}
\addtocounter{table}{-1}% Нужно откатить на единицу счетчик номеров таблиц, так как предыдующая таблица сделана для удобства представления информации по ГОСТ





