\chapter*{Вступ}\label{chap0}
\addcontentsline{toc}{chapter}{Вступ}	

З самого початку розвитку напівпровідникової електроніки добре відомо, що наявність
різноманітних дефектів є ключовим чинником, що визначає функціональні властивості пристроїв.
Це, насамперед, пов'язано з тим, що наявність дефектів викликає зміну густини енергетичних станів
у дозволених зонах напівпровідника, а також
появу локальних рівнів у забороненій зоні.
Останні можуть виступати у ролі центрів прилипання, рекомбінації (як безвипромінювальної, так і випромінювальної) та
слугувати джерелами збільшення вільних носіїв зарядів.
Далеко не завжди кристалічні дефекти спричинюють негативні чи небажані зміни властивостей;
яскравим прикладом протилежного впливу є введення легуючих домішок заміщення, що дозволяє варіювати
провідність матеріалу.
Проте очевидно, що для пояснення та передбачення властивостей напівпровідникових кристалів та приладів
на їхній основі необхідно мати інформацію про наявні дефекти, а відповідні методи
моніторингу є важливою складовою успішних технологічних процесів та наукових досліджень.

Основними параметрами дефектів є наступні:
%\begin{enumerate}[label=\asbuk*),leftmargin=0em,itemindent=1.5em]
\begin{enumerate}[label=\arabic*),leftmargin=0em,itemindent=1.5em]
\item тип, тобто атомна структура та конфігурація (місцеперебування компонент у кристалічній ґратці);
\item електронна структура, зокрема зарядність;
\item концентрація ($N_t$) та просторовий розподіл;
\item положення енергетичних рівнів ($E_t$) та пов'язана з цим енергія іонізації (оптична, термічна);
\item перерізи захоплення вільних носіїв заряду, тобто електронів ($\sigma_n$) та дірок ($\sigma_p$);
\item механізми дифузії та величини відповідних коефіцієнтів;
\item механізми утворення та відповідна енергія (ентальпія);
\item симетрія, тобто набір операцій симетрії, які властиві кристалу з даним порушенням періодичності;
\item механізми розпаду, зокрема взаємодії з іншими порушеннями кристалічної ґратки та відповідні кількісні параметри;
\item оптичні властивості, такі як перерізи фотоіонізації, випромінювального захоплення;
ймовірності внутрішньоцентрової люмінесценції тощо;
\item функціональність (центр рекомбінації, прилипання, розсіяння...);
\end{enumerate}

Для визначення цих властивостей розроблено чимало експериментальних методик.
Більшість з них дозволяють отримати інформацію лише про певні характеристики чи про обмежену низку  параметрів
і тому для всебічного вивчення дефекту необхідно проводити цілий комплекс досліджень,
переважно досить складних та громіздких.
Як наслідок, повна інформація відома лише для окремих дефектів.
Це стосується навіть кремнію, хоча цей матеріал вважається достатньо добре вивченим.

У цьому посібнику розглянуто лише деякі з експериментальних методів, які дозволяють досліджувати дефекти.
Додаткову інформацію щодо як розглянутих методів, так і низки інших можна знайти в
 \cite{tuomisto2019,Stavola,ThoricBook,Lanno,Schroder2006,Peaker,Mironov,ReinLS,PAS,LockInThermography,ACMMR10,Eliseev,Krause}.
